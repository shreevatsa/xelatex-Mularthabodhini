% This file is part of Mūlārthabodhinī XeLaTeX Source Code

% Mūlārthabodhinī XeLaTeX Source Code is free software: you can redistribute it and/or modify it under the terms of the GNU General Public License as published by the Free Software Foundation, either version 3 of the License, or (at your option) any later version.

% Mūlārthabodhinī XeLaTeX Source Code is distributed in the hope that it will be useful, but WITHOUT ANY WARRANTY; without even the implied warranty of MERCHANTABILITY or FITNESS FOR A PARTICULAR PURPOSE. See the GNU General Public License for more details.

% You should have received a copy of the GNU General Public License along with Mūlārthabodhinī XeLaTeX Source Code. If not, see <https://www.gnu.org/licenses/>.

{\bfseries
\setlength{\mylenone}{0pt}
\settowidth{\mylentwo}{भक्तान् भक्तिं रामभद्राचार्यो नत्वा हरिं गुरून्}
\setlength{\mylenone}{\maxof{\mylenone}{\mylentwo}}
\settowidth{\mylentwo}{श्रीभक्तमाले कुरुते टीकां मूलार्थबोधिनीम्}
\setlength{\mylenone}{\maxof{\mylenone}{\mylentwo}}
\setlength{\mylentwo}{\baselineskip}
\setlength{\mylenone}{\mylenone + 1pt}
\setlength{\mylen}{(\textwidth - \mylenone)*\real{0.5}}
\begin{longtable}[l]{@{\hspace*{\mylen}}>{\setlength\parfillskip{0pt}}p{\mylenone}@{}@{}l@{}}
 & \\[-\the\mylentwo]
भक्तान् भक्तिं रामभद्राचार्यो नत्वा हरिं गुरून् & ।\\ \nopagebreak
श्रीभक्तमाले कुरुते टीकां मूलार्थबोधिनीम् & ॥\\
\end{longtable}
}


{\bfseries
\setlength{\mylenone}{0pt}
\settowidth{\mylentwo}{जयति जगदघालं भग्नभक्ताधिजालं}
\setlength{\mylenone}{\maxof{\mylenone}{\mylentwo}}
\settowidth{\mylentwo}{हरिजनगुणमालं जुष्टराजत्तमालम्}
\setlength{\mylenone}{\maxof{\mylenone}{\mylentwo}}
\settowidth{\mylentwo}{विभुविरुदविशालं प्रेमपीयूषपालं}
\setlength{\mylenone}{\maxof{\mylenone}{\mylentwo}}
\settowidth{\mylentwo}{हरिहृदयरसालं भास्वरं भक्तमालम्}
\setlength{\mylenone}{\maxof{\mylenone}{\mylentwo}}
\setlength{\mylentwo}{\baselineskip}
\setlength{\mylenone}{\mylenone + 1pt}
\setlength{\mylen}{(\textwidth - \mylenone)*\real{0.5}}
\begin{longtable}[l]{@{\hspace*{\mylen}}>{\setlength\parfillskip{0pt}}p{\mylenone}@{}@{}l@{}}
 & \\[-\the\mylentwo]
जयति जगदघालं भग्नभक्ताधिजालं & \\ \nopagebreak
हरिजनगुणमालं जुष्टराजत्तमालम् & ।\\
विभुविरुदविशालं प्रेमपीयूषपालं & \\ \nopagebreak
हरिहृदयरसालं भास्वरं भक्तमालम् & ॥\\
\end{longtable}
}


{\bfseries
\setlength{\mylenone}{0pt}
\settowidth{\mylentwo}{प्रभू गौरश्यामौ विजितरतिकामौ तनुरुचा}
\setlength{\mylenone}{\maxof{\mylenone}{\mylentwo}}
\settowidth{\mylentwo}{विभू आत्मारामौ त्रिभुवनललामौ गुणनिधी}
\setlength{\mylenone}{\maxof{\mylenone}{\mylentwo}}
\settowidth{\mylentwo}{जनारामौ रामौ प्रथितपरिणामौ सुखकरौ}
\setlength{\mylenone}{\maxof{\mylenone}{\mylentwo}}
\settowidth{\mylentwo}{स्तुवे सीतारामौ जनदृगभिरामौ गिरिधरः}
\setlength{\mylenone}{\maxof{\mylenone}{\mylentwo}}
\setlength{\mylentwo}{\baselineskip}
\setlength{\mylenone}{\mylenone + 1pt}
\setlength{\mylen}{(\textwidth - \mylenone)*\real{0.5}}
\begin{longtable}[l]{@{\hspace*{\mylen}}>{\setlength\parfillskip{0pt}}p{\mylenone}@{}@{}l@{}}
 & \\[-\the\mylentwo]
प्रभू गौरश्यामौ विजितरतिकामौ तनुरुचा & \\ \nopagebreak
विभू आत्मारामौ त्रिभुवनललामौ गुणनिधी & ।\\
जनारामौ रामौ प्रथितपरिणामौ सुखकरौ & \\ \nopagebreak
स्तुवे सीतारामौ जनदृगभिरामौ गिरिधरः & ॥\\
\end{longtable}
}


\addcontentsline{toc}{section}{\texorpdfstring{पद १-४: मङ्गलाचरण एवं श्रीनाभाजीका परिचय}{१-४: मङ्गलाचरण एवं श्रीनाभाजीका परिचय}}

{\relscale{1.1875}
{\bfseries
\setlength{\mylenone}{0pt}
\settowidth{\mylentwo}{}
\setlength{\mylenone}{\maxof{\mylenone}{\mylentwo}}
\settowidth{\mylentwo}{भक्त भक्ति भगवंत गुरु चतुर नाम बपु एक}
\setlength{\mylenone}{\maxof{\mylenone}{\mylentwo}}
\settowidth{\mylentwo}{इनके पद बंदन किए नासहिं बिघ्न अनेक}
\setlength{\mylenone}{\maxof{\mylenone}{\mylentwo}}
\setlength{\mylentwo}{\baselineskip}
\setlength{\mylenone}{\mylenone + 1pt}
\setlength{\mylen}{(\textwidth - \mylenone)*\real{0.5}}
\begin{longtable}[l]{@{\hspace*{\mylen}}>{\setlength\parfillskip{0pt}}p{\mylenone}@{}@{}l@{}}
 & \\[-\the\mylentwo]
\centering{॥ १ \hspace*{-1.5mm}॥} & \\ \nopagebreak
भक्त भक्ति भगवंत गुरु चतुर नाम बपु एक & ।\\ \nopagebreak
इनके पद बंदन किए नासहिं बिघ्न अनेक & ॥
\end{longtable}
}
}
\fancyhead[LO,RE]{{\textmd{\Large १-४: मङ्गलाचरण व परिचय}}}
\begin{sloppypar}\justifying\hyphenrules{nohyphenation}
\textbf{मूलार्थ}—\textbf{भक्त} अर्थात् भगवान्‌के श्रीचरणारविन्दके अनुरागी भजकवृन्द, भगवान्‌की परमप्रेमा रूपिणी \textbf{भक्ति}, स्वयं षडैश्वर्यसंपन्न श्रीराम\-श्रीकृष्ण\-श्रीनारायणान्यतम \textbf{भगवान्}, और उनके तत्त्वका उपदेश करनेवाले \textbf{श्रीगुरुदेव}—ये चारों नाम और स्वरूपसे चार-चार दिखते हैं अर्थात् इनके पृथक्-पृथक् चार नाम हैं और पृथक्-पृथक् चार शरीर भी हैं। परन्तु वस्तुतः ये एक ही हैं, अर्थात् एक-दूसरेसे अभिन्न हैं, और एक परमेश्वर ही चार रूपोंमें हमें दिख रहे हैं। इनके श्रीचरणोंका वन्दन करनेसे अनेक विघ्न नष्ट हो जाते हैं। इसलिये मैं नारायणदास नाभा इन चारोंके श्रीचरणकमलोंका वन्दन कर रहा हूँ।
\end{sloppypar}

{\relscale{1.1875}
{\bfseries
\setlength{\mylenone}{0pt}
\settowidth{\mylentwo}{}
\setlength{\mylenone}{\maxof{\mylenone}{\mylentwo}}
\settowidth{\mylentwo}{मंगल आदि बिचारि रह बस्तु न और अनूप}
\setlength{\mylenone}{\maxof{\mylenone}{\mylentwo}}
\settowidth{\mylentwo}{हरिजन के जस गावते हरिजन मंगलरूप}
\setlength{\mylenone}{\maxof{\mylenone}{\mylentwo}}
\setlength{\mylentwo}{\baselineskip}
\setlength{\mylenone}{\mylenone + 1pt}
\setlength{\mylen}{(\textwidth - \mylenone)*\real{0.5}}
\begin{longtable}[l]{@{\hspace*{\mylen}}>{\setlength\parfillskip{0pt}}p{\mylenone}@{}@{}l@{}}
 & \\[-\the\mylentwo]
\centering{॥ २ \hspace*{-1.5mm}॥} & \\ \nopagebreak
मंगल आदि बिचारि रह बस्तु न और अनूप & ।\\ \nopagebreak
हरिजन के जस गावते हरिजन मंगलरूप & ॥
\end{longtable}
}
}
\begin{sloppypar}\justifying\hyphenrules{nohyphenation}
\textbf{मूलार्थ}—श्रीनाभाजी कहते हैं कि श्रीहरि भगवान्‌के भक्तोंके यशको गाते समय जब आदिमङ्गलका विचार किया गया तो यह निष्कर्ष निकला कि भगवान्‌के भक्तोंकी अपेक्षा और कोई दूसरी वस्तु अनुपम अर्थात् उत्कृष्ट है ही नहीं। अर्थात् भगवान्‌के भक्त ही स्वयं अनुपम हैं, उनका यशोगान अनुपम है। इसलिये भगवद्भक्तोंके यशोगानके प्रारम्भमें किसी और मङ्गलकी आवश्यकता नहीं है, क्योंकि भगवान्‌के भक्त स्वयं मङ्गलस्वरूप हैं।
\end{sloppypar}

{\relscale{1.1875}
{\bfseries
\setlength{\mylenone}{0pt}
\settowidth{\mylentwo}{}
\setlength{\mylenone}{\maxof{\mylenone}{\mylentwo}}
\settowidth{\mylentwo}{संतन निर्नय कियो मथि श्रुति पुरान इतिहास}
\setlength{\mylenone}{\maxof{\mylenone}{\mylentwo}}
\settowidth{\mylentwo}{भजिबे को दोई सुघर कै हरि कै हरिदास}
\setlength{\mylenone}{\maxof{\mylenone}{\mylentwo}}
\setlength{\mylentwo}{\baselineskip}
\setlength{\mylenone}{\mylenone + 1pt}
\setlength{\mylen}{(\textwidth - \mylenone)*\real{0.5}}
\begin{longtable}[l]{@{\hspace*{\mylen}}>{\setlength\parfillskip{0pt}}p{\mylenone}@{}@{}l@{}}
 & \\[-\the\mylentwo]
\centering{॥ ३ \hspace*{-1.5mm}॥} & \\ \nopagebreak
संतन निर्नय कियो मथि श्रुति पुरान इतिहास & ।\\ \nopagebreak
भजिबे को दोई सुघर कै हरि कै हरिदास & ॥
\end{longtable}
}
}
\begin{sloppypar}\justifying\hyphenrules{nohyphenation}
\textbf{मूलार्थ}—संतोंने चारों वेदोंका, अठारहों पुराणोंका, एवं श्रीरामायण तथा श्रीमहाभारत—इन दोनों इतिहासोंका आलोडन करके यह निर्णय कर लिया है कि भजन करनेके लिये दोनों ही श्रेष्ठ हैं—या श्रीहरिका भजन किया जाए या श्रीहरिके दासोंका भजन किया जाए (वस्तुतस्तु दोनोंका ही भजन करना अनिवार्य है, क्योंकि भगवद्भक्तोंके भजनसे भगवान् प्रसन्न होंगे और भगवान्‌के भजनसे भगवद्भक्त प्रसन्न होंगे)।
\end{sloppypar}

{\relscale{1.1875}
{\bfseries
\setlength{\mylenone}{0pt}
\settowidth{\mylentwo}{}
\setlength{\mylenone}{\maxof{\mylenone}{\mylentwo}}
\settowidth{\mylentwo}{(श्री)अग्रदेव आज्ञा दई भक्तन के जस गाउ}
\setlength{\mylenone}{\maxof{\mylenone}{\mylentwo}}
\settowidth{\mylentwo}{भवसागर के तरन को नाहिन और उपाउ}
\setlength{\mylenone}{\maxof{\mylenone}{\mylentwo}}
\setlength{\mylentwo}{\baselineskip}
\setlength{\mylenone}{\mylenone + 1pt}
\setlength{\mylen}{(\textwidth - \mylenone)*\real{0.5}}
\begin{longtable}[l]{@{\hspace*{\mylen}}>{\setlength\parfillskip{0pt}}p{\mylenone}@{}@{}l@{}}
 & \\[-\the\mylentwo]
\centering{॥ ४ \hspace*{-1.5mm}॥} & \\ \nopagebreak
(श्री)अग्रदेव आज्ञा दई भक्तन के जस गाउ & ।\\ \nopagebreak
भवसागर के तरन को नाहिन और उपाउ & ॥
\end{longtable}
}
}
\begin{sloppypar}\justifying\hyphenrules{nohyphenation}
\textbf{मूलार्थ}—नाभाजी कहते हैं कि मुझको मेरे सद्गुरुदेव श्रीअग्रदेव अर्थात् श्रीअग्रदासजीने यह आज्ञा दी—“हे नारायणदास नाभा! तुम भगवान्‌के भक्तोंका ही यश गाओ, क्योंकि भवसागरसे पार होनेके लिये और कोई दूसरा उपाय है ही नहीं। एकमात्र भगवद्भक्तोंका यशोगान ही भवसागरसे तरनेका उपाय है।”
\end{sloppypar}
\begin{sloppypar}\justifying\hyphenrules{nohyphenation}
श्रीनाभाजीके जीवनवृत्तके संबन्धमें एक महत्त्वपूर्ण, प्रेरणास्पद तथा रोचक प्रसिद्धि है। \textbf{हनुमान्‌वंश} अर्थात् श्रीहनुमान्‌जी द्वारा प्रचारित श्रीरामभक्तिकी परम्परामें श्रीनाभाजीका जन्म हुआ। वे जन्मना ब्राह्मण थे। जन्मसे ही नाभाजीके पास दोनों नेत्रोंके चिह्न भी नहीं थे। नाभाजी अत्यन्त दीन परिवारमें जन्मे थे और उनकी दृष्टिबाधित दशा और दरिद्रताको देखकर उनकी माताजीने अपने पञ्चवर्षीय अन्धबालकको दुष्कालसे पीड़ित होनेके कारण एक निर्जन वनमें छोड़ दिया था। अनाथ नाभाजी महाराज दृष्टिहीनताकी विडम्बनामें इतस्ततः भटक रहे थे। संयोगसे वहाँसे निकल पड़े थे श्रीपतितपावन पयहारीजी श्रीकृष्णदासजीके अनन्य कृपापात्र युगलसंतचरण—श्रीकील्हदासजी एवं श्रीअग्रदासजी। उन दोनों संतोंकी दृष्टि माताके द्वारा परित्यक्त, अनाथ, निरुपाय, क्षुधा-पिपासासे व्याकुल इस दृष्टिहीन बालकपर पड़ी। संतोंका हृदय पिघल गया। वे बालकके पास आए। बालक तो उनको देख ही नहीं रहा था। पूछा—“वत्स! कहाँसे आ रहे हो?” बालकने उत्तर दिया—“श्रीसीतारामजीके चरणोंसे।” पूछा—“कहाँ जाओगे?” बालकने उत्तर दिया—“जहाँ भगवान् और आप श्रीसंतगण भेज देंगे, वहीं चला जाऊँगा।” बालककी प्रत्युत्पन्न बुद्धि देखकर संतचरण भावुक हो उठे। श्रीकील्हदासजीने करुणा करते हुए अपने कमण्डलुका जल बालकके नेत्रस्थानपर छिड़क दिया। उनकी सिद्धिके बलसे बालकके नेत्र आ गए और बालकने प्रथम बार ही नवागत नेत्रोंसे इन युगल संत\-चरणोंके दर्शन किये। धन्य हो गया बालक! नाभाजीको निष्किञ्चन देखकर कील्हदासजी और अग्रदासजी उसे अपने संग गलता ले आए, और कील्हदासजीने अपने छोटे गुरुभ्राता अग्रदासको इस बालकको श्रीरामानन्दीय विरक्त परम्परामें दीक्षित करनेका आदेश दिया। अग्रदासजीने बालकको विरक्त परम्परामें पञ्चसंस्कारविधिसे दीक्षित किया और इनका विरक्तपरम्पराका नाम रखा \textbf{नारायणदास}। नारायणदास सद्गुरुदेव भगवान्‌की आज्ञासे गलतेमें चल रही संतसेवामें रुचि लेने लगे। आनेवाले प्रत्येक संतका वे चरण\-प्रक्षालन करते, उन्हें प्रसाद पवाते और उनका उच्छिष्ट अर्थात् जूठन प्रसाद लेकर स्वयं अपनी क्षुधा बुझाते थे। संत\-सेवासे जब अवसर मिलता तो वे अपने गुरुदेव श्रीअग्रदासजी महाराजकी सेवा भी करते थे।
\end{sloppypar}
\begin{sloppypar}\justifying\hyphenrules{nohyphenation}
एक दिन जब श्रीअग्रदासजी महाराज श्रीसीतारामजीकी मानसी सेवा कर रहे थे, उस समय नारायणदासजी पंखा झल रहे थे। संयोगसे उसी समय अग्रदासजीके किसी भक्तकी नाव समुद्रके भँवरमें अटक गई थी, फँस गई थी। उन्होंने अग्रदासजी महाराजका स्मरण किया। भक्तके मानसिक स्मरणसे अग्रदासजी महाराजकी मानसी सेवामें थोड़ी-सी बाधा पड़ रही थी। नाभाजीने उनकी मनोदशाको भाँप लिया और अपने पंखेको थोड़ा-सा वेगसे चलाया और उसकी वायुसे समुद्रके भँवरमें फँसी हुई भक्तकी नाव आगे चली गई। नाभाजीने विनम्रतासे प्रार्थना की—“गुरुदेव! आप प्रेमसे श्रीसीतारामजीकी मानसी सेवा कीजिये। आपके संकटका मैंने अपने पंखेकी वायुसे समाधान कर दिया है।” अग्रदासजी अपने शिष्यकी इस चामत्कारिक परिस्थितिको देखकर बहुत प्रसन्न हुए और उन्होंने कहा—“बेटे! तुमने मेरी नाभिकी भी परिस्थिति समझ ली, इसलिये आजसे तुम्हारा उपनाम मैं \textbf{नाभा} रख रहा हूँ।”
\end{sloppypar}
\begin{sloppypar}\justifying\hyphenrules{nohyphenation}
नाभा नामके संबन्धमें संतोंके मुखसे एक और कथा सुनी गई है। वह यह कि अग्रदासजी भगवान् श्रीसीतारामजीकी मानसी सेवा कर रहे थे। मानसी सेवामें प्रभुको मुकुट धारण करवा दिया था और माला धारण करानी थी। मानसी भावनामें माला छोटी थी जो मुकुटके ऊपरसे धारण करानेमें कुछ जटिल-सी लग रही थी। अग्रदासजी प्रयास कर रहे थे, परन्तु वह माला भगवान् श्रीसीतारामजीके गलेमें जा नहीं रही थी। उसी समय नारायणदासने कहा—“गुरुदेव! पहले मानसी सेवामें मुकुट उतार लिया जाए, माला धारण कराकर फिर मुकुट धारण करा दिया जाए, सब ठीक हो जाएगा।” तब अग्रदासजीने कहा कि—“तुमने तो मेरी नाभिकी बात जान ली, आजसे तुम्हारा उपनाम \textbf{नाभा} होगा। और नाभा! तुम भगवान् नारायणके नाभिकमलसे उत्पन्न हुए ब्रह्माजीके अवतार हो। तुममें ब्रह्माजीका अंश है। ब्रह्माजी भक्ति\-संप्रदायके प्रथम आचार्य हैं। इसलिये जैसे ब्रह्माजीकी प्रेरणासे वाल्मीकीय\-रामायणम्‌की रचना हुई, उसी प्रकार तुम ब्रह्माजीके अंश हो, अत एव तुम भक्तोंका ही यशोगान करो। श्रीरामकृष्णके यशोगानके लिये तो भगवान्‌ने कलिकालमें तुलसीदास एवं सूरदासको नियुक्त कर दिया है। श्रीरामका यशोगान करनेके लिये नियुक्त हुए हैं गोस्वामी तुलसीदास, जिन्होंने रामचरितमानस द्वारा श्रीरामचरितकी १००~करोड़ रामायणोंका इतिवृत्त गागरमें सागरकी भाँति संक्षिप्त किन्तु विशिष्ट शैलीमें प्रस्तुत किया है। श्रीकृष्णका यशोगान करनेके लिये अद्भुत\-दिव्यदृष्टि\-संपन्न महात्मा सूरदासजीको भगवान्‌ने नियुक्त किया है, जिन्होंने सूरसागरकी रचना कर दी है। अतः अब तुम भक्तोंका ही यशोगान करो, क्योंकि \textbf{भवसागर के तरन को}—भवसागर पार करनेके लिये और कोई दूसरा उपाय नहीं है। \textbf{भवसागर कहँ नाव शुद्ध संतन के चरन} (वि.प.~२०३.२०)। अतः भक्तोंका यश गाओ। चिन्ता मत करना! जैसे तुमने मेरी नाभिकी बात जान ली उसी प्रकार जिन भक्तचरणका तुम वर्णन करोगे वे अपने उपयुक्त चरित्रोंको तुम्हारे मनमें स्वयं प्रतिबिम्बित कर देंगे। निर्भीक हो जाओ, भक्तोंका यश गाओ! कल्याण होगा।”
\end{sloppypar}
\begin{sloppypar}\justifying\hyphenrules{nohyphenation}
उसी आज्ञाका पालन करते हुए नाभाजी कह रहे हैं कि मैं अब भक्तमालकी रचना कर रहा हूँ।
\end{sloppypar}

\addcontentsline{toc}{section}{\texorpdfstring{पद ५: भगवान्‌के चौबीस अवतार}{५: भगवान्‌के चौबीस अवतार}}

{\relscale{1.1875}
{\bfseries
\setlength{\mylenone}{0pt}
\settowidth{\mylentwo}{}
\setlength{\mylenone}{\maxof{\mylenone}{\mylentwo}}
\settowidth{\mylentwo}{चौबीस रूप लीला रुचिर (श्री)अग्रदास उर पद धरौ}
\setlength{\mylenone}{\maxof{\mylenone}{\mylentwo}}
\settowidth{\mylentwo}{जय जय मीन बराह कमठ नरहरि बलि बावन}
\setlength{\mylenone}{\maxof{\mylenone}{\mylentwo}}
\settowidth{\mylentwo}{परशुराम रघुबीर कृष्ण कीरति जगपावन}
\setlength{\mylenone}{\maxof{\mylenone}{\mylentwo}}
\settowidth{\mylentwo}{बुद्ध कलक्की ब्यास पृथू हरि हँस मन्वंतर}
\setlength{\mylenone}{\maxof{\mylenone}{\mylentwo}}
\settowidth{\mylentwo}{जग्य ऋषभ हयग्रीव ध्रुव बरदेन धन्वन्तर}
\setlength{\mylenone}{\maxof{\mylenone}{\mylentwo}}
\settowidth{\mylentwo}{बदरीपति दत कपिलदेव सनकादिक करुणा करौ}
\setlength{\mylenone}{\maxof{\mylenone}{\mylentwo}}
\settowidth{\mylentwo}{चौबीस रूप लीला रुचिर (श्री)अग्रदास उर पद धरौ}
\setlength{\mylenone}{\maxof{\mylenone}{\mylentwo}}
\setlength{\mylentwo}{\baselineskip}
\setlength{\mylenone}{\mylenone + 1pt}
\setlength{\mylen}{(\textwidth - \mylenone)*\real{0.5}}
\begin{longtable}[l]{@{\hspace*{\mylen}}>{\setlength\parfillskip{0pt}}p{\mylenone}@{}@{}l@{}}
 & \\[-\the\mylentwo]
\centering{॥ ५ \hspace*{-1.5mm}॥} & \\ \nopagebreak
चौबीस रूप लीला रुचिर (श्री)अग्रदास उर पद धरौ & ॥\\
जय जय मीन बराह कमठ नरहरि बलि बावन & ।\\ \nopagebreak
परशुराम रघुबीर कृष्ण कीरति जगपावन & ॥\\
बुद्ध कलक्की ब्यास पृथू हरि हँस मन्वंतर & ।\\ \nopagebreak
जग्य ऋषभ हयग्रीव ध्रुव बरदेन धन्वन्तर & ॥\\
बदरीपति दत कपिलदेव सनकादिक करुणा करौ & ।\\ \nopagebreak
चौबीस रूप लीला रुचिर (श्री)अग्रदास उर पद धरौ & ॥
\end{longtable}
}
}
\fancyhead[LO,RE]{{\textmd{\Large ५: चौबीस अवतार}}}
\begin{sloppypar}\justifying\hyphenrules{nohyphenation}
\textbf{मूलार्थ}—चूँकि भगवान् भक्तोंके लिये ही अवतार लेते हैं और भगवान्‌का यह संकेत भी है कि जो भक्त उनका भजन करते हैं, उनकी भगवान् चौबीसों घण्टे रक्षा करते हैं। अतः भक्तोंके आनन्दके लिये भगवान्‌ने यह निर्णय लिया कि मैं भक्तोंके मनमें विश्वास दिलानेके लिये चौबीस घण्टोंके क्रमसे चौबीस अवतार लूँगा। इसीलिये भगवान्‌के मुख्य चौबीस अवतार, जो भागवतजीके द्वितीय स्कन्धके सप्तम अध्यायमें वर्णित हैं, की यहाँ नाभाजी चर्चा कर रहे हैं। \textbf{मीन} अर्थात् मत्स्य, \textbf{कमठ} अर्थात् कच्छप, \textbf{नरहरि} अर्थात् नरसिंह।
\end{sloppypar}
\begin{sloppypar}\justifying\hyphenrules{nohyphenation}
हे मत्स्यावतार भगवान्! आपकी जय हो!! हे वराहावतार भगवान्! आपकी जय हो!! हे कच्छपावतार प्रभु! आपकी जय हो!! हे नरसिंह भगवान्! आपकी जय हो!! हे बलिके लिये वामन रूपमें उपस्थित अवतीर्ण वामन भगवान्! आपकी जय हो!! हे परशुराम भगवान्! आपकी जय हो!! हे \textbf{रघुबीर} अर्थात् रघुकुलमें वीर भगवान् श्रीराम! आपकी जय हो!! हे जगत्‌को पवित्र करनेवाली कीर्तिसे युक्त श्रीकृष्ण भगवान्! आपकी जय हो!! हे कीकट प्रदेशमें अजनको पिता मानकर जन्मे हुए बुद्ध भगवान्! आपकी जय हो!! हे सम्भल ग्राममें जन्म लेनेवाले युगान्तावतार कल्कि भगवान्! आपकी जय हो!! हे वेदव्यास भगवान्! आपकी जय हो!! हे पृथु भगवान्! आपकी जय हो!! हे गजेन्द्रको बचानेवाले हरि अवतार भगवान्! आपकी जय हो! हे सनकादिकोंके प्रश्नोंका उत्तर देनेके लिये हंस रूपमें अवतीर्ण हंसावतार भगवान्! आपकी जय हो!! हे चौदह मन्वन्तराधिपतियोंके रूपमें प्रकट हुए मन्वन्तरावतार भगवान्! आपकी जय हो!! हे यज्ञनारायण भगवान्! आपकी जय हो!! हे ऋषभदेव भगवान्! आपकी जय हो!! हे हयग्रीव भगवान्! आपकी जय हो!! हे ध्रुवको वर देनेवाले सहस्र सिरोंसे युक्त सहस्रशीर्षावतार भगवान्! आपकी जय हो!! हे धन्वन्तरि भगवान्! आपकी जय हो!! हे \textbf{बदरीपति} अर्थात् बदरीनारायण भगवान्! आपकी जय हो!! हे दत्तात्रेय भगवान्! आपकी जय हो!! हे कपिलदेव भगवान्! आपकी जय हो!! हे सनक, सनन्दन, सनातन, सनत्कुमार सनकादि भगवान्! आपकी जय हो!! इस प्रकार सुन्दर लीलाओंको करनेके लिये चौबीस रूप धारण किये हुए प्रभु! आप अग्रदास गुरुदेवजीके सहित मुझ नारायणदासके हृदयमें अपना श्रीचरण पधरा दें।
\end{sloppypar}
\begin{sloppypar}\justifying\hyphenrules{nohyphenation}
भगवान्‌के चौबीस अवतार भक्तोंके आनन्दके लिये ही तो हुए हैं। वे सभी पूर्ण हैं, वे सभी अनादि हैं, वे सभी अनन्त हैं, वे सभी नित्य हैं। उनमें न कभी हानि होती है, न उनमें कभी उपादान होता है। वे शाश्वत हैं, इसलिये कहा जाता है—\textbf{सर्वे देहाः शाश्वताश्च नित्यस्य परमात्मनः}।
\end{sloppypar}
\begin{sloppypar}\justifying\hyphenrules{nohyphenation}
(१)~चाक्षुष मन्वन्तरका जब प्रलय उपस्थित हुआ था, तब राजर्षि सत्यव्रतके समक्ष भगवान्‌ने मत्स्यावतार धारण किया था और उन्हींके सींगमें पृथ्वीको, जो नाव बनकर उपस्थित हुई थी, सत्यव्रतने बाँध दिया था, तथा उसीपर संपूर्ण बीजोंके सहित सत्यव्रत स्वयं आरूढ हुए थे और तब तक भगवान्‌ने उस नावको डूबनेसे बचाया जब तक प्रलयकी लीला चली। उसी समय निद्रावशीभूत ब्रह्माजीके मुखसे चारों वेद स्खलित हो गए थे। उन्हें शङ्खासुरने चुरा लिया था,\footnote{भागवतके अष्टम स्कन्धके अनुसार वेदोंको चुरानेवाले असुरका नाम हयग्रीव था (देखें भा.पु.~८.२४.८, ८.२४.९ और ८.२४.५७)। स्कन्दपुराण (देखें स्क.पु.~२.४.१३.२४, २.४.१३.३०, २.३.१३.३३ और २.४.१३.३८) और गर्गसंहिता (देखें ग.सं.~२.१.२०, २.१.२३, २.१.२५, २.१.२८) आदि ग्रन्थोंके अनुसार असुरका नाम शङ्खासुर था: संपादक।} और मत्स्यावतार भगवान्‌ने शङ्खासुरका वध करके चारों वेद फिर ब्रह्माजीको प्रत्यावर्तित किये थे। अतः भागवतकार भागवतजीके अष्टमस्कन्धके अन्तमें यह कहते हैं—
\end{sloppypar}

{\bfseries
\setlength{\mylenone}{0pt}
\settowidth{\mylentwo}{प्रलयपयसि धातुः सुप्तशक्तेर्मुखेभ्यः श्रुतिगणमपनीतं प्रत्युपादत्त हत्वा}
\setlength{\mylenone}{\maxof{\mylenone}{\mylentwo}}
\settowidth{\mylentwo}{दितिजमकथयद्यो ब्रह्म सत्यव्रतानां तमहमखिलहेतुं जिह्ममीनं नतोऽस्मि}
\setlength{\mylenone}{\maxof{\mylenone}{\mylentwo}}
\setlength{\mylentwo}{\baselineskip}
\setlength{\mylenone}{\mylenone + 1pt}
\setlength{\mylen}{(\textwidth - \mylenone)*\real{0.5}}
\begin{longtable}[l]{@{\hspace*{\mylen}}>{\setlength\parfillskip{0pt}}p{\mylenone}@{}@{}l@{}}
 & \\[-\the\mylentwo]
प्रलयपयसि धातुः सुप्तशक्तेर्मुखेभ्यः श्रुतिगणमपनीतं प्रत्युपादत्त हत्वा & ।\\ \nopagebreak
दितिजमकथयद्यो ब्रह्म सत्यव्रतानां तमहमखिलहेतुं जिह्ममीनं नतोऽस्मि & ॥\\ \nopagebreak
\caption*{(भा.पु.~८.२४.६१)}
\end{longtable}
}

\begin{sloppypar}\justifying\hyphenrules{nohyphenation}
मत्स्यावतार भगवान्‌की जय हो!!
\end{sloppypar}
\begin{sloppypar}\justifying\hyphenrules{nohyphenation}
(२)~श्वेतवाराह कल्पके प्रारम्भमें जब हिरण्याक्षने ब्रह्माजीके द्वारा सद्योरचित पृथ्वीको चुरा लिया था और अपने शरीरके दक्षिण भागसे प्रकट हुए मनुको जब ब्रह्माजीने यज्ञ करने और प्रजाकी उत्पत्ति करनेके लिये आदेश किया था, तब मनुने अपना यह धर्मसंकट बताया—“पृथ्वी तो है ही नहीं, फिर आपकी आज्ञाका पालन मैं कैसे करूँ?” उस समय ब्रह्माजी चिन्तित हुए और उनकी नासिकाके दक्षिण छिद्रसे छोटे-से शूकरके बच्चेके रूपमें भगवान् वराहका प्राकट्य हुआ—
\end{sloppypar}

{\bfseries
\setlength{\mylenone}{0pt}
\settowidth{\mylentwo}{इत्यभिध्यायतो नासाविवरात्सहसानघ}
\setlength{\mylenone}{\maxof{\mylenone}{\mylentwo}}
\settowidth{\mylentwo}{वराहतोको निरगादङ्गुष्ठपरिमाणकः}
\setlength{\mylenone}{\maxof{\mylenone}{\mylentwo}}
\setlength{\mylentwo}{\baselineskip}
\setlength{\mylenone}{\mylenone + 1pt}
\setlength{\mylen}{(\textwidth - \mylenone)*\real{0.5}}
\begin{longtable}[l]{@{\hspace*{\mylen}}>{\setlength\parfillskip{0pt}}p{\mylenone}@{}@{}l@{}}
 & \\[-\the\mylentwo]
इत्यभिध्यायतो नासाविवरात्सहसानघ & ।\\ \nopagebreak
वराहतोको निरगादङ्गुष्ठपरिमाणकः & ॥\\ \nopagebreak
\caption*{(भा.पु.~३.१३.१८)}
\end{longtable}
}

\begin{sloppypar}\justifying\hyphenrules{nohyphenation}
क्षणभरमें सबके देखते-देखते भगवान्‌का शरीर बढ़ा और वे विशालकाय होकर सबको दर्शन देने लगे। अपने घर्घरा शब्दसे, घुरघुराहटसे चिन्तित हुए ब्रह्माजी और उनके मानस\-पुत्रोंके खेदको नष्ट करते हुए भगवान् शूकर समुद्रमें कूद पड़े, और जलगर्भमें जाकर शयन कर रही पृथ्वीको भगवान्‌ने अपने दाँतके अग्रभागमें स्थापित किया। लेकर ऊपर आ रहे थे, वहीं हिरण्याक्षने भगवान्‌का प्रतिरोध किया, और वराह भगवान्‌ने योगबलसे पृथ्वीको स्थापित करके तुमुल युद्ध करके हिरण्याक्षका वध किया। वराह भगवान्‌की जय!!
\end{sloppypar}
\begin{sloppypar}\justifying\hyphenrules{nohyphenation}
(३)~समुद्रके मन्थनके समय जब गरुड द्वारा लाया गया मन्दराचल पर्वत पातालमें धँसने लगा, तब उसे संभालनेके लिये भगवान्‌ने अनन्तयोजनायत कच्छपावतार धारण किया। कच्छप भगवान्‌ने अपनी पीठपर मन्दराचलको स्थापित कर लिया, और तब तक उसे अपनी पीठपर रखा जब तक समुद्र\-मन्थनकी लीला चली। कच्छप भगवान्‌की जय!!
\end{sloppypar}
\begin{sloppypar}\justifying\hyphenrules{nohyphenation}
(४)~हिरण्यकशिपुके अत्याचारसे जब समस्त जीवजात भयभीत हो गया और हिरण्यकशिपुने ब्रह्माजीसे यह वरदान माँग लिया कि \textbf{भूतेभ्यस्त्त्वद्विसृष्टेभ्यो मृत्युर्मा भून्मम प्रभो} (भा.पु.~७.३.३५) अर्थात् आपके द्वारा रचे हुए किसी प्राणीसे मेरी मृत्यु न हो, तब भगवान्‌ने प्रह्लादकी भक्तिसे प्रभावित होकर लोहेके खंभेके मध्यसे उसे फाड़कर नरसिंहावतार स्वीकारा—
\end{sloppypar}

{\bfseries
\setlength{\mylenone}{0pt}
\settowidth{\mylentwo}{सत्यं विधातुं निजभृत्यभाषितं व्याप्तिं च भूतेष्वखिलेषु चात्मनः}
\setlength{\mylenone}{\maxof{\mylenone}{\mylentwo}}
\settowidth{\mylentwo}{अदृश्यतात्यद्भुतरूपमुद्वहन् स्तम्भे सभायां न मृगं न मानुषम्}
\setlength{\mylenone}{\maxof{\mylenone}{\mylentwo}}
\setlength{\mylentwo}{\baselineskip}
\setlength{\mylenone}{\mylenone + 1pt}
\setlength{\mylen}{(\textwidth - \mylenone)*\real{0.5}}
\begin{longtable}[l]{@{\hspace*{\mylen}}>{\setlength\parfillskip{0pt}}p{\mylenone}@{}@{}l@{}}
 & \\[-\the\mylentwo]
सत्यं विधातुं निजभृत्यभाषितं व्याप्तिं च भूतेष्वखिलेषु चात्मनः & ।\\ \nopagebreak
अदृश्यतात्यद्भुतरूपमुद्वहन् स्तम्भे सभायां न मृगं न मानुषम् & ॥\\ \nopagebreak
\caption*{(भा.पु.~७.८.१८)}
\end{longtable}
}

\begin{sloppypar}\justifying\hyphenrules{nohyphenation}
अर्थात् अपने भक्तके वचनको सत्य करनेके लिये, अपनी व्याप्तिको संपूर्ण जीवोंमें प्रमाणित करनेके लिये, स्तम्भके मध्यसे अत्यन्त अद्भुतरूप धारण करते हुए भगवान् प्रकट हो रहे हैं जो पूर्णरूपसे न तो सिंह हैं न मनुष्य, अर्थात् अधःकायसे भगवान् मनुष्य हैं और ऊर्ध्वकायसे सिंह। इन्हीं नरसिंह भगवान्‌ने हिरण्यकशिपुके वक्षःस्थलको विदीर्ण किया। श्रीनरसिंह भगवान्‌की जय!!
\end{sloppypar}
\begin{sloppypar}\justifying\hyphenrules{nohyphenation}
(५)~जब बलिजीने निन्यानवे अश्वमेध यज्ञ कर लिये, उनका सौवाँ अश्वमेध यज्ञ प्रारम्भ हुआ। यदि वह पूर्ण हो जाता तो बलि इन्द्र हो जाते। अदितिने इस व्यवहारसे दुःखी होकर पयोव्रतके माध्यमसे भगवान्‌को संतुष्ट कर लिया। फिर भाद्रपद शुक्ल द्वादशीको अभिजित् मुहूर्त अर्थात् मध्याह्नमें भगवान् शङ्ख, चक्र, गदा और पद्मके साथ अदितिके समक्ष प्रकट हुए, परन्तु अदिति-कश्यपकी प्रार्थनासे उन्होंने छोटे-से वामन बटुका रूप धारण कर लिया। उपवीत संस्कारके अनन्तर अग्निका परिसमूहन करके, दिव्य पादुका धारण करते हुए, दण्ड एवं कमण्डलु लिये हुए, वाजपेय छत्रको स्वीकार करते हुए भगवान् बलिकी यज्ञशाला भृगुकच्छमें पधारे।
\end{sloppypar}

{\bfseries
\setlength{\mylenone}{0pt}
\settowidth{\mylentwo}{श्रुत्वाश्वमेधैर्यजमानमूर्जितं बलिं भृगूणामुपकल्पितैस्ततः}
\setlength{\mylenone}{\maxof{\mylenone}{\mylentwo}}
\settowidth{\mylentwo}{जगाम तत्राखिलसारसम्भृतो भारेण गां सन्नमयन्पदे पदे}
\setlength{\mylenone}{\maxof{\mylenone}{\mylentwo}}
\setlength{\mylentwo}{\baselineskip}
\setlength{\mylenone}{\mylenone + 1pt}
\setlength{\mylen}{(\textwidth - \mylenone)*\real{0.5}}
\begin{longtable}[l]{@{\hspace*{\mylen}}>{\setlength\parfillskip{0pt}}p{\mylenone}@{}@{}l@{}}
 & \\[-\the\mylentwo]
श्रुत्वाश्वमेधैर्यजमानमूर्जितं बलिं भृगूणामुपकल्पितैस्ततः & ।\\ \nopagebreak
जगाम तत्राखिलसारसम्भृतो भारेण गां सन्नमयन्पदे पदे & ॥\\ \nopagebreak
\caption*{(भा.पु.~८.१८.२०)}
\end{longtable}
}

\begin{sloppypar}\justifying\hyphenrules{nohyphenation}
अर्थात् अपने श्रीचरणोंके भारसे पृथ्वीको पग-पगपर झुकाते हुए, संपूर्ण तत्त्वोंसे मण्डित भगवान् वामन बलिको अश्वमेधोंके कारण ऊर्जित सुनकर उनकी यज्ञशालामें पधारे। भगवान् वामनको देखकर बलिने नमन किया और कुछ माँगनेकी प्रार्थना की। भगवान्‌ने बलिसे कहा—“मैं तुमसे केवल तीन पद भूमि माँग रहा हूँ, वह भी मैं अपने चरणोंसे नापूँगा,”—
\end{sloppypar}

{\bfseries
\setlength{\mylenone}{0pt}
\settowidth{\mylentwo}{तस्मात्त्वत्तो महीमीषद्वृणेऽहं वरदर्षभात्}
\setlength{\mylenone}{\maxof{\mylenone}{\mylentwo}}
\settowidth{\mylentwo}{पदानि त्रीणि दैत्येन्द्र सम्मितानि पदा मम}
\setlength{\mylenone}{\maxof{\mylenone}{\mylentwo}}
\setlength{\mylentwo}{\baselineskip}
\setlength{\mylenone}{\mylenone + 1pt}
\setlength{\mylen}{(\textwidth - \mylenone)*\real{0.5}}
\begin{longtable}[l]{@{\hspace*{\mylen}}>{\setlength\parfillskip{0pt}}p{\mylenone}@{}@{}l@{}}
 & \\[-\the\mylentwo]
तस्मात्त्वत्तो महीमीषद्वृणेऽहं वरदर्षभात् & ।\\ \nopagebreak
पदानि त्रीणि दैत्येन्द्र सम्मितानि पदा मम & ॥\\ \nopagebreak
\caption*{(भा.पु.~८.१९.१६)}
\end{longtable}
}

\begin{sloppypar}\justifying\hyphenrules{nohyphenation}
बलिने संकल्प ले लिया और भगवान्‌ने विराट् रूप धारण करके प्रथम पदसे संपूर्ण नीचेके लोकोंको, और द्वितीय पदसे ऊर्ध्वके लोकोंको नाप लिया। उसी द्वितीय पदके अङ्गुष्ठको धोकर ब्रह्माजीने गङ्गाजीको प्रकट कर लिया। तृतीय पदके लिये कुछ भी भूभाग अवशिष्ट न रहा। अनन्तर, दान न देनेके अपराधमें भगवान्‌ने बलिको गरुड द्वारा वारुणपाशमें बँधवाया और उन्हें पाताल भेज दिया। वामन भगवान्‌की जय!!
\end{sloppypar}
\begin{sloppypar}\justifying\hyphenrules{nohyphenation}
(६)~जब हैहयवंशमें प्रसूत सहस्रबाहु आवश्यकतासे अधिक उद्धत हो गया और उसने ब्राह्मणोंके प्रति विद्रोह करनेकी ठानी, तब महर्षि जमदग्निके संकल्पसे रेणुकाके गर्भसे वैशाख शुक्ल तृतीयाको भगवान् परशुरामजीका प्राकट्य हुआ और उन्होंने इक्कीस बार ब्राह्मणद्रोही क्षत्रियोंका संहार किया, संपूर्ण पृथ्वी कश्यपको दे दी, और अन्ततोगत्वा अपनेमें उपस्थित नारायणकी समस्त कलाओंको भगवान् श्रीरामके चरणोंमें सौंप दिया। अवतारका कार्य पूर्ण हुआ। परशुराम भगवान्‌की जय!!
\end{sloppypar}
\begin{sloppypar}\justifying\hyphenrules{nohyphenation}
(७)~जब रावणके अत्याचारसे संपूर्ण पृथ्वी देवताओं, मुनियों, और सिद्धों सहित व्याकुल हो गई, तब देवताओंकी प्रार्थनापर परिपूर्णतम परात्पर परब्रह्म परमात्मा साकेतविहारी श्रीरामजीने रामरूपमें अवतार प्रस्तुत किया। ये अवतार भी हैं और अवतारी भी हैं। और इन्हीं भगवान् श्रीरामके चरित्रको महर्षि वाल्मीकिने सौ करोड़ रामायणोंमें गाया। अन्य महर्षियोंने भी श्रीरामकथा लिखी, रामायण लिखी। मर्यादा\-मानदण्डको स्थापित करके भगवान् श्रीरामने अन्तमें एक ही बात कही—
\end{sloppypar}

{\bfseries
\setlength{\mylenone}{0pt}
\settowidth{\mylentwo}{भूयो भूयो भाविनो भूमिपाला नत्वा नत्वा याचते रामचन्द्रः}
\setlength{\mylenone}{\maxof{\mylenone}{\mylentwo}}
\settowidth{\mylentwo}{सामान्योऽयं धर्मसेतुर्नृपाणां स्वे स्वे काले पालनीयो भवद्भिः}
\setlength{\mylenone}{\maxof{\mylenone}{\mylentwo}}
\setlength{\mylentwo}{\baselineskip}
\setlength{\mylenone}{\mylenone + 1pt}
\setlength{\mylen}{(\textwidth - \mylenone)*\real{0.5}}
\begin{longtable}[l]{@{\hspace*{\mylen}}>{\setlength\parfillskip{0pt}}p{\mylenone}@{}@{}l@{}}
 & \\[-\the\mylentwo]
भूयो भूयो भाविनो भूमिपाला नत्वा नत्वा याचते रामचन्द्रः & ।\\ \nopagebreak
सामान्योऽयं धर्मसेतुर्नृपाणां स्वे स्वे काले पालनीयो भवद्भिः & ॥\\ \nopagebreak
\caption*{(स्क.पु.ब्र.ख.ध.मा.~३४.४०)}
\end{longtable}
}

\begin{sloppypar}\justifying\hyphenrules{nohyphenation}
अर्थात् हे मेरे पश्चात् होनेवाले राजाओं! मैं रामचन्द्र आपको बार-बार प्रणाम करके यह याचना कर रहा हूँ कि \textbf{सामान्योऽयं धर्मसेतुर्नृपाणां स्वे स्वे काले पालनीयो भवद्भिः} अर्थात् मेरे द्वारा मनुष्योंके लिये जो सामान्य धर्मसेतु बनाया गया है, उसका आप लोगोंके द्वारा समय-समयपर रक्षण होना ही चाहिये। ऐसे मर्यादा\-पुरुषोत्तम परिपूर्णतम परब्रह्म परमात्मा परमेश्वर भगवान् श्रीरामकी जय!!
\end{sloppypar}
\begin{sloppypar}\justifying\hyphenrules{nohyphenation}
(८)~कंसके अत्याचारसे पृथ्वी और देवताओंको भयभीत देखकर साधुओंकी रक्षा करनेके लिये, दुष्टोंका नाश करनेके लिये, और धर्मकी स्थापना करनेके लिये भगवान् देवकी-वसुदेवके यहाँ प्रकट हुए। भगवान्‌ने दिव्य बाल\-लीलाएँ कीं, पूतनासे लेकर विदूरथ पर्यन्त दुर्दान्त असुरोंका संहार किया, अर्जुनको कुरुक्षेत्रमें गीता सुनाई, और अनन्तर अपने परिवारको ही राष्ट्रद्रोही व उद्दण्ड देखकर अपने ही शस्त्रोंसे उपसंहृत कर स्वयं प्रभुने अपनी ऐहलौकिक लीलाका संवरण कर लिया। श्रीकृष्ण भगवान्‌की जय!!
\end{sloppypar}
\begin{sloppypar}\justifying\hyphenrules{nohyphenation}
(९)~युगसन्ध्यामें राजाओंके दस्युप्राय हो जानेपर हिंसाकी बहुलताको देखकर कीकट प्रदेशमें अजन नामक क्षत्रियके यहाँ भगवान् बुद्धका अवतार हुआ। वही बुद्ध भगवान् अन्ततोगत्वा उड़ीसामें जगन्नाथके रूपमें प्रसिद्ध हुए। जगन्नाथ बुद्ध भगवान्‌की जय!!
\end{sloppypar}
\begin{sloppypar}\justifying\hyphenrules{nohyphenation}
(१०)~इस कलिकालके अन्तमें सम्भल ग्राममें कल्किके रूपमें भगवान्‌का आविर्भाव होगा, जो शङ्करजीसे शस्त्रविद्या प्राप्त करके, सूर्यनारायणसे दिव्य घोड़ा प्राप्त करके, असुरोंका संहार कर पुनः कृतयुगकी प्रतिष्ठापना करेंगे। कल्कि भगवान्‌की जय!!
\end{sloppypar}
\begin{sloppypar}\justifying\hyphenrules{nohyphenation}
(११)~द्वापरके तृतीय भागमें पराशर महर्षिके मानसिक संकल्पसे सत्यवतीके गर्भसे भगवान् वेदव्यासका आविर्भाव हुआ, जिन्होंने वेदको ऋक्, यजुष्, साम, और अथर्व– इन चार भागोंमें विभक्त किया, अठारह पुराणोंकी रचना की और महाभारत जैसे विशालकाय लक्ष\-श्लोकात्मक ग्रन्थकी रचना की। वेदव्यास भगवान्‌की जय!!
\end{sloppypar}
\begin{sloppypar}\justifying\hyphenrules{nohyphenation}
(१२)~ध्रुवके ही वंशमें अङ्गके पौत्रके रूपमें नास्तिक वेनकी दक्षिण भुजाको मथनेपर भगवान् पृथुका आविर्भाव हुआ। इन्हीं पृथुने अपने सौवें अश्वमेध यज्ञमें इन्द्रको ही हवनकुण्डमें गिरनेके लिये विवश कर दिया, और भगवान्‌के अनुरोध करनेपर कह दिया—“मुझे संतोंके मुखसे कथा सुनते समय दो कानोंमें दस हजार कानोंकी शक्ति दे दी जाए।” अनन्तर सनकादिके उपदेशसे उन्होंने अपनी लौकिक लीलाका संवरण कर लिया। भगवान् पृथुदेवकी जय!!
\end{sloppypar}
\begin{sloppypar}\justifying\hyphenrules{nohyphenation}
(१३)~ग्राहके द्वारा ग्रसे जानेपर गजेन्द्रने जब पुकार लगाई तब हरिमेधा महर्षिके आश्रममें रहनेवाली मृगीको ही माँ बनाकर उसीके गर्भसे प्रभुका हरि अवतार हुआ, और भगवान्‌ने दौड़कर सुदर्शनचक्रसे ग्राहका मुख फाड़कर गजेन्द्रकी रक्षा कर ली। गजेन्द्ररक्षक हरि भगवान्‌की जय!!
\end{sloppypar}
\begin{sloppypar}\justifying\hyphenrules{nohyphenation}
(१४)~सनकादिके द्वारा पूछे हुए प्रश्नोंका उत्तर जब ब्रह्माजी नहीं दे सके तब सनकादिके प्रश्नोंका उत्तर देनेके लिये ही ब्रह्मसभामें भगवान्‌का हंसके रूपमें अवतार हुआ, और सनकादिके प्रश्नोंका उत्तर देकर भगवान्‌ने उन्हें संतुष्ट किया। हंसावतार भगवान्‌की जय!!
\end{sloppypar}
\begin{sloppypar}\justifying\hyphenrules{nohyphenation}
(१५)~चौदह मन्वन्तरोंके अधिपतिके रूपमें भगवान्‌का मन्वन्तरावतार होता है, भगवान् चौदह रूपोंमें देखे जाते हैं और उनके द्वारा वैदिक धर्मकी रक्षा होती है। वर्तमानमें सप्तम मनुके कार्यकालमें हम लोग रह रहे हैं, जिन्हें हम वैवस्वत मनु कहते हैं। मन्वन्तरावतार भगवान्‌की जय!!
\end{sloppypar}
\begin{sloppypar}\justifying\hyphenrules{nohyphenation}
(१६)~स्वायम्भुव मनुकी प्रथम पुत्री आकूति, जिनका विवाह रुचिके साथ हुआ था, उनके गर्भसे यज्ञनारायणका आविर्भाव हुआ। उनको मनुने दत्तक पुत्रके रूपमें स्वीकार कर ले लिया था और उन्होंने अपने सुयम नामक\footnote{भागवतके द्वितीय स्कन्धके सप्तम अध्यायके अनुसार यज्ञनारायण के पुत्रोंको सुयम कहा जाता है, यथा \textbf{जातो रुचेरजनयत्सुयमान् सुयज्ञ आकूतिसूनुरमरानथ दक्षिणायाम्} (भा.पु.~२.७.२)। अन्यत्र यज्ञनारायणके पुत्रोंको तुषित और याम भी कहा गया है (देखें भा.पु.~१.३.१२, भा.पु.~४.१.८, वि.पु.~१.७.२१): संपादक।} पुत्रोंके साथ मनुकी रक्षा की और इन्द्र बनकर यज्ञका विस्तार किया। यज्ञनारायण भगवान्‌की जय!!
\end{sloppypar}
\begin{sloppypar}\justifying\hyphenrules{nohyphenation}
(१७)~प्रियव्रतके प्रपौत्रके रूपमें महाराज नाभिकी धर्मपत्नी मेरुदेवीमें भगवान् ऋषभदेवका आविर्भाव हुआ। इन्द्रने उनसे अपनी जयन्ती नामक कन्याका विवाह किया। सौ पुत्रोंको जन्म देकर भगवान् ऋषभदेवने परमहंस\-पद्धतिका जनताके समक्ष प्रस्ताव किया और अन्ततोगत्वा उसी अवधारणाके फलस्वरूप उन्होंने अपने अवतारको समेट लिया। ऋषभदेव भगवान्‌की जय!!
\end{sloppypar}
\begin{sloppypar}\justifying\hyphenrules{nohyphenation}
(१८)~ब्रह्माजीके यज्ञमें जब मधु-कैटभ दानवोंने वेदको चुरा लिया था तब भगवान् हयग्रीवके रूपमें अवतीर्ण हुए, और उन्होंने मधु-कैटभको मारकर पुनः वेद भगवान्‌को ब्रह्माजीके लिये उपस्थित कर दिया। हयग्रीव भगवान्‌की जय!!
\end{sloppypar}
\begin{sloppypar}\justifying\hyphenrules{nohyphenation}
(१९)~मनुजीके पौत्र ध्रुवजीको वर देनेके लिये भगवान्‌का सहस्रशीर्षावतार हुआ—
\end{sloppypar}

{\bfseries
\setlength{\mylenone}{0pt}
\settowidth{\mylentwo}{त एवमुत्सन्नभया उरुक्रमे कृतावनामाः प्रययुस्त्रिविष्टपम्}
\setlength{\mylenone}{\maxof{\mylenone}{\mylentwo}}
\settowidth{\mylentwo}{सहस्रशीर्षाऽपि ततो गरुत्मता मधोर्वनं भृत्यदिदृक्षया गतः}
\setlength{\mylenone}{\maxof{\mylenone}{\mylentwo}}
\setlength{\mylentwo}{\baselineskip}
\setlength{\mylenone}{\mylenone + 1pt}
\setlength{\mylen}{(\textwidth - \mylenone)*\real{0.5}}
\begin{longtable}[l]{@{\hspace*{\mylen}}>{\setlength\parfillskip{0pt}}p{\mylenone}@{}@{}l@{}}
 & \\[-\the\mylentwo]
त एवमुत्सन्नभया उरुक्रमे कृतावनामाः प्रययुस्त्रिविष्टपम् & ।\\ \nopagebreak
सहस्रशीर्षाऽपि ततो गरुत्मता मधोर्वनं भृत्यदिदृक्षया गतः & ॥\\ \nopagebreak
\caption*{(भा.पु.~४.९.१)}
\end{longtable}
}

\begin{sloppypar}\justifying\hyphenrules{nohyphenation}
उन्हीं सहस्रशीर्षा भगवान्‌ने अपने एक सहस्र मुखोंसे ध्रुवको चूमा, दुलारा और अन्तमें उन्हें सर्वोच्च ध्रुवपद दे दिया। श्रीसहस्रशीर्षा भगवान्‌की जय!!
\end{sloppypar}
\begin{sloppypar}\justifying\hyphenrules{nohyphenation}
(२०)~भगवान्‌ने अमृतमन्थनके समय ही आयुर्वेदके प्रवर्तक धन्वन्तरिके रूपमें अवतार लिया, आयुर्वेदका आविष्कार किया और पुनः भगवान् काशिराजके यहाँ भी पुत्रके रूपमें धन्वन्तरिके रूपमें अवतीर्ण हुए। वह अवतार तिथि है कार्तिक कृष्ण त्रयोदशी, जिसे भाषामें धनतेरस भी कहते हैं। धन्वन्तरि भगवान्‌की जय!!
\end{sloppypar}
\begin{sloppypar}\justifying\hyphenrules{nohyphenation}
(२१)~मनुजीकी तृतीय पुत्री प्रसूतिकी तेरहवीं पुत्री मूर्ति, जिनका विवाह धर्मके साथ हुआ था, उनके गर्भसे भगवान् नर-नारायणके रूपमें प्रकट हुए। अर्थात् नर-नारायणने धर्मको पिता और मूर्तिको माता माना। गन्धमादन पर्वतपर भगवान् उपस्थित हुए। उन्होंने ही सहस्रकवच नामक राक्षसका संहार किया और उर्वशीको अर्पित कर इन्द्रका मद चूर्ण कर दिया। आज भी बदरीक्षेत्रमें विराज कर अपने दर्शनसे वे सतत प्रत्येक व्यक्तिके एक-एक जन्मके पापोंको नष्ट करते रहते हैं। वही हैं इस भारतवर्षके प्रधान देवता। नर-नारायण भगवान्‌की जय!!
\end{sloppypar}
\begin{sloppypar}\justifying\hyphenrules{nohyphenation}
(२२)~मनुजीकी द्वितीय पुत्री देवहूतिजीकी भी द्वितीय पुत्री अनसूयाजीके यहाँ भगवान् दत्तात्रेयके रूपमें प्रकट हुए। भगवान्‌ने अत्रि-अनसूयाको कह दिया कि क्योंकि मैंने अपनेको ही आपको दे दिया है, इसलिये मेरा नाम अब \textbf{दत्त} होगा। इन्हीं दत्तात्रेय भगवान्‌के चरण\-कमलकी धूलिका सेवन करके यदु, हैहय आदियोंने योगसिद्धि प्राप्त की। उनके लोक और परलोक दोनों बन गए। दत्तात्रेय भगवान्‌ने गुरु\-परम्पराका पूर्णरूपसे प्रवर्तन किया। आज भी गिरनार अर्थात् रैवतक पर्वतपर दत्तात्रेय भगवान्‌की पादुकाएँ विराजमान हैं। दत्तात्रेय भगवान्‌की जय!!
\end{sloppypar}
\begin{sloppypar}\justifying\hyphenrules{nohyphenation}
(२३)~मनुजीकी द्वितीय पुत्री देवहूति, जिनका विवाह महर्षि कर्दमके साथ हुआ था, उनके गर्भसे दशम सन्तानके रूपमें कपिलदेव भगवान्‌का प्राकट्य हुआ—
\end{sloppypar}

{\bfseries
\setlength{\mylenone}{0pt}
\settowidth{\mylentwo}{तस्यां बहुतिथे काले भगवान्मधुसूदनः}
\setlength{\mylenone}{\maxof{\mylenone}{\mylentwo}}
\settowidth{\mylentwo}{कार्दमं वीर्यमापन्नो जज्ञेऽग्निरिव दारुणि}
\setlength{\mylenone}{\maxof{\mylenone}{\mylentwo}}
\setlength{\mylentwo}{\baselineskip}
\setlength{\mylenone}{\mylenone + 1pt}
\setlength{\mylen}{(\textwidth - \mylenone)*\real{0.5}}
\begin{longtable}[l]{@{\hspace*{\mylen}}>{\setlength\parfillskip{0pt}}p{\mylenone}@{}@{}l@{}}
 & \\[-\the\mylentwo]
तस्यां बहुतिथे काले भगवान्मधुसूदनः & ।\\ \nopagebreak
कार्दमं वीर्यमापन्नो जज्ञेऽग्निरिव दारुणि & ॥\\ \nopagebreak
\caption*{(भा.पु.~३.२४.६)}
\end{longtable}
}

\begin{sloppypar}\justifying\hyphenrules{nohyphenation}
कपिलदेवने अपनी माँको ब्रह्मविद्याका उपदेश दिया, जिसे \textbf{कपिलाष्टाध्यायी} कहते हैं। माताजीको आध्यात्मिक उपदेश देकर स्वयं प्रभु गङ्गासागरको पधार गए, जिसके लिये आज भी यह सूक्ति प्रचलित है—\textbf{सौ तीरथ बार-बार गङ्गासागर एक बार}। ऐसे कपिलदेव भगवान्‌की जय!!
\end{sloppypar}
\begin{sloppypar}\justifying\hyphenrules{nohyphenation}
(२४)~ब्रह्माजीकी प्रथम मानसी सृष्टिके रूपमें सनक, सनातन, सनन्दन, और सनत्कुमारका प्राकट्य हुआ। ये साक्षात् भगवान् ही हैं, जिनके लिये गोस्वामीजी उत्तरकाण्डमें कहते हैं—
\end{sloppypar}

{\bfseries
\setlength{\mylenone}{0pt}
\setlength{\mylenthree}{0pt}
\settowidth{\mylentwo}{जानि समय सनकादिक आए}
\setlength{\mylenone}{\maxof{\mylenone}{\mylentwo}}
\settowidth{\mylenfour}{तेज पुंज गुन शील सुहाए}
\setlength{\mylenthree}{\maxof{\mylenthree}{\mylenfour}}
\settowidth{\mylentwo}{ब्रह्मानंद सदा लयलीना}
\setlength{\mylenone}{\maxof{\mylenone}{\mylentwo}}
\settowidth{\mylenfour}{देखत बालक बहुकालीना}
\setlength{\mylenthree}{\maxof{\mylenthree}{\mylenfour}}
\settowidth{\mylentwo}{रूप धरे जनु चारिउ बेदा}
\setlength{\mylenone}{\maxof{\mylenone}{\mylentwo}}
\settowidth{\mylenfour}{समदरशी मुनि बिगत बिभेदा}
\setlength{\mylenthree}{\maxof{\mylenthree}{\mylenfour}}
\settowidth{\mylentwo}{आशा बसन ब्यसन यह तिनहीं}
\setlength{\mylenone}{\maxof{\mylenone}{\mylentwo}}
\settowidth{\mylenfour}{रघुपति चरित होइ तहँ सुनहीं}
\setlength{\mylenthree}{\maxof{\mylenthree}{\mylenfour}}
\settowidth{\mylentwo}{तहाँ रहे सनकादि भवानी}
\setlength{\mylenone}{\maxof{\mylenone}{\mylentwo}}
\settowidth{\mylenfour}{जहँ घटसंभव मुनिवर ग्यानी}
\setlength{\mylenthree}{\maxof{\mylenthree}{\mylenfour}}
\setlength{\mylentwo}{\baselineskip}
\setlength{\mylenone}{\mylenone + 1pt}
\setlength{\mylenfour}{\baselineskip}
\setlength{\mylenthree}{\mylenthree + 1pt}
\setlength{\mylen}{(\textwidth - \mylenone)}
\setlength{\mylen}{(\mylen - \mylenthree)*\real{0.5}}
\setlength{\mylen}{(\mylen - 4pt)}
\begin{longtable}[l]{@{\hspace*{\mylen}}>{\setlength\parfillskip{0pt}}p{\mylenone}@{}@{}l@{\hspace{6pt}}>{\setlength\parfillskip{0pt}}p{\mylenthree}@{}@{}l@{}}
 & & & \\[-\the\mylentwo]
जानि समय सनकादिक आए & । & तेज पुंज गुन शील सुहाए & ॥\\
ब्रह्मानंद सदा लयलीना & । & देखत बालक बहुकालीना & ॥\\
रूप धरे जनु चारिउ बेदा & । & समदरशी मुनि बिगत बिभेदा & ॥\\
आशा बसन ब्यसन यह तिनहीं & । & रघुपति चरित होइ तहँ सुनहीं & ॥\\
तहाँ रहे सनकादि भवानी & । & जहँ घटसंभव मुनिवर ग्यानी & ॥\\ \nopagebreak
\caption*{(मा.~७.३२.३-७)}
\end{longtable}
}

\begin{sloppypar}\justifying\hyphenrules{nohyphenation}
श्रीसनकादि भगवान्‌की जय!!
\end{sloppypar}
\begin{sloppypar}\justifying\hyphenrules{nohyphenation}
इस प्रकार दिव्य-दिव्य लीलाएँ करके भगवान् भक्तोंका सतत अनुरञ्जन करते रहते हैं। चौबीस अवतार धारण करनेवाले भगवान्‌की जय!!
\end{sloppypar}

\addcontentsline{toc}{section}{\texorpdfstring{पद ६: श्रीरामके चरणचिह्न}{६: श्रीरामके चरणचिह्न}}

{\relscale{1.1875}
{\bfseries
\setlength{\mylenone}{0pt}
\settowidth{\mylentwo}{}
\setlength{\mylenone}{\maxof{\mylenone}{\mylentwo}}
\settowidth{\mylentwo}{चरन चिन्ह रघुबीर के संतन सदा सहायका}
\setlength{\mylenone}{\maxof{\mylenone}{\mylentwo}}
\settowidth{\mylentwo}{अंकुश अंबर कुलिश कमल जव ध्वजा धेनुपद}
\setlength{\mylenone}{\maxof{\mylenone}{\mylentwo}}
\settowidth{\mylentwo}{शंख चक्र स्वस्तीक जम्बुफल कलस सुधाह्रद}
\setlength{\mylenone}{\maxof{\mylenone}{\mylentwo}}
\settowidth{\mylentwo}{अर्धचंद्र षटकोन मीन बिँदु ऊरधरेषा}
\setlength{\mylenone}{\maxof{\mylenone}{\mylentwo}}
\settowidth{\mylentwo}{अष्टकोन त्रयकोन इंद्र धनु पुरुष बिशेषा}
\setlength{\mylenone}{\maxof{\mylenone}{\mylentwo}}
\settowidth{\mylentwo}{सीतापतिपद नित बसत एते मंगलदायका}
\setlength{\mylenone}{\maxof{\mylenone}{\mylentwo}}
\settowidth{\mylentwo}{चरन चिन्ह रघुबीर के संतन सदा सहायका}
\setlength{\mylenone}{\maxof{\mylenone}{\mylentwo}}
\setlength{\mylentwo}{\baselineskip}
\setlength{\mylenone}{\mylenone + 1pt}
\setlength{\mylen}{(\textwidth - \mylenone)*\real{0.5}}
\begin{longtable}[l]{@{\hspace*{\mylen}}>{\setlength\parfillskip{0pt}}p{\mylenone}@{}@{}l@{}}
 & \\[-\the\mylentwo]
\centering{॥ ६ \hspace*{-1.5mm}॥} & \\ \nopagebreak
चरन चिन्ह रघुबीर के संतन सदा सहायका & ॥\\
अंकुश अंबर कुलिश कमल जव ध्वजा धेनुपद & ।\\ \nopagebreak
शंख चक्र स्वस्तीक जम्बुफल कलस सुधाह्रद & ॥\\
अर्धचंद्र षटकोन मीन बिँदु ऊरधरेषा & ।\\ \nopagebreak
अष्टकोन त्रयकोन इंद्र धनु पुरुष बिशेषा & ॥\\
सीतापतिपद नित बसत एते मंगलदायका & ।\\ \nopagebreak
चरन चिन्ह रघुबीर के संतन सदा सहायका & ॥
\end{longtable}
}
}
\fancyhead[LO,RE]{{\textmd{\Large ६: श्रीरामके चरणचिह्न}}}
\begin{sloppypar}\justifying\hyphenrules{nohyphenation}
\textbf{मूलार्थ}—चूँकि नाभाजी महाराज श्रीसंप्रदायानुगत श्रीरामानन्दी श्रीवैष्णव हैं और श्रीरामोपासक हैं, इसलिये भक्तमाल\-लेखनके पूर्व यह उनके लिये आवश्यक हो जाता है कि वे भगवान्‌के चरणचिह्नोंका ध्यान करें। और जैसा कि हम पूर्वमें कह चुके हैं कि मुख्य रूपसे \textbf{भगवत्}पदवाच्य प्रभु श्रीरामजी ही हैं, इसलिये नाभाजी महाराजने भक्तमाल ग्रन्थकी रचनाके प्रारम्भमें भगवान् रामके श्रीचरणचिह्नोंका चिन्तन किया है। वे कहते हैं कि रघुकुलमें वीर अर्थात् त्यागवीरता, दयावीरता, विद्यावीरता, पराक्रमवीरता और धर्मवीरतासे युक्त भगवान् श्रीरामके श्रीचरणकमलोंके चिह्न संतोंके सदैव सहायक रहते हैं, संतोंकी निरन्तर सहायता करते रहते हैं। ये हैं—\textbf{अङ्कुश} अर्थात् बर्छी, \textbf{अम्बर}—‘अम्बर’ शब्दका तात्पर्य \textbf{आकाश} और \textbf{वस्त्र} इन दोनोंसे है, \textbf{कुलिश} अर्थात् वज्र, \textbf{कमल}, \textbf{जव} अर्थात् यव (धान्य विशेष), \textbf{ध्वजा}, \textbf{धेनुपद} अर्थात् गोपद, \textbf{शङ्ख}, \textbf{चक्र}, \textbf{स्वस्तिक} चिह्न, \textbf{जम्बूफल} (जामुनका फल), \textbf{कलश} एवं \textbf{अमृतका सरोवर}, \textbf{अर्धचन्द्र}, \textbf{षट्कोण}, \textbf{मछली}का चिह्न, \textbf{बिन्दु}, \textbf{ऊर्ध्वरेखा}, \textbf{अष्टकोण}, \textbf{त्रिकोण}, \textbf{इन्द्र} अर्थात् इन्द्रदेवताका चिह्न, \textbf{धनुष}का चिह्न एवं \textbf{विशेष पुरुष} अर्थात् नित्य जीवात्माका चिह्न। अर्थात् (१)~अङ्कुश (२)~आकाश (३)~वस्त्र (४)~वज्र (५)~कमल (६)~यव (७)~ध्वजा (८)~गोपद (९)~शङ्ख (१०)~चक्र (११)~स्वस्तिक (१२)~जम्बूफल (१३)~कलश (१४)~अमृतसरोवर (१५)~अर्धचन्द्र (१६)~षट्कोण (१७)~मछली (१८)~बिन्दु (१९)~ऊर्ध्वरेखा (२०)~अष्टकोण (२१)~त्रिकोण (२२)~इन्द्र (२३)~धनुष एवं (२४)~विशेष जीवात्मा—ये चौबीसों चिह्नके रूपमें सीतापति भगवान् श्रीरामके चरणोंमें निरन्तर विराजते रहते हैं। ये निरन्तर स्मरण\-मात्रसे मङ्गलदायक बन जाते हैं और ये ही भगवान् श्रीरामके चरणकमलोंके चौबीसों चिह्न संतोंके लिये निरन्तर सहायक सिद्ध होते हैं।
\end{sloppypar}
\begin{sloppypar}\justifying\hyphenrules{nohyphenation}
यहाँ यह ध्यान रखना होगा कि प्रियादासजीसे प्रारम्भ करके आज तकके जितने टीकाकार हुए हैं, प्रायः सबके मतमें भगवान्‌के २२ चरणचिह्न ही कहे जाते हैं। किन्तु भगवान् श्रीरामकी कृपासे मेरे मनमें इस प्रकारकी स्फुरणा हुई कि अम्बर शब्दमें श्लेषका यहाँ प्रयोग हुआ है, और \textbf{इन्द्र} तथा \textbf{धनु}—ये दो अलग-अलग शब्द हैं। अब दोनोंको मिलाकर ये २४ चरणचिह्न बन जाते हैं, अर्थात् १२ चरणचिह्न दक्षिण चरणमें और १२ चरणचिह्न वाम चरणमें—ऐसी भी योजना की जा सकती है, अथवा दोनों चरणोंमें ये ही २४ चरणचिह्न समझने होंगे। यह तो साधकके ध्यानमें जैसे स्फुरित होंगे, वैसे उसे योजना करनी होगी। वस्तुतस्तु मेरे ध्यानमें जो स्फुरित हो रहे हैं भगवान्‌के चरणचिह्न, वे इसी प्रकार हैं कि १२ दक्षिण चरणचिह्न हैं और १२ वाम चरणचिह्न हैं।
\end{sloppypar}
\begin{sloppypar}\justifying\hyphenrules{nohyphenation}
प्रत्येक चरणचिह्नका कोई न कोई अभिप्राय है—
\end{sloppypar}
\begin{sloppypar}\justifying\hyphenrules{nohyphenation}
(१)~भगवान्‌के चरणमें \textbf{अङ्कुश}का चिह्न इसलिये है कि इसके ध्यानसे मन रूप मतवाला हाथी वशमें हो जाता है।
\end{sloppypar}
\begin{sloppypar}\justifying\hyphenrules{nohyphenation}
(२)~\textbf{अम्बर} शब्द श्लिष्ट है। प्रथम अम्बरका अर्थ है आकाश, द्वितीय अम्बरका अर्थ है वस्त्र। \textbf{आकाश}के चरणचिह्नका अभिप्राय यह है कि भगवान् आकाशकी भाँति सबको अपने चरणोंमें अवकाश देते हैं, सबको स्वीकार करते हैं। इसलिये गोस्वामी तुलसीदासजीने कहा भी—\textbf{नभ शतकोटि अमित अवकासा} (मा.~७.९१.८)।
\end{sloppypar}
\begin{sloppypar}\justifying\hyphenrules{nohyphenation}
(३)~पुनः \textbf{अम्बर} शब्दका अभिप्राय वस्त्रसे है। इसका तात्पर्य है कि भगवान् अपने भक्तको कभी भी साधनहीन नहीं रखते, सबको वस्त्रादि प्रदान करके धन्य करते रहते हैं। भगवान्‌के यहाँ कोई दिगम्बर नहीं रहता, सबको अन्न-वस्त्र मिलते ही हैं।
\end{sloppypar}
\begin{sloppypar}\justifying\hyphenrules{nohyphenation}
(४)~\textbf{कुलिश}का तात्पर्य यह है कि जैसे वज्र पर्वतको नष्ट करता है, उसी प्रकार भगवान्‌के इस वज्रचिह्नका ध्यान करनेसे पापपर्वत नष्ट हो जाता है।
\end{sloppypar}
\begin{sloppypar}\justifying\hyphenrules{nohyphenation}
(५)~\textbf{कमल}का तात्पर्य है कि जैसे कमल सबमें सुगन्धिका संचार करता है, उसी प्रकार भगवान्‌का यह चरणचिह्न स्मरणमात्रसे भक्तकी दुर्वासना रूप दुर्गन्धको दूर करके उपासनाकी सुगन्धि उसमें ला देता है।
\end{sloppypar}
\begin{sloppypar}\justifying\hyphenrules{nohyphenation}
(६)~\textbf{यव} संपूर्ण धान्योंका उपलक्षण है। भगवान्‌का भक्त धन-धान्यसे पूर्ण ही रहता है।
\end{sloppypar}
\begin{sloppypar}\justifying\hyphenrules{nohyphenation}
(७)~\textbf{ध्वजा} या पताकाका दण्ड। जैसे ध्वजा पताकाको ऊपर किये रहती है, उसी प्रकार भगवद्भक्तका जीवन निरन्तर ऊर्ध्वगामी होता रहता है, सबसे ऊपर ही रहता है, वह कभी किसीके नीचे नहीं रहता।
\end{sloppypar}
\begin{sloppypar}\justifying\hyphenrules{nohyphenation}
(८)~\textbf{धेनुपद} अर्थात् गोपदका तात्पर्य है कि भगवान्‌के चरणकमलका स्मरण करके व्यक्ति संसार-सागरको गोपदकी भाँति अर्थात् गौके खुरकी भाँति सरलतासे पार कर लेता है, और उसपर गोमाताकी कृपा बनी रहती है।
\end{sloppypar}
\begin{sloppypar}\justifying\hyphenrules{nohyphenation}
(९)~\textbf{शङ्ख} मङ्गलका सूचक है।
\end{sloppypar}
\begin{sloppypar}\justifying\hyphenrules{nohyphenation}
(१०)~\textbf{चक्र} स्मरण\-मात्रसे भक्तको कालके भयसे छुड़ाता रहता है।
\end{sloppypar}
\begin{sloppypar}\justifying\hyphenrules{nohyphenation}
(११)~\textbf{स्वस्तिक} मङ्गलका सूचक है। शङ्ख और स्वस्तिकमें अन्तर यह है कि शङ्ख चरणचिह्नके स्मरणसे विजयपूर्वक मङ्गल होता है और स्वस्तिक चरणचिह्नका स्मरण करनेसे और सभी मङ्गल होते रहते हैं।
\end{sloppypar}
\begin{sloppypar}\justifying\hyphenrules{nohyphenation}
(१२)~\textbf{जम्बूफल}का तात्पर्य है कि इसके स्मरणसे व्यक्तिको सभी फल उपलब्ध होते रहते हैं। और जम्बूफल भगवान्‌के समान श्याम रङ्गका है, इसके स्मरणसे श्यामशरीर, जम्बूफलश्याम भगवान् रामका निरन्तर ध्यान होता रहता है।
\end{sloppypar}
\begin{sloppypar}\justifying\hyphenrules{nohyphenation}
(१३)~\textbf{कलश} भी माङ्गलिक चिह्न है। व्यक्तिका हृदय-कलश भक्तिके जलसे पूर्ण रहता है।
\end{sloppypar}
\begin{sloppypar}\justifying\hyphenrules{nohyphenation}
(१४)~\textbf{सुधाह्रद} अर्थात् अमृतका सरोवर। भगवान् श्रीराम स्मरण\-मात्रसे अपने भक्तको आनन्दामृतका वितरण करते रहते हैं और उसे आनन्दसरोवरमें स्नान कराते रहते हैं।
\end{sloppypar}
\begin{sloppypar}\justifying\hyphenrules{nohyphenation}
(१५)~\textbf{अर्धचन्द्र}—इसका तात्पर्य है कि भगवान् अपने स्मरण करनेवाले भक्तको अपने अर्धचन्द्राकार बाणसे बचाते रहते हैं, उसके शत्रुओंका नाश करते रहते हैं।
\end{sloppypar}
\begin{sloppypar}\justifying\hyphenrules{nohyphenation}
(१६)~\textbf{षट्कोण}का तात्पर्य है कि भगवान् स्मरण\-मात्रसे भक्तको काम, क्रोध, लोभ, मोह, मद, मात्सर्य—इन छः विकारोंसे दूर करते रहते हैं।
\end{sloppypar}
\begin{sloppypar}\justifying\hyphenrules{nohyphenation}
(१७)~\textbf{मीन} अर्थात् मछलीका तात्पर्य है कि भगवान्‌का भक्त इस चरण\-चिह्नके स्मरणसे मछलीकी ही भाँति भगवत्प्रेमी बना रहता है, अर्थात् जैसे मछली जलके बिना नहीं रह पाती, उसी प्रकार भक्त भगवान्‌के बिना नहीं रह पाता। जैसा कि गोस्वामीजी विनयपत्रिकाके २६९वें पदमें कहते हैं—\textbf{राम कबहुँ प्रिय लागिहौ जैसे नीर मीनको} (वि.प.~२६९.१)। यही अवधारणा श्रीरामचरितमानसके बालकाण्डके १५१वें दोहेकी ७वीं पङ्क्तिमें स्वायम्भुव मनुजी महाराज भगवान्‌के सम्मुख प्रस्तुत करते हैं कि हे प्रभु श्रीराघव—
\end{sloppypar}

{\bfseries
\setlength{\mylenone}{0pt}
\setlength{\mylenthree}{0pt}
\settowidth{\mylentwo}{मनि बिनु फनि जिमि जल बिनु मीना}
\setlength{\mylenone}{\maxof{\mylenone}{\mylentwo}}
\settowidth{\mylenfour}{मम जीवन तिमि तुमहि अधीना}
\setlength{\mylenthree}{\maxof{\mylenthree}{\mylenfour}}
\setlength{\mylentwo}{\baselineskip}
\setlength{\mylenone}{\mylenone + 1pt}
\setlength{\mylenfour}{\baselineskip}
\setlength{\mylenthree}{\mylenthree + 1pt}
\setlength{\mylen}{(\textwidth - \mylenone)}
\setlength{\mylen}{(\mylen - \mylenthree)*\real{0.5}}
\setlength{\mylen}{(\mylen - 4pt)}
\begin{longtable}[l]{@{\hspace*{\mylen}}>{\setlength\parfillskip{0pt}}p{\mylenone}@{}@{}l@{\hspace{6pt}}>{\setlength\parfillskip{0pt}}p{\mylenthree}@{}@{}l@{}}
 & & & \\[-\the\mylentwo]
मनि बिनु फनि जिमि जल बिनु मीना & । & मम जीवन तिमि तुमहि अधीना & ॥\\ \nopagebreak
\caption*{(मा.~१.१५१.७)}
\end{longtable}
}

\begin{sloppypar}\justifying\hyphenrules{nohyphenation}
(१८)~\textbf{बिन्दु}—बिन्दुका तात्पर्य है कि व्यक्तिके जीवनमें भगवदनुरागका बिन्दु उपस्थित रहता है, और वह शून्यतासे सर्वथा दूर रहता है। बिन्दु सबको दशगुणित करता है। अर्थात् जैसे एकके साथ शून्य जब जुड़ता है तो वह एकको दश गुना बना देता है। उसी प्रकार भगवन्नाम एक अङ्क है और सभी साधन शून्यके समान हैं—
\end{sloppypar}

{\bfseries
\setlength{\mylenone}{0pt}
\settowidth{\mylentwo}{राम नाम इक अंक है सब साधन हैं सून}
\setlength{\mylenone}{\maxof{\mylenone}{\mylentwo}}
\settowidth{\mylentwo}{अंक गये कछु हाथ नहीं अंक रहे दस गून}
\setlength{\mylenone}{\maxof{\mylenone}{\mylentwo}}
\setlength{\mylentwo}{\baselineskip}
\setlength{\mylenone}{\mylenone + 1pt}
\setlength{\mylen}{(\textwidth - \mylenone)*\real{0.5}}
\begin{longtable}[l]{@{\hspace*{\mylen}}>{\setlength\parfillskip{0pt}}p{\mylenone}@{}@{}l@{}}
 & \\[-\the\mylentwo]
राम नाम इक अंक है सब साधन हैं सून & ।\\ \nopagebreak
अंक गये कछु हाथ नहीं अंक रहे दस गून & ॥\\ \nopagebreak
\caption*{(दो.~१०)}
\end{longtable}
}

\begin{sloppypar}\justifying\hyphenrules{nohyphenation}
वह बिन्दु भगवन्नामसे जुड़कर उसके फलको दश गुना बना दिया करता है।
\end{sloppypar}
\begin{sloppypar}\justifying\hyphenrules{nohyphenation}
(१९)~\textbf{ऊर्ध्वरेखा}—ऊर्ध्वरेखाका तात्पर्य है कि यह रेखा भगवान्‌के भक्तको स्मरण\-मात्रसे सतत ऊपर उठाती रहती है।
\end{sloppypar}
\begin{sloppypar}\justifying\hyphenrules{nohyphenation}
(२०)~\textbf{अष्टकोण}का तात्पर्य है कि स्मरण\-मात्रसे यह चिह्न भगवद्भक्तको आठों प्रकृतियोंकी विडम्बनाओंसे दूर करता रहता है।
\end{sloppypar}
\begin{sloppypar}\justifying\hyphenrules{nohyphenation}
(२१)~\textbf{त्रिकोण}—यह चिह्न स्मरण\-मात्रसे भगवद्भक्तको काम, क्रोध, लोभसे दूर किये रहता है, अथवा त्रिगुणोंसे अतीत कर देता है।
\end{sloppypar}
\begin{sloppypar}\justifying\hyphenrules{nohyphenation}
(२२)~\textbf{इन्द्र}—ये देवराज हैं। इस चरणचिह्नका तात्पर्य है कि स्मरण\-मात्रसे भगवान् अपने भक्तको इन्द्र जैसा पद भी दे देते हैं, जैसे महाराज बलिको दे दिया।
\end{sloppypar}
\begin{sloppypar}\justifying\hyphenrules{nohyphenation}
(२३)~\textbf{धनु}—यह भगवान्‌का आयुध विशेष है। इसका तात्पर्य है कि यह प्रणव है, जो स्मरण\-मात्रसे व्यक्तिको वैदिक मर्यादाओंसे जोड़े रहता है। \textbf{प्रणवं धनुः शरो ह्यात्मा} (मु.उ.~२.२.४)—प्रणव ही धनुष है, बाण ही आत्मा है, और \textbf{ब्रह्म तल्लक्ष्यमुच्यते} (मु.उ.~२.२.४)—ब्रह्म उसका लक्ष्य है। \textbf{अप्रमत्तेन वेद्धव्यं शरवत्तन्मयो भवेत्} (मु.उ.~२.२.४)—अप्रमत्त होकर लक्ष्यकी सिद्धि कर लेनी चाहिये, अर्थात् प्रणवसे सतत जीवात्माका संपर्क बना रहना चाहिये। और दूसरी बात—धनुषका यह भी तात्पर्य है कि धनुष टेढ़ा होता है। इसका संकेत यह है कि भगवान्‌के यहाँ सीधे और टेढ़े—दोनोंको ही स्थान मिलता है। किसीको भगवान् ठुकराते नहीं, चाहे वह सीधा हो या टेढ़ा हो। और तीसरा तात्पर्य है कि जो धनुषकी भाँति झुक जाता है, उसीको भगवान् अपना आश्रय देते हैं।
\end{sloppypar}
\begin{sloppypar}\justifying\hyphenrules{nohyphenation}
(२४)~\textbf{पुरुष बिशेषा}—\textbf{पुरुष} पद यहाँ जीवात्माका वाचक है, और जीवात्मा तीन प्रकारका होता है—बद्ध, मुक्त और नित्य। यहाँ \textbf{पुरुष विशेष}का तात्पर्य यह है कि भगवान्‌के स्मरण\-मात्रसे जीव नित्य बन जाता है अर्थात् बद्ध और मुक्त दोनों परिस्थतियोंसे मुक्त होकर भगवान्‌की नित्य सेवामें लग जाता है। यहाँ \textbf{नित्य} जीवात्माका अर्थ है भगवान्‌का नित्य परिकर जैसे श्रीहनुमान् आदि।
\end{sloppypar}
\begin{sloppypar}\justifying\hyphenrules{nohyphenation}
इस प्रकार ये चौबीसों चरणचिह्न सीतापति भगवान् श्रीरामके चरणोंमें निरन्तर निवास करते हैं। ये स्वयं स्मरण\-मात्रसे मङ्गलप्रदान करते हैं और ये ही भगवान् श्रीरामके चौबीसों चरण\-चिह्न संतोंके लिये निरन्तर सहायक बने रहते हैं।
\end{sloppypar}
\begin{sloppypar}\justifying\hyphenrules{nohyphenation}
अब चरणचिह्नोंका स्मरण संपन्न हुआ, और नाभाजीको भक्तमाल जैसे ग्रन्थकी रचना करनी है, तो उन्हें आचार्योंका स्मरण पहले कर लेना चाहिये। भागवतधर्मके बारह आचार्य हैं, जिनका स्मरण यमराजजी करते हैं भागवतजीके षष्ठ स्कन्धके तृतीय अध्यायमें—
\end{sloppypar}

{\bfseries
\setlength{\mylenone}{0pt}
\settowidth{\mylentwo}{स्वयम्भूर्नारदः शम्भुः कुमारः कपिलो मनुः}
\setlength{\mylenone}{\maxof{\mylenone}{\mylentwo}}
\settowidth{\mylentwo}{प्रह्लादो जनको भीष्मो बलिर्वैयासकिर्वयम्}
\setlength{\mylenone}{\maxof{\mylenone}{\mylentwo}}
\setlength{\mylentwo}{\baselineskip}
\setlength{\mylenone}{\mylenone + 1pt}
\setlength{\mylen}{(\textwidth - \mylenone)*\real{0.5}}
\begin{longtable}[l]{@{\hspace*{\mylen}}>{\setlength\parfillskip{0pt}}p{\mylenone}@{}@{}l@{}}
 & \\[-\the\mylentwo]
स्वयम्भूर्नारदः शम्भुः कुमारः कपिलो मनुः & ।\\ \nopagebreak
प्रह्लादो जनको भीष्मो बलिर्वैयासकिर्वयम् & ॥\\ \nopagebreak
\caption*{(भा.पु.~६.३.२०)}
\end{longtable}
}

\begin{sloppypar}\justifying\hyphenrules{nohyphenation}
अर्थात् हे बटुओं! (१)~\textbf{श्रीब्रह्माजी} (२)~\textbf{श्रीनारदजी} (३)~\textbf{श्रीशङ्करजी} (४)~सनक, सनन्दन, सनातन, और सनत्कुमार—ये चार \textbf{सनकादि} (५)~\textbf{श्रीकपिलजी} (६)~\textbf{स्वायम्भुव मनुजी} (७)~\textbf{श्रीप्रह्लादजी} (८)~\textbf{श्रीजनकजी} (९)~\textbf{श्रीभीष्मजी} (१०)~\textbf{श्रीबलिजी} (११)~\textbf{श्रीशुकाचार्यजी} और (१२)~\textbf{वयम्} अर्थात् मैं धर्मस्वरूप यमराज—यही बारह भागवत\-धर्मोंको जानते हैं। श्रीनाभाजी उन्हींका यहाँ स्मरण कर रहे हैं—
\end{sloppypar}

\addcontentsline{toc}{section}{\texorpdfstring{पद ७: प्रधान द्वादश भक्त}{७: प्रधान द्वादश भक्त}}

{\relscale{1.1875}
{\bfseries
\setlength{\mylenone}{0pt}
\settowidth{\mylentwo}{}
\setlength{\mylenone}{\maxof{\mylenone}{\mylentwo}}
\settowidth{\mylentwo}{इनकी कृपा और पुनि समुझे द्वादस भक्त प्रधान}
\setlength{\mylenone}{\maxof{\mylenone}{\mylentwo}}
\settowidth{\mylentwo}{बिधि नारद शंकर सनकादिक कपिलदेव मनु भूप}
\setlength{\mylenone}{\maxof{\mylenone}{\mylentwo}}
\settowidth{\mylentwo}{नरहरिदास जनक भीषम बलि शुकमुनि धर्मस्वरूप}
\setlength{\mylenone}{\maxof{\mylenone}{\mylentwo}}
\settowidth{\mylentwo}{अन्तरंग अनुचर हरिजू के जो इनको जस गावै}
\setlength{\mylenone}{\maxof{\mylenone}{\mylentwo}}
\settowidth{\mylentwo}{आदि अंतलौं मंगल तिनको श्रोता बक्ता पावै}
\setlength{\mylenone}{\maxof{\mylenone}{\mylentwo}}
\settowidth{\mylentwo}{अजामेल परसंग यह निर्णय परम धर्म के जान}
\setlength{\mylenone}{\maxof{\mylenone}{\mylentwo}}
\settowidth{\mylentwo}{इनकी कृपा और पुनि समुझे द्वादस भक्त प्रधान}
\setlength{\mylenone}{\maxof{\mylenone}{\mylentwo}}
\setlength{\mylentwo}{\baselineskip}
\setlength{\mylenone}{\mylenone + 1pt}
\setlength{\mylen}{(\textwidth - \mylenone)*\real{0.5}}
\begin{longtable}[l]{@{\hspace*{\mylen}}>{\setlength\parfillskip{0pt}}p{\mylenone}@{}@{}l@{}}
 & \\[-\the\mylentwo]
\centering{॥ ७ \hspace*{-1.5mm}॥} & \\ \nopagebreak
इनकी कृपा और पुनि समुझे द्वादस भक्त प्रधान & ॥\\
बिधि नारद शंकर सनकादिक कपिलदेव मनु भूप & ।\\ \nopagebreak
नरहरिदास जनक भीषम बलि शुकमुनि धर्मस्वरूप & ॥\\
अन्तरंग अनुचर हरिजू के जो इनको जस गावै & ।\\ \nopagebreak
आदि अंतलौं मंगल तिनको श्रोता बक्ता पावै & ॥\\
अजामेल परसंग यह निर्णय परम धर्म के जान & ।\\ \nopagebreak
इनकी कृपा और पुनि समुझे द्वादस भक्त प्रधान & ॥
\end{longtable}
}
}
\fancyhead[LO,RE]{{\textmd{\Large ७: द्वादश भक्त}}}
\begin{sloppypar}\justifying\hyphenrules{nohyphenation}
\textbf{मूलार्थ}—(१)~\textbf{श्रीब्रह्मा} (२)~\textbf{श्रीनारद} (३)~\textbf{श्रीशङ्कर} (४)~\textbf{श्रीसनकादि} (सनक, सनातन, सनन्दन, सनत्कुमार) (५)~\textbf{श्रीकपिलदेव} (६)~महाराज \textbf{स्वायम्भुव मनु} (७)~\textbf{नरहरिदास} अर्थात् नरसिंह भगवान्‌के दास \textbf{श्रीप्रह्लाद} (८)~\textbf{श्रीजनक} (जो भगवती सीताजीके पिताजी हैं) (९)~\textbf{पितामह भीष्म} (१०)~\textbf{श्रीबलि} (११)~\textbf{श्रीशुकाचार्य} और (१२)~धर्मस्वरूप \textbf{श्रीयमराजजी}—ये बारह श्रीहरिके अन्तरङ्ग अनुचर हैं। जो इनका यश गा रहे हैं या गाएँगे, उनके लिये आदिसे अन्त पर्यन्त मङ्गल ही मङ्गल होगा। और इनका यशोगान करके श्रोता और वक्ता आदिसे अन्त पर्यन्त मङ्गल प्राप्त करते रहेंगे। अजामिलका यह प्रसंग \textbf{परम धर्म} अर्थात् भक्तिके निर्णयका ही प्रसंग समझना चाहिये। इन्हींकी कृपासे और लोग भी भक्तिका रहस्य समझ सकते हैं, क्योंकि ये ही प्रधान द्वादश भक्त हैं।
\end{sloppypar}
\begin{sloppypar}\justifying\hyphenrules{nohyphenation}
यहाँ \textbf{मनु}से स्वायम्भुव मनु समझना चाहिये और \textbf{जनक} पदसे भगवती सीताजीके पिता सीरध्वज जनकको समझना चाहिये, जिनके संबन्धमें मानसकार कहते हैं—
\end{sloppypar}

{\bfseries
\setlength{\mylenone}{0pt}
\setlength{\mylenthree}{0pt}
\settowidth{\mylentwo}{प्रनवउँ परिजन सहित बिदेहू}
\setlength{\mylenone}{\maxof{\mylenone}{\mylentwo}}
\settowidth{\mylenfour}{जाहि राम पद गूढ़ सनेहु}
\setlength{\mylenthree}{\maxof{\mylenthree}{\mylenfour}}
\settowidth{\mylentwo}{जोग भोग महँ राखेउ गोई}
\setlength{\mylenone}{\maxof{\mylenone}{\mylentwo}}
\settowidth{\mylenfour}{राम बिलोकत प्रगटेउ सोई}
\setlength{\mylenthree}{\maxof{\mylenthree}{\mylenfour}}
\setlength{\mylentwo}{\baselineskip}
\setlength{\mylenone}{\mylenone + 1pt}
\setlength{\mylenfour}{\baselineskip}
\setlength{\mylenthree}{\mylenthree + 1pt}
\setlength{\mylen}{(\textwidth - \mylenone)}
\setlength{\mylen}{(\mylen - \mylenthree)*\real{0.5}}
\setlength{\mylen}{(\mylen - 4pt)}
\begin{longtable}[l]{@{\hspace*{\mylen}}>{\setlength\parfillskip{0pt}}p{\mylenone}@{}@{}l@{\hspace{6pt}}>{\setlength\parfillskip{0pt}}p{\mylenthree}@{}@{}l@{}}
 & & & \\[-\the\mylentwo]
प्रनवउँ परिजन सहित बिदेहू & । & जाहि राम पद गूढ़ सनेहु & ॥\\
जोग भोग महँ राखेउ गोई & । & राम बिलोकत प्रगटेउ सोई & ॥\\ \nopagebreak
\caption*{(मा.~१.१७.१-२)}
\end{longtable}
}

\begin{sloppypar}\justifying\hyphenrules{nohyphenation}
पितामह \textbf{भीष्म} भी भागवतधर्मके आचार्य हैं। ये नवम आचार्य हैं। कदाचित् इसीलिये भागवतकारने पितामह भीष्मका वर्णन भागवतजीके प्रथम स्कन्धके नवम अध्यायमें ही किया है। भीष्मके संबन्धमें एक जिज्ञासा स्वाभाविक है कि जब वे इतने बड़े भागवत\-धर्माचार्य हैं, तो उन्होंने द्रौपदीकी चीरहरण\-परिस्थितिको मूकदर्शक होकर क्यों देखा? दुर्योधनका विरोध क्यों नहीं किया? इसका मुझे जो उत्तर सूझ रहा है, वह यह है कि भीष्म अपनी मूकदर्शकतासे भगवान्‌की महिमाका आख्यान करना चाहते हैं, भगवान्‌की महिमाको लोगोंके समक्ष प्रकट करना चाहते हैं। यदि वे दुर्योधनका विरोध कर देते, तब द्रौपदीजीका चीरहरण तो रुक जाता, परन्तु भगवान्‌की चीरवर्धन\-लीलाको जनताके समक्ष कैसे प्रकट किया जाता? इसलिये मूकदर्शक रहकर पितामह भीष्मने दिखाया कि भगवान् अपने भक्तकी कैसे रक्षा करते हैं। यहाँ भगवान्‌ने द्रौपदीकी लज्जाको रखनेके लिये ग्यारहवाँ वस्त्रावतार स्वीकार कर लिया, जैसा कि गोस्वामी तुलसीदासजी दोहावलीके १६७वें दोहेमें कहते हैं—
\end{sloppypar}

{\bfseries
\setlength{\mylenone}{0pt}
\settowidth{\mylentwo}{सभा सभासद निरखि पट पकरि उठायो हाथ}
\setlength{\mylenone}{\maxof{\mylenone}{\mylentwo}}
\settowidth{\mylentwo}{तुलसी कियो इगारहों बसन बेस जदुनाथ}
\setlength{\mylenone}{\maxof{\mylenone}{\mylentwo}}
\setlength{\mylentwo}{\baselineskip}
\setlength{\mylenone}{\mylenone + 1pt}
\setlength{\mylen}{(\textwidth - \mylenone)*\real{0.5}}
\begin{longtable}[l]{@{\hspace*{\mylen}}>{\setlength\parfillskip{0pt}}p{\mylenone}@{}@{}l@{}}
 & \\[-\the\mylentwo]
सभा सभासद निरखि पट पकरि उठायो हाथ & ।\\ \nopagebreak
तुलसी कियो इगारहों बसन बेस जदुनाथ & ॥\\ \nopagebreak
\caption*{(दो.~१६७)}
\end{longtable}
}

\begin{sloppypar}\justifying\hyphenrules{nohyphenation}
\textbf{बलि}—जो आत्मनिवेदन जैसी भक्तिके एकमात्र उदाहरण हैं।
\end{sloppypar}
\begin{sloppypar}\justifying\hyphenrules{nohyphenation}
\textbf{शुकाचार्य}का तो कहना ही क्या! उन्होंने तो जन्मसे ही संसारकी असारताका अनुभव कर लिया था, इसलिये संसारमें किसीसे संबन्ध ही नहीं रखा। उनके स्मरणका यह कितना रोचक श्लोक है, जो भागवतके प्रथम स्कन्धके द्वितीय अध्यायके द्वितीय श्लोकके रूपमें प्रस्तुत हो रहा है—
\end{sloppypar}

{\bfseries
\setlength{\mylenone}{0pt}
\settowidth{\mylentwo}{यं प्रव्रजन्तमनुपेतमपेतकृत्यं द्वैपायनो विरहकातर आजुहाव}
\setlength{\mylenone}{\maxof{\mylenone}{\mylentwo}}
\settowidth{\mylentwo}{पुत्रेति तन्मयतया तरवोऽभिनेदुस्तं सर्वभूतहृदयं मुनिमानतोऽस्मि}
\setlength{\mylenone}{\maxof{\mylenone}{\mylentwo}}
\setlength{\mylentwo}{\baselineskip}
\setlength{\mylenone}{\mylenone + 1pt}
\setlength{\mylen}{(\textwidth - \mylenone)*\real{0.5}}
\begin{longtable}[l]{@{\hspace*{\mylen}}>{\setlength\parfillskip{0pt}}p{\mylenone}@{}@{}l@{}}
 & \\[-\the\mylentwo]
यं प्रव्रजन्तमनुपेतमपेतकृत्यं द्वैपायनो विरहकातर आजुहाव & ।\\ \nopagebreak
पुत्रेति तन्मयतया तरवोऽभिनेदुस्तं सर्वभूतहृदयं मुनिमानतोऽस्मि & ॥\\ \nopagebreak
\caption*{(भा.पु.~१.२.२)}
\end{longtable}
}

\begin{sloppypar}\justifying\hyphenrules{nohyphenation}
अर्थात् जो जन्म लेनेके पश्चात् किसीके पास भी नहीं गए, जो अपने संपूर्ण कृत्योंको समाप्त किये हुए थे, और कभी न आनेके लिये जाते हुए जिनको देखकर विरहसे व्याकुल होकर द्वैपायन वेदव्यासजीने \textbf{पुत्र!} इस प्रकार चिल्लाया, और उन्हींकी भावनासे भावित होकर वृक्ष भी जिनको जाते हुए देखकर \textbf{पुत्र! पुत्र!} कहकर चिल्लाने लगे और फिर भी जो अपने निश्चयसे नहीं डिगे और नहीं लौटे—उन्हीं संपूर्ण प्राणिमात्रके हृदयमें विराजमान और संपूर्ण प्राणिमात्रको भगवान्‌के चरणमें आकृष्ट करनेवाले भगवान् शुकाचार्यके चरणकमलमें मैं आदरपूर्वक नमन कर रहा हूँ। ऐसे शुकाचार्य, जिन्हें यहाँ ग्यारहवें आचार्यके रूपमें कहा जा रहा है।
\end{sloppypar}
\begin{sloppypar}\justifying\hyphenrules{nohyphenation}
\textbf{धर्मस्वरूप यमराज}—जो पापियोंको दण्ड देनेके लिये यम बन जाते हैं और धार्मिकोंके लिये धर्मके रूपमें रहते ही हैं।
\end{sloppypar}
\begin{sloppypar}\justifying\hyphenrules{nohyphenation}
ये बारहों परम\-भागवत\-धर्मके वेत्ता आचार्य हैं। ये बारहों भगवान् श्रीहरिजूके अन्तरङ्ग अनुचर हैं। इनके स्मरण, मनन, कीर्तन और गानसे श्रोता और वक्ता आदिसे अन्त पर्यन्त मङ्गल ही प्राप्त करेंगे। चूँकि इनकी अजामिल\-प्रसंगमें वेदव्यासजीने चर्चा की है, इसलिये इस प्रसंगको परमधर्मका निर्णय\-प्रसंग समझना चाहिये और इन्हींकी कृपासे और लोग भी भक्तिका सिद्धान्त समझ सकते हैं।
\end{sloppypar}
\begin{sloppypar}\justifying\hyphenrules{nohyphenation}
अब भक्तमालका प्रारम्भ हो रहा है। जैसा कि हम प्रथम ही कह चुके हैं कि श्रीनाभाजी इस ग्रन्थमें \textbf{भगवत्}पदसे अर्थात् भगवान्‌से तीन विशेष भगवान्‌को अभिप्रेत करते हैं—श्रीराम, श्रीकृष्ण और श्रीनारायण। इन्हीं तीनों भगवानोंकी छत्रच्छायामें इस भक्तमालका पल्लवन होता है, यद्यपि अवतारी तो भगवान् श्रीराम ही हैं। अत एव जब यहाँ भगवान् श्रीराम, श्रीकृष्ण और श्रीनारायणके भक्तोंकी चर्चा करनी है, तो पहले श्रीनारायणके सोलह पार्षदोंकी चर्चा अनिवार्य हो जाती है। इसलिये भक्तमालकार नाभाजी कहते हैं—
\end{sloppypar}

\addcontentsline{toc}{section}{\texorpdfstring{पद ८: भगवान् नारायणके सोलह पार्षद}{८: भगवान् नारायणके सोलह पार्षद}}

{\relscale{1.1875}
{\bfseries
\setlength{\mylenone}{0pt}
\settowidth{\mylentwo}{}
\setlength{\mylenone}{\maxof{\mylenone}{\mylentwo}}
\settowidth{\mylentwo}{मो चित्तबृत्ति नित तहँ रहो जहँ नारायण पारषद}
\setlength{\mylenone}{\maxof{\mylenone}{\mylentwo}}
\settowidth{\mylentwo}{बिष्वक्सेन जय बिजय प्रबल बल मंगलकारी}
\setlength{\mylenone}{\maxof{\mylenone}{\mylentwo}}
\settowidth{\mylentwo}{नंद सुनंद सुभद्र भद्र जग आमयहारी}
\setlength{\mylenone}{\maxof{\mylenone}{\mylentwo}}
\settowidth{\mylentwo}{चंड प्रचंड बिनीत कुमुद कुमुदाच्छ करुणालय}
\setlength{\mylenone}{\maxof{\mylenone}{\mylentwo}}
\settowidth{\mylentwo}{शील सुशील सुषेण भाव भक्तन प्रतिपालय}
\setlength{\mylenone}{\maxof{\mylenone}{\mylentwo}}
\settowidth{\mylentwo}{लक्ष्मीपति प्रीणन प्रबीन भजनानँद भक्तन सुहृद}
\setlength{\mylenone}{\maxof{\mylenone}{\mylentwo}}
\settowidth{\mylentwo}{मो चित्तबृत्ति नित तहँ रहो जहँ नारायण पारषद}
\setlength{\mylenone}{\maxof{\mylenone}{\mylentwo}}
\setlength{\mylentwo}{\baselineskip}
\setlength{\mylenone}{\mylenone + 1pt}
\setlength{\mylen}{(\textwidth - \mylenone)*\real{0.5}}
\begin{longtable}[l]{@{\hspace*{\mylen}}>{\setlength\parfillskip{0pt}}p{\mylenone}@{}@{}l@{}}
 & \\[-\the\mylentwo]
\centering{॥ ८ \hspace*{-1.5mm}॥} & \\ \nopagebreak
मो चित्तबृत्ति नित तहँ रहो जहँ नारायण पारषद & ॥\\
बिष्वक्सेन जय बिजय प्रबल बल मंगलकारी & ।\\ \nopagebreak
नंद सुनंद सुभद्र भद्र जग आमयहारी & ॥\\
चंड प्रचंड बिनीत कुमुद कुमुदाच्छ करुणालय & ।\\ \nopagebreak
शील सुशील सुषेण भाव भक्तन प्रतिपालय & ॥\\
लक्ष्मीपति प्रीणन प्रबीन भजनानँद भक्तन सुहृद & ।\\ \nopagebreak
मो चित्तबृत्ति नित तहँ रहो जहँ नारायण पारषद & ॥
\end{longtable}
}
}
\fancyhead[LO,RE]{{\textmd{\Large ८: सोलह पार्षद}}}
\begin{sloppypar}\justifying\hyphenrules{nohyphenation}
\textbf{मूलार्थ}—अर्थात् मेरी चित्तवृत्ति वहींपर निरन्तर निवास करे, जहाँ भगवान् श्रीमन्नारायण विष्णुजीके सोलह पार्षद विराजते रहते हैं। उनमेंसे श्रीविष्वक्सेन, श्रीजय, श्रीविजय, श्रीप्रबल और श्रीबल—ये मङ्गलकारी पार्षद हैं। श्रीनन्द, श्रीसुनन्द, श्रीसुभद्र और श्रीभद्र—ये जगत्‌के आमय अर्थात् रोगोंको हरनेवाले हैं। श्रीचण्ड, श्रीप्रचण्ड, श्रीकुमुद और श्रीकुमुदाक्ष—ये विनम्र हैं और करुणाके घर हैं। श्रीशील, श्रीसुशील और श्रीसुषेण—ये भावसे भगवद्भजन करनेवाले भक्तोंका प्रतिपालन करते रहते हैं। इस प्रकार (१)~\textbf{विष्वक्सेन} (२)~\textbf{जय} (३)~\textbf{विजय} (४)~\textbf{प्रबल} (५)~\textbf{बल} (६)~\textbf{नन्द} (७)~\textbf{सुनन्द} (८)~\textbf{सुभद्र} (९)~\textbf{भद्र} (१०)~\textbf{चण्ड} (११)~\textbf{प्रचण्ड} (१२)~\textbf{कुमुद} (१३)~\textbf{कुमुदाक्ष} (१४)~\textbf{शील} (१५)~\textbf{सुशील} (१६)~\textbf{सुषेण}—ये सोलहों पार्षद लक्ष्मीजीके पति भगवान् नारायणके \textbf{प्रीणन} अर्थात् उनको प्रसन्न करनेमें कुशल हैं, निरन्तर भगवान्‌को प्रसन्न करते रहते हैं, और भजनमें आनन्द लेनेवाले भक्तोंके सुहृद् हैं। ऐसे सोलहों नारायण\-पार्षद जहाँ विराज रहे हों, वहाँ मेरी चित्तवृत्ति निरन्तर निवास करती रहे।
\end{sloppypar}
\begin{sloppypar}\justifying\hyphenrules{nohyphenation}
विष्वक्सेन प्रथम आचार्य हैं और भगवान्‌के प्रथम पार्षद हैं। जय और विजय—ये भगवान्‌के प्यारे द्वारपाल हैं, जिनके लिये कहा जाता है—
\end{sloppypar}

{\bfseries
\setlength{\mylenone}{0pt}
\setlength{\mylenthree}{0pt}
\settowidth{\mylentwo}{द्वारपाल हरि के प्रिय दोऊ}
\setlength{\mylenone}{\maxof{\mylenone}{\mylentwo}}
\settowidth{\mylenfour}{जय अरु बिजय जान सब कोऊ}
\setlength{\mylenthree}{\maxof{\mylenthree}{\mylenfour}}
\setlength{\mylentwo}{\baselineskip}
\setlength{\mylenone}{\mylenone + 1pt}
\setlength{\mylenfour}{\baselineskip}
\setlength{\mylenthree}{\mylenthree + 1pt}
\setlength{\mylen}{(\textwidth - \mylenone)}
\setlength{\mylen}{(\mylen - \mylenthree)*\real{0.5}}
\setlength{\mylen}{(\mylen - 4pt)}
\begin{longtable}[l]{@{\hspace*{\mylen}}>{\setlength\parfillskip{0pt}}p{\mylenone}@{}@{}l@{\hspace{6pt}}>{\setlength\parfillskip{0pt}}p{\mylenthree}@{}@{}l@{}}
 & & & \\[-\the\mylentwo]
द्वारपाल हरि के प्रिय दोऊ & । & जय अरु बिजय जान सब कोऊ & ॥\\ \nopagebreak
\caption*{(मा.~१.१२२.४)}
\end{longtable}
}

\begin{sloppypar}\justifying\hyphenrules{nohyphenation}
यहाँ एक बात विशेष ध्यान देनेकी है, वह यह कि ये पार्षद कभी भी भगवान्‌से दूर नहीं होते। जय-विजय भी एक रूपमें सनकादिका शाप स्वीकार करके प्रथम जन्ममें हिरण्यकशिपु-हिरण्याक्ष, द्वितीय जन्ममें रावण-कुम्भकर्ण और तृतीय जन्ममें शिशुपाल-दन्तवक्रके रूपमें उपस्थित रहे। पर दूसरे रूपमें वे निरन्तर भगवान्‌की सेवामें ही रहे, वे कभी सेवासे दूर नहीं होते। इसलिये विष्वक्सेन, जय, विजय, प्रबल और बल—ये सदैव मङ्गल ही करते रहते हैं। नन्द, सुनन्द, सुभद्र और भद्र—ये चारों जगत्‌के काम, क्रोध, लोभ, मोहसे उत्पन्न \textbf{आमय} अर्थात् रोगोंको दूर करते रहते हैं। चण्ड और प्रचण्ड नामसे भयंकर प्रतीत होते हैं, पर स्वभावसे बहुत विनीत हैं। कुमुद और कुमुदाक्ष—ये करुणाके आगार हैं। शील, सुशील और सुषेण भावुक भक्तोंका निरन्तर प्रतिपालन करते रहते हैं।
\end{sloppypar}
\begin{sloppypar}\justifying\hyphenrules{nohyphenation}
अब नाभाजी हरिवल्लभोंसे प्रार्थना कर रहे हैं—
\end{sloppypar}

\addcontentsline{toc}{section}{\texorpdfstring{पद ९: हरिवल्लभ}{९: हरिवल्लभ}}

{\relscale{1.1875}
{\bfseries
\setlength{\mylenone}{0pt}
\settowidth{\mylentwo}{}
\setlength{\mylenone}{\maxof{\mylenone}{\mylentwo}}
\settowidth{\mylentwo}{हरिबल्लभ सब प्रारथों (जिन) चरनरेनु आशा धरी}
\setlength{\mylenone}{\maxof{\mylenone}{\mylentwo}}
\settowidth{\mylentwo}{कमला गरुड सुनंद आदि षोडस प्रभुपदरति}
\setlength{\mylenone}{\maxof{\mylenone}{\mylentwo}}
\settowidth{\mylentwo}{(हनुमंत) जामवंत सुग्रीव विभीषण शबरी खगपति}
\setlength{\mylenone}{\maxof{\mylenone}{\mylentwo}}
\settowidth{\mylentwo}{ध्रुव उद्धव अँबरीष बिदुर अक्रूर सुदामा}
\setlength{\mylenone}{\maxof{\mylenone}{\mylentwo}}
\settowidth{\mylentwo}{चंद्रहास चित्रकेतु ग्राह गज पांडव नामा}
\setlength{\mylenone}{\maxof{\mylenone}{\mylentwo}}
\settowidth{\mylentwo}{कौषारव कुंतीबधू पट ऐंचत लज्जाहरी}
\setlength{\mylenone}{\maxof{\mylenone}{\mylentwo}}
\settowidth{\mylentwo}{हरिबल्लभ सब प्रारथों (जिन) चरनरेनु आशा धरी}
\setlength{\mylenone}{\maxof{\mylenone}{\mylentwo}}
\setlength{\mylentwo}{\baselineskip}
\setlength{\mylenone}{\mylenone + 1pt}
\setlength{\mylen}{(\textwidth - \mylenone)*\real{0.5}}
\begin{longtable}[l]{@{\hspace*{\mylen}}>{\setlength\parfillskip{0pt}}p{\mylenone}@{}@{}l@{}}
 & \\[-\the\mylentwo]
\centering{॥ ९ \hspace*{-1.5mm}॥} & \\ \nopagebreak
हरिबल्लभ सब प्रारथों (जिन) चरनरेनु आशा धरी & ॥\\
कमला गरुड सुनंद आदि षोडस प्रभुपदरति & ।\\ \nopagebreak
(हनुमंत) जामवंत सुग्रीव विभीषण शबरी खगपति & ॥\\
ध्रुव उद्धव अँबरीष बिदुर अक्रूर सुदामा & ।\\ \nopagebreak
चंद्रहास चित्रकेतु ग्राह गज पांडव नामा & ॥\\
कौषारव कुंतीबधू पट ऐंचत लज्जाहरी & ।\\ \nopagebreak
हरिबल्लभ सब प्रारथों (जिन) चरनरेनु आशा धरी & ॥
\end{longtable}
}
}
\fancyhead[LO,RE]{{\textmd{\Large ९: हरिवल्लभ}}}
\begin{sloppypar}\justifying\hyphenrules{nohyphenation}
\textbf{मूलार्थ}—मैं उन भागवतोंकी प्रार्थना कर रहा हूँ, जो श्रीहरिको प्रिय हैं, और श्रीहरि जिनको प्रिय हैं, जिन्होंने भगवान् श्रीहरिकी चरणरेणुको प्राप्त करनेके लिये आशा धारण की है, और मैंने अर्थात् नारायणदास नाभाने भी जिन भक्तोंके श्रीचरणोंकी धूलिको प्राप्त करनेके लिये अपने जीवनमें आशा धारण की है, आशा लगाए बैठा हूँ कि कभी-न-कभी मुझे इनके चरणोंकी धूलि प्राप्त हो ही जाएगी। ये हैं—(१)~\textbf{कमला} अर्थात् \textbf{श्रीलक्ष्मीजी} (२)~\textbf{गरुडजी} (३)~\textbf{सुनन्द} आदि भगवान् नारायणके सोलह पार्षद (४)~\textbf{श्रीहनुमान्‌जी} (५)~\textbf{श्रीजाम्बवान्‌जी} (६)~\textbf{श्रीसुग्रीवजी} (७)~\textbf{श्रीविभीषणजी} (८)~\textbf{माँ शबरीजी} (९)~खगपति पक्षिराज \textbf{श्रीजटायुजी} (१०)~\textbf{श्रीध्रुवजी} (११)~\textbf{श्रीउद्धवजी} (१२)~\textbf{श्रीअम्बरीषजी} (१३)~\textbf{श्रीविदुरजी} (१४)~\textbf{श्रीअक्रूरजी} (१५)~\textbf{श्रीसुदामाजी} (१६)~\textbf{श्रीचन्द्रहासजी} (१७)~\textbf{श्रीचित्रकेतुजी} (१८)~\textbf{ग्राह} और (१९)~\textbf{गजेन्द्र} तथा (२०)~पाण्डवनामसे प्रसिद्ध पाँचों \textbf{पाण्डुपुत्र} (श्रीयुधिष्ठिर, श्रीभीम, श्रीअर्जुन, श्रीनकुल, और श्रीसहदेवजी) (२१)~\textbf{कौषारव} अर्थात् कुषारव मुनिके पुत्र \textbf{श्रीमैत्रेयजी} (२२)~\textbf{माँ कुन्तीजी} एवं (२३)~कुन्तीजीकी आज्ञासे अपनी जीवनचर्या चलानेवालीं \textbf{द्रौपदीजी} जिनके वस्त्रको दुःशासन द्वारा खींचते समय प्रभु श्रीकृष्ण भगवान्‌ने जिनकी \textbf{लज्जा आहरी} अर्थात् लौटा दी थी, जाती हुई लज्जा लौटा दी थी—ऐसे श्रीहरिवल्लभोंसे मैं प्रार्थना करता हूँ कि आप दया करें, मुझपर कृपा करें, भगवत्प्रेमामृत प्रदान करें।
\end{sloppypar}
\begin{sloppypar}\justifying\hyphenrules{nohyphenation}
\textbf{लक्ष्मीजी}के संबन्धमें तो हम सभी जानते हैं कि वे भगवान्‌के चरणकी सेवा ही करती रहती हैं, और कुछ भी नहीं करना चाहती हैं। उनके लिये भागवतका यह श्लोक बहुत प्रसिद्ध है—
\end{sloppypar}

{\bfseries
\setlength{\mylenone}{0pt}
\settowidth{\mylentwo}{ब्रह्मादयो बहुतिथं यदपाङ्गमोक्षकामास्तपः समचरन् भगवत्प्रपन्नाः}
\setlength{\mylenone}{\maxof{\mylenone}{\mylentwo}}
\settowidth{\mylentwo}{सा श्रीः स्ववासमरविन्दवनं विहाय यत्पादसौभगमलं भजतेऽनुरक्ता}
\setlength{\mylenone}{\maxof{\mylenone}{\mylentwo}}
\setlength{\mylentwo}{\baselineskip}
\setlength{\mylenone}{\mylenone + 1pt}
\setlength{\mylen}{(\textwidth - \mylenone)*\real{0.5}}
\begin{longtable}[l]{@{\hspace*{\mylen}}>{\setlength\parfillskip{0pt}}p{\mylenone}@{}@{}l@{}}
 & \\[-\the\mylentwo]
ब्रह्मादयो बहुतिथं यदपाङ्गमोक्षकामास्तपः समचरन् भगवत्प्रपन्नाः & ।\\ \nopagebreak
सा श्रीः स्ववासमरविन्दवनं विहाय यत्पादसौभगमलं भजतेऽनुरक्ता & ॥\\ \nopagebreak
\caption*{(भा.पु.~१.१६.३२)}
\end{longtable}
}

\begin{sloppypar}\justifying\hyphenrules{nohyphenation}
अर्थात् ब्रह्मा आदि देवता जिन भगवती लक्ष्मीके कृपाकटाक्ष\-मोक्षकी कामना करते हुए भगवत्प्रपन्न होनेपर भी बहुत काल तक तपस्या किये, फिर भी लक्ष्मीजीने उन्हें एक भी बार टेढ़ी दृष्टिसे भी नहीं देखा, अर्थात् नेत्रके कोनेसे भी नहीं देखा, वही लक्ष्मीजी अपने निवास रूप कमलवनको छोड़कर अनुरक्त भावसे जिन प्रभुके चरणकमलके सौन्दर्यका ही भजन करती रहती हैं—अन्य देवपत्नियाँ अपने भिन्न-भिन्न कार्योंमें लगती हैं, जैसे पार्वतीजी भी गृहस्थधर्मका पालन करती हैं, दोनों बेटोंकी सम्भाल और शिवजीकी भिन्न-भिन्न प्रवृत्तियोंमें पार्वतीजी लगती हैं, वे कभी क्रोध भी करती हैं, युद्ध भी करती हैं—परन्तु लक्ष्मीजीको हमने-आपने कभी युद्ध करते नहीं देखा होगा, क्योंकि उनको भगवान्‌के चरणकी सेवासे समय ही नहीं मिलता, बस उनके चरणका लालन ही करती रहती हैं लक्ष्मीजी।
\end{sloppypar}
\begin{sloppypar}\justifying\hyphenrules{nohyphenation}
\textbf{गरुड} भगवान् नारायणके वाहन हैं और ये ही अन्ततोगत्वा भगवान् रामको मेघनाद द्वारा नागपाशमें बँधे हुए देखकर भ्रमित हो जाते हैं। फिर भुशुण्डिजीके चरणोंमें जाकर वे अपना भ्रम दूर करते हैं, और भुशुण्डिजीके श्रीमुखसे श्रीरामकथाके ८४ प्रसंगोंका श्रवण करते हैं। और गरुड मुक्तकण्ठसे कहते हैं—
\end{sloppypar}

{\bfseries
\setlength{\mylenone}{0pt}
\settowidth{\mylentwo}{गयउ मोर संदेह सुनेउँ सकल रघुपति चरित}
\setlength{\mylenone}{\maxof{\mylenone}{\mylentwo}}
\settowidth{\mylentwo}{भयउ राम पद नेह तव प्रसाद बायस तिलक}
\setlength{\mylenone}{\maxof{\mylenone}{\mylentwo}}
\setlength{\mylentwo}{\baselineskip}
\setlength{\mylenone}{\mylenone + 1pt}
\setlength{\mylen}{(\textwidth - \mylenone)*\real{0.5}}
\begin{longtable}[l]{@{\hspace*{\mylen}}>{\setlength\parfillskip{0pt}}p{\mylenone}@{}@{}l@{}}
 & \\[-\the\mylentwo]
गयउ मोर संदेह सुनेउँ सकल रघुपति चरित & ।\\ \nopagebreak
भयउ राम पद नेह तव प्रसाद बायस तिलक & ॥\\ \nopagebreak
\caption*{(मा.~७.६८क)}
\end{longtable}
}

\begin{sloppypar}\justifying\hyphenrules{nohyphenation}
ये ही गरुडदेव संपूर्ण रामकथा सुननेके पश्चात् उपासनामें थोड़ा अन्तर करते हुए प्रतीत होते हैं। अपनी पीठपर तो वे भगवान् नारायणको विराजमान कराते हैं उनके वाहन बनकर और अपने हृदयमें भगवान् रामको विराजमान कराते हैं—
\end{sloppypar}

{\bfseries
\setlength{\mylenone}{0pt}
\settowidth{\mylentwo}{तासु चरन सिर नाइ करि प्रेम सहित मतिधीर}
\setlength{\mylenone}{\maxof{\mylenone}{\mylentwo}}
\settowidth{\mylentwo}{गयउ गरुड बैकुंठ तब हृदय राखि रघुबीर}
\setlength{\mylenone}{\maxof{\mylenone}{\mylentwo}}
\setlength{\mylentwo}{\baselineskip}
\setlength{\mylenone}{\mylenone + 1pt}
\setlength{\mylen}{(\textwidth - \mylenone)*\real{0.5}}
\begin{longtable}[l]{@{\hspace*{\mylen}}>{\setlength\parfillskip{0pt}}p{\mylenone}@{}@{}l@{}}
 & \\[-\the\mylentwo]
तासु चरन सिर नाइ करि प्रेम सहित मतिधीर & ।\\ \nopagebreak
गयउ गरुड बैकुंठ तब हृदय राखि रघुबीर & ॥\\ \nopagebreak
\caption*{(मा.~७.१२५क)}
\end{longtable}
}

\begin{sloppypar}\justifying\hyphenrules{nohyphenation}
\textbf{सुनन्द} आदि भगवान् नारायणके सोलह पार्षद हैं, जिनकी चर्चा इसके पूर्व छप्पयमें की जा चुकी है।
\end{sloppypar}
\begin{sloppypar}\justifying\hyphenrules{nohyphenation}
\textbf{श्रीहनुमान्‌जी}की चर्चा कौन नहीं जानता? वही एक ऐसे व्यक्तित्व हैं, जो भगवान्‌की बहिरङ्ग और अन्तरङ्ग दोनों सेवाएँ करना जानते हैं। वे भगवान्‌से दूर रहकर भी भगवद्भजन करते हैं और निकट रहकर भी, और उनके लिये वियोग और संयोग दोनों समान होते हैं। इसलिये हनुमान्‌जी जैसी प्रीति और हनुमान्‌जी जैसी सेवा—एक साथ दोनों किसीके वशकी नहीं है। सेवा लक्ष्मण कर सकते हैं, प्रीतिका निर्वहण भरत कर सकते हैं, पर दोनों निर्वहण तो हनुमान्‌जी ही करना जानते हैं, इसलिये कहा गया—
\end{sloppypar}

{\bfseries
\setlength{\mylenone}{0pt}
\setlength{\mylenthree}{0pt}
\settowidth{\mylentwo}{हनूमान सम नहिं बड़भागी}
\setlength{\mylenone}{\maxof{\mylenone}{\mylentwo}}
\settowidth{\mylenfour}{नहिं कोउ राम चरन अनुरागी}
\setlength{\mylenthree}{\maxof{\mylenthree}{\mylenfour}}
\settowidth{\mylentwo}{गिरिजा जासु प्रीति सेवकाई}
\setlength{\mylenone}{\maxof{\mylenone}{\mylentwo}}
\settowidth{\mylenfour}{बार बार प्रभु निज मुख गाई}
\setlength{\mylenthree}{\maxof{\mylenthree}{\mylenfour}}
\setlength{\mylentwo}{\baselineskip}
\setlength{\mylenone}{\mylenone + 1pt}
\setlength{\mylenfour}{\baselineskip}
\setlength{\mylenthree}{\mylenthree + 1pt}
\setlength{\mylen}{(\textwidth - \mylenone)}
\setlength{\mylen}{(\mylen - \mylenthree)*\real{0.5}}
\setlength{\mylen}{(\mylen - 4pt)}
\begin{longtable}[l]{@{\hspace*{\mylen}}>{\setlength\parfillskip{0pt}}p{\mylenone}@{}@{}l@{\hspace{6pt}}>{\setlength\parfillskip{0pt}}p{\mylenthree}@{}@{}l@{}}
 & & & \\[-\the\mylentwo]
हनूमान सम नहिं बड़भागी & । & नहिं कोउ राम चरन अनुरागी & ॥\\
गिरिजा जासु प्रीति सेवकाई & । & बार बार प्रभु निज मुख गाई & ॥\\ \nopagebreak
\caption*{(मा.~७.५०.८-९)}
\end{longtable}
}

\begin{sloppypar}\justifying\hyphenrules{nohyphenation}
हनुमान्‌जीके संबन्धमें यह कहा जाता है कि जब भगवान् श्रीरामके द्वारा सीताजीको यह कहा गया कि आप जिसको चाहें उसे हार दे दें, तो सीताजीने हनुमान्‌जीको अपना हार दे दिया। और यहाँ लोगोंका कहना है कि हनुमान्‌जीने उसकी सारी मणियाँ तोड़-तोड़कर नीचे गिराईं। जब लोगोंने पूछा—“इतने बहुमूल्य हारको आपने क्यों तोड़ डाला?” तब हनुमान्‌जीने कह दिया—“इसमें रामनाम नहीं है।” तब लोगोंने कहा—“तो क्या आपके हृदयमें रामनाम है?” तब उन्होंने अपनी छाती चीरकर दिखा दी। यह उक्ति सुननेमें रोचक लगती है, परन्तु इसका हमें अभी कोई लिखित प्रमाण उपलब्ध नहीं हुआ है। और यह भी मानना ठीक नहीं होगा कि सीताजी द्वारा दिया हुआ वह हार रामनाममय न हो, राममय न हो। यह आख्यान कुछ अतिरञ्जना जैसा लगता है, अतिशयोक्ति जैसा लगता है। इसलिये यहाँ मेरा तो यही निवेदन है कि हनुमान्‌जीकी भक्तिके लिये अनेक उद्धरण वाल्मीकीय\-रामायण, रामचरितमानस, महाभारत और किं बहुना वाल्मीकि द्वारा लिखित सौ करोड़ रामायणोंमें पर्याप्त रूपसे वर्णित हैं, तो हनुमन्त\-लालजीके संबन्धमें तो कुछ भी और कहनेकी आवश्यकता ही नहीं है।
\end{sloppypar}
\begin{sloppypar}\justifying\hyphenrules{nohyphenation}
\textbf{जाम्बवान्}—ये परम भागवत हैं और भगवान्‌के प्रति इनकी अत्यन्त भक्ति है। स्वयं ब्रह्माजी ही तो जाम्बवान्‌के रूपमें आए। शिवजी हनुमान्‌जी बनकर और ब्रह्माजी जाम्बवान् बनकर। गोस्वामीजी कहते हैं—
\end{sloppypar}

{\bfseries
\setlength{\mylenone}{0pt}
\settowidth{\mylentwo}{जानि राम सेवा सरस समुझि करब अनुमान}
\setlength{\mylenone}{\maxof{\mylenone}{\mylentwo}}
\settowidth{\mylentwo}{पुरुषा ते सेवक भए हर ते भे हनुमान}
\setlength{\mylenone}{\maxof{\mylenone}{\mylentwo}}
\setlength{\mylentwo}{\baselineskip}
\setlength{\mylenone}{\mylenone + 1pt}
\setlength{\mylen}{(\textwidth - \mylenone)*\real{0.5}}
\begin{longtable}[l]{@{\hspace*{\mylen}}>{\setlength\parfillskip{0pt}}p{\mylenone}@{}@{}l@{}}
 & \\[-\the\mylentwo]
जानि राम सेवा सरस समुझि करब अनुमान & ।\\ \nopagebreak
पुरुषा ते सेवक भए हर ते भे हनुमान & ॥\\ \nopagebreak
\caption*{(दो.~१४३)}
\end{longtable}
}

\begin{sloppypar}\justifying\hyphenrules{nohyphenation}
वही जाम्बवान् श्रीरामको जब मिलते हैं, समय-समयपर भगवान्‌के कार्यमें पूर्ण सहायता करते हैं। अङ्गदको विचलित हुआ देखकर जाम्बवान् उन्हें भगवत्कथा सुनाकर एक बहुत मङ्गलमय सिद्धान्तको प्रस्तुत करते हैं। जाम्बवान् कहते हैं—“अङ्गद! तुम समझ नहीं रहे हो। हम सभी सेवक अत्यन्त बड़भागी हैं, जो सतत सगुण साकार ब्रह्म श्रीरामजीके चरणोंमें अनुराग रखते हैं। भगवान् अपनी इच्छासे और अपने भक्तोंकी इच्छाका पालन करनेके लिये पृथ्वी, देवताओं, गौओं और ब्राह्मणोंके हितके लिये अवतार लेते हैं, और जब-जब भगवान् अवतार लेते हैं, तब-तब हम सगुणोपासक भक्तजन मोक्षसुखको त्यागकर प्रभुके अवतारकालमें उनकी लीलाके उपकरण बन जाते हैं,”—
\end{sloppypar}

{\bfseries
\setlength{\mylenone}{0pt}
\setlength{\mylenthree}{0pt}
\settowidth{\mylentwo}{तात राम कहँ नर जनि मानहु}
\setlength{\mylenone}{\maxof{\mylenone}{\mylentwo}}
\settowidth{\mylenfour}{निर्गुन ब्रह्म अजित अज जानहु}
\setlength{\mylenthree}{\maxof{\mylenthree}{\mylenfour}}
\settowidth{\mylentwo}{हम सब सेवक अति बड़भागी}
\setlength{\mylenone}{\maxof{\mylenone}{\mylentwo}}
\settowidth{\mylenfour}{संतत सगुन ब्रह्म अनुरागी}
\setlength{\mylenthree}{\maxof{\mylenthree}{\mylenfour}}
\setlength{\mylentwo}{\baselineskip}
\setlength{\mylenone}{\mylenone + 1pt}
\setlength{\mylenfour}{\baselineskip}
\setlength{\mylenthree}{\mylenthree + 1pt}
\setlength{\mylen}{(\textwidth - \mylenone)}
\setlength{\mylen}{(\mylen - \mylenthree)*\real{0.5}}
\setlength{\mylen}{(\mylen - 4pt)}
\begin{longtable}[l]{@{\hspace*{\mylen}}>{\setlength\parfillskip{0pt}}p{\mylenone}@{}@{}l@{\hspace{6pt}}>{\setlength\parfillskip{0pt}}p{\mylenthree}@{}@{}l@{}}
 & & & \\[-\the\mylentwo]
तात राम कहँ नर जनि मानहु & । & निर्गुन ब्रह्म अजित अज जानहु & ॥\\ \nopagebreak
हम सब सेवक अति बड़भागी & । & संतत सगुन ब्रह्म अनुरागी & ॥\\
\end{longtable}
}


{\bfseries
\setlength{\mylenone}{0pt}
\settowidth{\mylentwo}{निज इच्छा प्रभु अवतरइ सुर महि गो द्विज लागि}
\setlength{\mylenone}{\maxof{\mylenone}{\mylentwo}}
\settowidth{\mylentwo}{सगुन उपासक संग तहँ रहहिं मोक्ष सुख त्यागि}
\setlength{\mylenone}{\maxof{\mylenone}{\mylentwo}}
\setlength{\mylentwo}{\baselineskip}
\setlength{\mylenone}{\mylenone + 1pt}
\setlength{\mylen}{(\textwidth - \mylenone)*\real{0.5}}
\begin{longtable}[l]{@{\hspace*{\mylen}}>{\setlength\parfillskip{0pt}}p{\mylenone}@{}@{}l@{}}
 & \\[-\the\mylentwo]
निज इच्छा प्रभु अवतरइ सुर महि गो द्विज लागि & ।\\ \nopagebreak
सगुन उपासक संग तहँ रहहिं मोक्ष सुख त्यागि & ॥\\ \nopagebreak
\caption*{(मा.~४.२६.१२-४.२६)}
\end{longtable}
}

\begin{sloppypar}\justifying\hyphenrules{nohyphenation}
जाम्बवान् द्वापर तक भगवान्‌की सेवामें रहते हैं और अन्ततोगत्वा स्यमन्तक\-मणिके सन्दर्भमें गुफामें प्रविष्ट हुए भगवान् श्रीकृष्णसे तुमुल युद्ध करके उन्हें श्रीराम रूपमें पहचानकर उनसे क्षमा माँगते हैं और त्रेतासे भगवान्‌की प्रतीक्षा कर रहीं और तपस्या कर रहीं अपनी प्रिय पुत्री जाम्बवतीजीको भगवान्‌को सौंप देते हैं।
\end{sloppypar}
\begin{sloppypar}\justifying\hyphenrules{nohyphenation}
\textbf{सुग्रीव}—ये भगवान्‌के अन्तरङ्ग सखा हैं। सूर्यनारायण ही भगवान्‌की सेवा करनेके लिये सुग्रीवके रूपमें प्रस्तुत हुए हैं। सुग्रीव ही तो कहते हैं—
\end{sloppypar}

{\bfseries
\setlength{\mylenone}{0pt}
\setlength{\mylenthree}{0pt}
\settowidth{\mylentwo}{अब प्रभु कृपा करहु एहि भाँती}
\setlength{\mylenone}{\maxof{\mylenone}{\mylentwo}}
\settowidth{\mylenfour}{सब तजि भजन करौं दिन राती}
\setlength{\mylenthree}{\maxof{\mylenthree}{\mylenfour}}
\setlength{\mylentwo}{\baselineskip}
\setlength{\mylenone}{\mylenone + 1pt}
\setlength{\mylenfour}{\baselineskip}
\setlength{\mylenthree}{\mylenthree + 1pt}
\setlength{\mylen}{(\textwidth - \mylenone)}
\setlength{\mylen}{(\mylen - \mylenthree)*\real{0.5}}
\setlength{\mylen}{(\mylen - 4pt)}
\begin{longtable}[l]{@{\hspace*{\mylen}}>{\setlength\parfillskip{0pt}}p{\mylenone}@{}@{}l@{\hspace{6pt}}>{\setlength\parfillskip{0pt}}p{\mylenthree}@{}@{}l@{}}
 & & & \\[-\the\mylentwo]
अब प्रभु कृपा करहु एहि भाँती & । & सब तजि भजन करौं दिन राती & ॥\\ \nopagebreak
\caption*{(मा.~४.७.२१)}
\end{longtable}
}

\begin{sloppypar}\justifying\hyphenrules{nohyphenation}
\textbf{विभीषणजी}का तो कहना ही क्या! ये तो परम भागवत हैं ही। ये रावणके अत्याचारसे खिन्न होकर भगवान्‌की शरणमें आते हैं। परन्तु इसके पहले भी तो जब ब्रह्माजीने रावण, कुम्भकर्ण और विभीषणजीके पास जाकर भिन्न-भिन्न प्रकारसे वरदान माँगनेके लिये उन्हें प्रेरित किया था, तब दोनों भाइयोंने भिन्न-भिन्न वरदान माँगे, परन्तु विभीषणने तो ब्रह्माजीसे भगवान् श्रीरामके चरणकमलमें निर्मल अनुराग ही माँगा—
\end{sloppypar}

{\bfseries
\setlength{\mylenone}{0pt}
\settowidth{\mylentwo}{गएउ विभीषण पास पुनि कहेउ पुत्र बर माँगु}
\setlength{\mylenone}{\maxof{\mylenone}{\mylentwo}}
\settowidth{\mylentwo}{तेहिं माँगेउ भगवंत पद कमल अमल अनुरागु}
\setlength{\mylenone}{\maxof{\mylenone}{\mylentwo}}
\setlength{\mylentwo}{\baselineskip}
\setlength{\mylenone}{\mylenone + 1pt}
\setlength{\mylen}{(\textwidth - \mylenone)*\real{0.5}}
\begin{longtable}[l]{@{\hspace*{\mylen}}>{\setlength\parfillskip{0pt}}p{\mylenone}@{}@{}l@{}}
 & \\[-\the\mylentwo]
गएउ विभीषण पास पुनि कहेउ पुत्र बर माँगु & ।\\ \nopagebreak
तेहिं माँगेउ भगवंत पद कमल अमल अनुरागु & ॥\\ \nopagebreak
\caption*{(मा.~१.१७७)}
\end{longtable}
}

\begin{sloppypar}\justifying\hyphenrules{nohyphenation}
यही विभीषण हनुमान्‌जीसे कहते हैं—
\end{sloppypar}

{\bfseries
\setlength{\mylenone}{0pt}
\setlength{\mylenthree}{0pt}
\settowidth{\mylentwo}{सुनहु पवनसुत रहनि हमारी}
\setlength{\mylenone}{\maxof{\mylenone}{\mylentwo}}
\settowidth{\mylenfour}{जिमि दशनन महँ जीभ बिचारी}
\setlength{\mylenthree}{\maxof{\mylenthree}{\mylenfour}}
\settowidth{\mylentwo}{तात कबहुँ मोहि जानि अनाथा}
\setlength{\mylenone}{\maxof{\mylenone}{\mylentwo}}
\settowidth{\mylenfour}{करिहैं कृपा भानुकुल नाथा}
\setlength{\mylenthree}{\maxof{\mylenthree}{\mylenfour}}
\setlength{\mylentwo}{\baselineskip}
\setlength{\mylenone}{\mylenone + 1pt}
\setlength{\mylenfour}{\baselineskip}
\setlength{\mylenthree}{\mylenthree + 1pt}
\setlength{\mylen}{(\textwidth - \mylenone)}
\setlength{\mylen}{(\mylen - \mylenthree)*\real{0.5}}
\setlength{\mylen}{(\mylen - 4pt)}
\begin{longtable}[l]{@{\hspace*{\mylen}}>{\setlength\parfillskip{0pt}}p{\mylenone}@{}@{}l@{\hspace{6pt}}>{\setlength\parfillskip{0pt}}p{\mylenthree}@{}@{}l@{}}
 & & & \\[-\the\mylentwo]
सुनहु पवनसुत रहनि हमारी & । & जिमि दशनन महँ जीभ बिचारी & ॥\\
तात कबहुँ मोहि जानि अनाथा & । & करिहैं कृपा भानुकुल नाथा & ॥\\ \nopagebreak
\caption*{(मा.~५.७.१-२)}
\end{longtable}
}

\begin{sloppypar}\justifying\hyphenrules{nohyphenation}
परम\-भागवत विभीषण समराङ्गणमें प्रेमकी अधिकताके कारण जब माधुर्य\-भावसे भावित होकर प्रभु श्रीरामके प्रति सन्देह कर बैठते हैं—“आपके पास युद्धके उपकरण नहीं हैं, आप कैसे रावणको जीत पाएँगे,” तब भगवान् विभीषणको धर्मरथका उपदेश करते हैं।
\end{sloppypar}
\begin{sloppypar}\justifying\hyphenrules{nohyphenation}
\textbf{माँ शबरी} क्या ही विलक्षण महिला हैं! भले ही वे भिल्लकुलमें उत्पन्न हुई हों, कोई वैदिक संस्कारके उनको अधिकार न मिले हों, परन्तु इतना तो है कि भगवान् श्रीरामने उन्हें माँका गौरव दिया, माँ माना। \textbf{भामिनी} शब्द माताके लिये पहले ही प्रयुक्त हुआ है। कपिलदेवने देवहूतिको भागवतमें भामिनी कहा—
\end{sloppypar}

{\bfseries
\setlength{\mylenone}{0pt}
\settowidth{\mylentwo}{भक्तियोगो बहुविधो मार्गैर्भामिनि भाव्यते}
\setlength{\mylenone}{\maxof{\mylenone}{\mylentwo}}
\settowidth{\mylentwo}{स्वभावगुणमार्गेण पुंसां भावो विभिद्यते}
\setlength{\mylenone}{\maxof{\mylenone}{\mylentwo}}
\setlength{\mylentwo}{\baselineskip}
\setlength{\mylenone}{\mylenone + 1pt}
\setlength{\mylen}{(\textwidth - \mylenone)*\real{0.5}}
\begin{longtable}[l]{@{\hspace*{\mylen}}>{\setlength\parfillskip{0pt}}p{\mylenone}@{}@{}l@{}}
 & \\[-\the\mylentwo]
भक्तियोगो बहुविधो मार्गैर्भामिनि भाव्यते & ।\\ \nopagebreak
स्वभावगुणमार्गेण पुंसां भावो विभिद्यते & ॥\\ \nopagebreak
\caption*{(भा.पु.~३.२९.७)}
\end{longtable}
}

\begin{sloppypar}\justifying\hyphenrules{nohyphenation}
हनुमान्‌जीने वाल्मीकीय\-रामायणमें सीताजीको भामिनी कहा—
\end{sloppypar}

{\bfseries
\setlength{\mylenone}{0pt}
\settowidth{\mylentwo}{रामो भामिनि लोकस्य चातुर्वर्ण्यस्य रक्षिता}
\setlength{\mylenone}{\maxof{\mylenone}{\mylentwo}}
\settowidth{\mylentwo}{मर्यादानां च लोकस्य कर्ता कारयिता च सः}
\setlength{\mylenone}{\maxof{\mylenone}{\mylentwo}}
\setlength{\mylentwo}{\baselineskip}
\setlength{\mylenone}{\mylenone + 1pt}
\setlength{\mylen}{(\textwidth - \mylenone)*\real{0.5}}
\begin{longtable}[l]{@{\hspace*{\mylen}}>{\setlength\parfillskip{0pt}}p{\mylenone}@{}@{}l@{}}
 & \\[-\the\mylentwo]
रामो भामिनि लोकस्य चातुर्वर्ण्यस्य रक्षिता & ।\\ \nopagebreak
मर्यादानां च लोकस्य कर्ता कारयिता च सः & ॥\\ \nopagebreak
\caption*{(वा.रा.~५.३५.११)}
\end{longtable}
}

\begin{sloppypar}\justifying\hyphenrules{nohyphenation}
इसी प्रकार भगवान् रामने शबरीजीको मानसमें तीन बार भामिनी कहा—
\end{sloppypar}

{\bfseries
\setlength{\mylenone}{0pt}
\setlength{\mylenthree}{0pt}
\settowidth{\mylentwo}{कह रघुपति सुनु भामिनि बाता}
\setlength{\mylenone}{\maxof{\mylenone}{\mylentwo}}
\settowidth{\mylenfour}{मानउँ एक भगति कर नाता}
\setlength{\mylenthree}{\maxof{\mylenthree}{\mylenfour}}
\setlength{\mylentwo}{\baselineskip}
\setlength{\mylenone}{\mylenone + 1pt}
\setlength{\mylenfour}{\baselineskip}
\setlength{\mylenthree}{\mylenthree + 1pt}
\setlength{\mylen}{(\textwidth - \mylenone)}
\setlength{\mylen}{(\mylen - \mylenthree)*\real{0.5}}
\setlength{\mylen}{(\mylen - 4pt)}
\begin{longtable}[l]{@{\hspace*{\mylen}}>{\setlength\parfillskip{0pt}}p{\mylenone}@{}@{}l@{\hspace{6pt}}>{\setlength\parfillskip{0pt}}p{\mylenthree}@{}@{}l@{}}
 & & & \\[-\the\mylentwo]
कह रघुपति सुनु भामिनि बाता & । & मानउँ एक भगति कर नाता & ॥\\ \nopagebreak
\caption*{(मा.~३.३७.४)}
\end{longtable}
}


{\bfseries
\setlength{\mylenone}{0pt}
\settowidth{\mylentwo}{सोइ अतिशय प्रिय भामिनि मोरे}
\setlength{\mylenone}{\maxof{\mylenone}{\mylentwo}}
\setlength{\mylentwo}{\baselineskip}
\setlength{\mylenone}{\mylenone + 1pt}
\setlength{\mylen}{(\textwidth - \mylenone)*\real{0.5}}
\begin{longtable}[l]{@{\hspace*{\mylen}}>{\setlength\parfillskip{0pt}}p{\mylenone}@{}@{}l@{}}
 & \\[-\the\mylentwo]
सोइ अतिशय प्रिय भामिनि मोरे & ।\\ \nopagebreak
\caption*{(मा.~३.३८.७)}
\end{longtable}
}


{\bfseries
\setlength{\mylenone}{0pt}
\settowidth{\mylentwo}{जनकसुता कइ सुधि भामिनी}
\setlength{\mylenone}{\maxof{\mylenone}{\mylentwo}}
\setlength{\mylentwo}{\baselineskip}
\setlength{\mylenone}{\mylenone + 1pt}
\setlength{\mylen}{(\textwidth - \mylenone)*\real{0.5}}
\begin{longtable}[l]{@{\hspace*{\mylen}}>{\setlength\parfillskip{0pt}}p{\mylenone}@{}@{}l@{}}
 & \\[-\the\mylentwo]
जनकसुता कइ सुधि भामिनी & ।\\ \nopagebreak
\caption*{(मा.~३.३८.१०)}
\end{longtable}
}

\begin{sloppypar}\justifying\hyphenrules{nohyphenation}
शबरीको प्रभुका मातृस्नेह प्राप्त हुआ, माता बनाया भगवान्‌ने शबरीको। भावुकजन विशेष जाननेके लिये मेरे द्वारा रचित \textbf{माँ शबरी} ग्रन्थ पढ़ें।
\end{sloppypar}
\begin{sloppypar}\justifying\hyphenrules{nohyphenation}
\textbf{खगपति} अर्थात् \textbf{जटायुजी}—यही खगोंमें श्रेष्ठ हैं। यद्यपि \textbf{खगपति} शब्दसे गरुड अभिहित होते हैं, परन्तु भक्तमालकारने खगपति जटायुजीको ही कहा, सबसे श्रेष्ठ यही हैं, यही पक्षिराज हैं, जिन्हें परमात्माने अपना पिता बनाया और दशरथजीकी अपेक्षा दशगुनी अधिक भक्तिसे युक्त होकर भगवान्‌ने जटायुजीका दाहसंस्कार अपने ही हाथसे किया—
\end{sloppypar}

{\bfseries
\setlength{\mylenone}{0pt}
\settowidth{\mylentwo}{दसरथ तें दसगुन भगति सहित तासु करि काज}
\setlength{\mylenone}{\maxof{\mylenone}{\mylentwo}}
\settowidth{\mylentwo}{सोचत बंधु समेत प्रभु कृपासिंधु रघुराज}
\setlength{\mylenone}{\maxof{\mylenone}{\mylentwo}}
\setlength{\mylentwo}{\baselineskip}
\setlength{\mylenone}{\mylenone + 1pt}
\setlength{\mylen}{(\textwidth - \mylenone)*\real{0.5}}
\begin{longtable}[l]{@{\hspace*{\mylen}}>{\setlength\parfillskip{0pt}}p{\mylenone}@{}@{}l@{}}
 & \\[-\the\mylentwo]
दसरथ तें दसगुन भगति सहित तासु करि काज & ।\\ \nopagebreak
सोचत बंधु समेत प्रभु कृपासिंधु रघुराज & ॥\\ \nopagebreak
\caption*{(दो.~२२७)}
\end{longtable}
}

\begin{sloppypar}\justifying\hyphenrules{nohyphenation}
श्रीरामने जटायुको गोदमें लिया—ऐसे परमभक्त जटायु, जिन्होंने भगवान्‌से कुछ नहीं माँगा और एक बात कह दी—“प्रभो! पिताकी मर्यादामें मुझे भी तो रहना पड़ेगा। आपके जीवनमें दो पिता आए—चक्रवर्ती महाराज दशरथ और मैं जटायु। दशरथजी पुत्र\-वियोगमें अपने प्राण त्याग सकते हैं, तो मैं भी पुत्रवधूके वियोगमें अपने प्राण त्याग दूँगा, क्योंकि सीताजीकी रक्षा मैं नहीं कर पाया।” जटायुके लिये ही शुकाचार्यने श्रीरामके संबन्धमें \textbf{प्रियविरहरुषा} (भा.पु.~९.१०.४) कहा। \textbf{प्रियविरहरुषा}का तात्पर्य है \textbf{प्रियेण जटायुषा विरहः प्रियविरहः तेन रुट् प्रियविरहरुट् तया प्रियविरहरुषा}। अर्थात् अपने अत्यन्त प्रिय पिता श्रीजटायुसे जब वियोग हुआ, तब भगवान् रामको क्रोध आ गया और उन्होंने रावणके वधकी प्रतिज्ञा कर ली।
\end{sloppypar}
\begin{sloppypar}\justifying\hyphenrules{nohyphenation}
इसके पश्चात् \textbf{ध्रुव} जिन्हें पाँचवें वर्षमें ही दर्शन देनेके लिये भगवान्‌को एक नया अवतार सहस्रशीर्षावतार लेना पड़ा—
\end{sloppypar}

{\bfseries
\setlength{\mylenone}{0pt}
\settowidth{\mylentwo}{त एवमुत्सन्नभया उरुक्रमे कृतावनामाः प्रययुस्त्रिविष्टपम्}
\setlength{\mylenone}{\maxof{\mylenone}{\mylentwo}}
\settowidth{\mylentwo}{सहस्रशीर्षापि ततो गरुत्मता मधोर्वनं भृत्यदिदृक्षया गतः}
\setlength{\mylenone}{\maxof{\mylenone}{\mylentwo}}
\setlength{\mylentwo}{\baselineskip}
\setlength{\mylenone}{\mylenone + 1pt}
\setlength{\mylen}{(\textwidth - \mylenone)*\real{0.5}}
\begin{longtable}[l]{@{\hspace*{\mylen}}>{\setlength\parfillskip{0pt}}p{\mylenone}@{}@{}l@{}}
 & \\[-\the\mylentwo]
त एवमुत्सन्नभया उरुक्रमे कृतावनामाः प्रययुस्त्रिविष्टपम् & ।\\ \nopagebreak
सहस्रशीर्षापि ततो गरुत्मता मधोर्वनं भृत्यदिदृक्षया गतः & ॥\\ \nopagebreak
\caption*{(भा.पु.~४.९.१)}
\end{longtable}
}

\begin{sloppypar}\justifying\hyphenrules{nohyphenation}
ऐसे ध्रुव!
\end{sloppypar}
\begin{sloppypar}\justifying\hyphenrules{nohyphenation}
\textbf{उद्धव}—जो भगवान् श्रीकृष्णचन्द्रजीके परम मित्र हैं, उनके लिये कहा जाता है—
\end{sloppypar}

{\bfseries
\setlength{\mylenone}{0pt}
\settowidth{\mylentwo}{वृष्णीनां प्रवरो मन्त्री कृष्णस्य दयितः सखा}
\setlength{\mylenone}{\maxof{\mylenone}{\mylentwo}}
\settowidth{\mylentwo}{शिष्यो बृहस्पतेः साक्षादुद्धवो बुद्धिसत्तमः}
\setlength{\mylenone}{\maxof{\mylenone}{\mylentwo}}
\setlength{\mylentwo}{\baselineskip}
\setlength{\mylenone}{\mylenone + 1pt}
\setlength{\mylen}{(\textwidth - \mylenone)*\real{0.5}}
\begin{longtable}[l]{@{\hspace*{\mylen}}>{\setlength\parfillskip{0pt}}p{\mylenone}@{}@{}l@{}}
 & \\[-\the\mylentwo]
वृष्णीनां प्रवरो मन्त्री कृष्णस्य दयितः सखा & ।\\ \nopagebreak
शिष्यो बृहस्पतेः साक्षादुद्धवो बुद्धिसत्तमः & ॥\\ \nopagebreak
\caption*{(भा.पु.~१०.४६.१)}
\end{longtable}
}

\begin{sloppypar}\justifying\hyphenrules{nohyphenation}
\textbf{अम्बरीष}—इनकी चर्चा भागवतमें नवम स्कन्धमें चौथे और पाँचवें अध्यायमें की गई है। चरणामृतका महत्त्व ख्यापित करनेके लिये महर्षि दुर्वासा पारणाकी अवधिका उल्लङ्घन करके अम्बरीषके पास आए, तब तक अम्बरीषजीने वसिष्ठजीके अनुरोधसे भगवान्‌का चरणामृत लेकर पारणा कर ली थी। कुपित होकर दुर्वासाने कृत्याका प्रयोग किया, जो सुदर्शन चक्र द्वारा विफल किया गया, और सुदर्शन चक्रने दुर्वासाका पीछा किया। सर्वत्र भ्रमण करनेपर भी जब दुर्वासाकी कहींसे रक्षा नहीं हो सकी, तो भगवान् नारायणने कह दिया कि तुम अम्बरीषके पास जाओ, वहीं तुम्हारी रक्षा सम्भव है। दुर्वासा अम्बरीषके पास आए और अम्बरीषने प्रार्थना कर ली, दुर्वासाकी रक्षा हो गई। अन्यत्र तो दुर्वासाने कोप करके अम्बरीषको दस जन्मका शाप दिया तो भगवान्‌ने अम्बरीषके शापको स्वयं स्वीकार करके दशावतार स्वीकार कर लिया—
\end{sloppypar}

{\bfseries
\setlength{\mylenone}{0pt}
\settowidth{\mylentwo}{अम्बरीष हित लागि कृपानिधि सो जन्मे दस बार}
\setlength{\mylenone}{\maxof{\mylenone}{\mylentwo}}
\setlength{\mylentwo}{\baselineskip}
\setlength{\mylenone}{\mylenone + 1pt}
\setlength{\mylen}{(\textwidth - \mylenone)*\real{0.5}}
\begin{longtable}[l]{@{\hspace*{\mylen}}>{\setlength\parfillskip{0pt}}p{\mylenone}@{}@{}l@{}}
 & \\[-\the\mylentwo]
अम्बरीष हित लागि कृपानिधि सो जन्मे दस बार & ।\\ \nopagebreak
\caption*{(वि.प.~९८.५)}
\end{longtable}
}

\begin{sloppypar}\justifying\hyphenrules{nohyphenation}
\textbf{विदुर}—जो व्यासजीके संकल्पसे विचित्रवीर्यकी एक दासीके गर्भसे जन्मे। मुनि माण्डव्यके शापसे स्वयं यमराज ही विदुर बनकर आए थे। उनकी प्रीतिका वर्णन तो तब स्पष्ट हो जाता है, जब भगवान् कृष्ण दुर्योधनको समझानेके लिये दूत बनकर हास्तिनपुर आते हैं, और वहाँ दुर्योधनका आमन्त्रण ठुकराकर भगवान् विदुरजीके यहाँ जाकर केलेका छिलका और बथुएका साग खाते हैं। प्रसिद्ध ही है—\textbf{दुर्योधन घर मेवा त्यागे साग बिदुर घर खाई}। इनका विशेष चरित्र जाननेके लिये मेरे द्वारा लिखा हुआ \textbf{काका विदुर} नामक खण्डकाव्य पढ़िये।
\end{sloppypar}
\begin{sloppypar}\justifying\hyphenrules{nohyphenation}
\textbf{अक्रूर}—ये भगवान्‌के परम अन्तरङ्ग हैं। कंसके द्वारा जब इन्हें भेजा गया, इनके मनका एक मनोरथ था—\textbf{मां वक्ष्यतेऽक्रूर ततेत्युरुश्रवाः} (भा.पु.~१०.३८.२१)। एक बार भगवान् मुझे काका कह दें, मैं धन्य हो जाऊँगा। भगवान्‌ने वैसा ही किया। श्रीव्रजभूमिमें भगवान्‌की चरण\-रेखाओंको देखकर अक्रूरके मनमें जो उद्गार प्रकट हुआ, वह तो देखते ही बनता है—
\end{sloppypar}

{\bfseries
\setlength{\mylenone}{0pt}
\settowidth{\mylentwo}{पदानि तस्याखिललोकपालकिरीटजुष्टामलपादरेणोः}
\setlength{\mylenone}{\maxof{\mylenone}{\mylentwo}}
\settowidth{\mylentwo}{ददर्श गोष्ठे क्षितिकौतुकानि विलक्षितान्यब्जयवाङ्कुशाद्यैः}
\setlength{\mylenone}{\maxof{\mylenone}{\mylentwo}}
\settowidth{\mylentwo}{तद्दर्शनाह्लादविवृद्धसम्भ्रमः प्रेम्णोर्ध्वरोमाश्रुकलाकुलेक्षणः}
\setlength{\mylenone}{\maxof{\mylenone}{\mylentwo}}
\settowidth{\mylentwo}{रथादवस्कन्द्य स तेष्वचेष्टत प्रभोरमून्यङ्घ्रिरजांस्यहो इति}
\setlength{\mylenone}{\maxof{\mylenone}{\mylentwo}}
\setlength{\mylentwo}{\baselineskip}
\setlength{\mylenone}{\mylenone + 1pt}
\setlength{\mylen}{(\textwidth - \mylenone)*\real{0.5}}
\begin{longtable}[l]{@{\hspace*{\mylen}}>{\setlength\parfillskip{0pt}}p{\mylenone}@{}@{}l@{}}
 & \\[-\the\mylentwo]
पदानि तस्याखिललोकपालकिरीटजुष्टामलपादरेणोः & ।\\ \nopagebreak
ददर्श गोष्ठे क्षितिकौतुकानि विलक्षितान्यब्जयवाङ्कुशाद्यैः & ॥\\
तद्दर्शनाह्लादविवृद्धसम्भ्रमः प्रेम्णोर्ध्वरोमाश्रुकलाकुलेक्षणः & ।\\ \nopagebreak
रथादवस्कन्द्य स तेष्वचेष्टत प्रभोरमून्यङ्घ्रिरजांस्यहो इति & ॥\\ \nopagebreak
\caption*{(भा.पु.~१०.३८.२५-२६)}
\end{longtable}
}

\begin{sloppypar}\justifying\hyphenrules{nohyphenation}
भागवतजीके दशम स्कन्धके अड़तीसवें अध्यायका यह प्रकरण देखने ही लायक है। अक्रूरने जब श्रीव्रजमें भगवान्‌के चरणचिह्नोंके दर्शन किये, उससे उनके मनमें सात्त्विक भाव जगा, उन्हें रोमाञ्च हो उठा, और उनके नेत्रोंसे अश्रुपात होने लगा। अक्रूर रथसे लुढ़ककर व्रजभूमिकी उस धूलिमें लोटने लगे, जिसे आज \textbf{रमण रेती} कहा जाता है।
\end{sloppypar}
\begin{sloppypar}\justifying\hyphenrules{nohyphenation}
\textbf{सुदामा}—भगवान् श्रीकृष्णके विद्यार्थी-मित्र हैं। दोनों विद्याध्ययन करके अपने-अपने प्रवृत्तमें संलग्न हुए। प्रभु द्वारकाधीश बन बैठे और सुदामाजीको लक्ष्मीजीकी बड़ी बहनने वरण कर लिया अर्थात् वे दरिद्र हो गए, दरिद्रापति हो गए। एक दिन सुदामाजीकी धर्मपत्नीने यह कहा—“यदि भगवान् आपके मित्र हैं तो आप उनके पास जाएँ, वे आपको बहुत-सा धन देंगे।” सुदामाने जब धनके प्रति अनिच्छा व्यक्त की तो सुशीलाजीने कहा—“तो आप दर्शनके लिये तो उनके पास जा ही सकते हैं।” द्वारकाधीशके पास सुदामाजी आए। द्वारकाधीशजीने उनका बहुत सम्मान किया, गले मिले और उनके चरणोंको अपने हाथसे धोया। क्या ही नरोत्तमदासने कहा है—
\end{sloppypar}

{\bfseries
\setlength{\mylenone}{0pt}
\settowidth{\mylentwo}{ऐसे बेहाल बेवाइन ते पद कंटक जाल लगे पुनि जोये}
\setlength{\mylenone}{\maxof{\mylenone}{\mylentwo}}
\settowidth{\mylentwo}{हाय महादुख पायो सखा तुम आये इतै न किते दिन खोये}
\setlength{\mylenone}{\maxof{\mylenone}{\mylentwo}}
\settowidth{\mylentwo}{देखि सुदामा की दीन दसा करुना करिके करुनानिधि रोये}
\setlength{\mylenone}{\maxof{\mylenone}{\mylentwo}}
\settowidth{\mylentwo}{पानि परात को हाथ छुयो नहिं नैनन के जल सों पग धोये}
\setlength{\mylenone}{\maxof{\mylenone}{\mylentwo}}
\setlength{\mylentwo}{\baselineskip}
\setlength{\mylenone}{\mylenone + 1pt}
\setlength{\mylen}{(\textwidth - \mylenone)*\real{0.5}}
\begin{longtable}[l]{@{\hspace*{\mylen}}>{\setlength\parfillskip{0pt}}p{\mylenone}@{}@{}l@{}}
 & \\[-\the\mylentwo]
ऐसे बेहाल बेवाइन ते पद कंटक जाल लगे पुनि जोये & ।\\ \nopagebreak
हाय महादुख पायो सखा तुम आये इतै न किते दिन खोये & ॥\\
देखि सुदामा की दीन दसा करुना करिके करुनानिधि रोये & ।\\ \nopagebreak
पानि परात को हाथ छुयो नहिं नैनन के जल सों पग धोये & ॥
\end{longtable}
}

\begin{sloppypar}\justifying\hyphenrules{nohyphenation}
सुदामाजीका चरित्र भागवतजीके दशम स्कन्धके ८०वें और ८१वें अध्यायोंमें उपनिबद्ध है, अद्भुत झाँकी है। यद्यपि भागवतजीमें इनका सुदामा नाम नहीं लिखा है, परन्तु अन्य पुराणोंसे यह नाम स्पष्ट हो जाता है। स्कन्दपुराणके रेवाखण्डमें सत्यनारायण\-व्रतकथामें स्पष्ट लिखा ही गया है—
\end{sloppypar}

{\bfseries
\setlength{\mylenone}{0pt}
\settowidth{\mylentwo}{शतानन्दो महाप्राज्ञः सुदामा ब्राह्मणो बभूव}
\setlength{\mylenone}{\maxof{\mylenone}{\mylentwo}}
\setlength{\mylentwo}{\baselineskip}
\setlength{\mylenone}{\mylenone + 1pt}
\setlength{\mylen}{(\textwidth - \mylenone)*\real{0.5}}
\begin{longtable}[l]{@{\hspace*{\mylen}}>{\setlength\parfillskip{0pt}}p{\mylenone}@{}@{}l@{}}
 & \\[-\the\mylentwo]
शतानन्दो महाप्राज्ञः सुदामा ब्राह्मणो बभूव & ।\\ \nopagebreak
\caption*{(स्क.पु.रे.ख.स.क.~५.१९)}
\end{longtable}
}

\begin{sloppypar}\justifying\hyphenrules{nohyphenation}
\textbf{चन्द्रहास}—इनकी कथा जैमिनीयाश्व\-मेध\-पर्वमें महाभारतमें लिखी गई है। क्या व्यक्तित्व है चन्द्रहासका! जन्मते ही पिता-माताका वियोग हुआ, अनाथवत् भ्रमण करते रहे। एक दिन कुन्तलपुर राज्यके मन्त्री धृष्टबुद्धिके यहाँ प्रीतिभोजमें चन्द्रहास भी आ गए। ज्योतिषियोंने कह दिया कि यही निष्किञ्चन बालक धृष्टबुद्धिकी बेटीका पति बनेगा। धृष्टबुद्धिने उन्हें मारनेके लिये वधिकोंको आदेश दिया। चन्द्रहास नारदजी द्वारा दिये हुए शालग्रामजीकी सेवा किया करते थे और फिर अपने मुखमें रख लेते थे। उस दिन भी उन्होंने यही किया। वधिकोंसे कहा—“पहले मुझे शालग्रामकी सेवा कर लेने दो, फिर मुझे मार डालना।” भावनासे सेवा की और तब यह कहा—“प्रभु! आज यह अन्तिम सेवा है।” सेवा करके मुखमें भर लिया और वधिकोंको चरणामृत दिया। वधिकोंकी बुद्धि बदल गई, उन्होंने केवल चन्द्रहासजीकी छठी उँगली काटकर धृष्टबुद्धिको दिखा दिया। संयोगसे चन्दनावतीके राजा इन्हें अपने घर ले आए, वे निःसंतान थे, उन्होंने इनको राजा बना दिया। और अब तो कुन्तलपुरको कर देना क्या, चन्द्रहास स्वयं राज्य करने लगे। कुन्तलपुरके उपराज्यमें धृष्टबुद्धि आया और उसने चन्द्रहासको देखा। वह पहचान गया कि यह तो वही बालक है। अन्तमें उसने इनसे कहा—“मैं थोड़ा यहाँ विश्राम करूँगा, तुम मेरे पुत्रको जाकर यह संदेश दे आओ।” धृष्टबुद्धिने एक श्लोक लिखा—
\end{sloppypar}

{\bfseries
\setlength{\mylenone}{0pt}
\settowidth{\mylentwo}{विषमस्मै प्रदातव्यं त्वया मदन शत्रवे}
\setlength{\mylenone}{\maxof{\mylenone}{\mylentwo}}
\settowidth{\mylentwo}{कार्याकार्यं न द्रष्टव्यं कर्तव्यं खलु मे प्रियम्}
\setlength{\mylenone}{\maxof{\mylenone}{\mylentwo}}
\setlength{\mylentwo}{\baselineskip}
\setlength{\mylenone}{\mylenone + 1pt}
\setlength{\mylen}{(\textwidth - \mylenone)*\real{0.5}}
\begin{longtable}[l]{@{\hspace*{\mylen}}>{\setlength\parfillskip{0pt}}p{\mylenone}@{}@{}l@{}}
 & \\[-\the\mylentwo]
विषमस्मै प्रदातव्यं त्वया मदन शत्रवे & ।\\ \nopagebreak
कार्याकार्यं न द्रष्टव्यं कर्तव्यं खलु मे प्रियम् & ॥
\end{longtable}
}

\begin{sloppypar}\justifying\hyphenrules{nohyphenation}
अर्थात् “हे मदनसेन! इस शत्रुको तुम विष दे देना,”—यह पत्र लिखकर दिया। चन्द्रहास कुन्तलपुरके निकट एक बागमें आकर भगवान्‌की सेवा करके थोड़ा-सा विश्राम करने लगे। संयोगसे धृष्टबुद्धिकी पुत्री विषया वहाँ आई, चन्द्रहासके सौन्दर्यको देखकर वह मुग्ध हुई। सहसा उसकी दृष्टि पड़ गई चन्द्रहासकी पगड़ीपर, जिसमें यह पत्र रखा था। उसने कुतूहलवशात् पत्र निकालकर पढ़ा कि अरे! मेरे पिताने इस युवकको विष देनेको कह दिया? अपनी आँखके काजलसे ‘म’को उसने ‘या’ बना दिया, और ‘प्रदातव्यं’के स्थानपर ‘प्रदातव्या’ कर दिया अर्थात् \textbf{विषयास्मै प्रदातव्या} कर दिया जिसका अर्थ हुआ—“इस युवकको तुम मेरी विषया नामक कन्या दे देना।” मदनसेनको चन्द्रहासने वह पत्र दिया। मदनसेन प्रसन्न हुए। विवाहका आयोजन हुआ और अपनी बहनको उन्होंने प्रेमपूर्वक चन्द्रहासजीको प्रदान कर दिया। संयोगसे थोड़े दिनके पश्चात् जब धृष्टबुद्धि आया, तो यहाँ तो कुछ परिस्थिति ही बदल गई थी। चन्द्रहास धृष्टबुद्धिके जामाता बन गए थे। उसने मदनसेनसे पूछा तो मदनसेनने कह दिया—“आपने पत्रमें यही लिखा है।” पत्र दिखा दिया, उसे आश्चर्य हुआ, बोला—“कोई बात नहीं!” अन्तमें उसने कुछ वधिकोंको कहा कि आज जो देवीपूजनमें आए उसका वध कर देना और चन्द्रहाससे कह दिया—“आप अकेले जाकर देवीपूजन कर आइये।” चन्द्रहासको क्या? ये तो चल पड़े। उधर कुन्तलपुरके राजाने भी यह कह दिया कि मेरे पास कोई सन्तान नहीं है, अब मैं यह राज्य चन्द्रहासको सौंपना चाहता हूँ। राजाने मदनसेनसे कहा—“तुम तुरन्त जाकर अपने जीजाको राजसभामें ले आओ।” मदनसेन आए और उन्होंने चन्द्रहाससे कहा—“भगवन्! आप कुन्तलपुरकी राजसभामें पधारें, आपका राज्याभिषेक होगा। आपके स्थानपर मैं ही देवीपूजन कर लेता हूँ।” धृष्टबुद्धिके निर्देशानुसार वधिकोंने वही किया। उनको क्या पता था कि जो देवीपूजन करने आया है वह चन्द्रहास है या मदनसेन। वधिकोंने मदनसेनकी हत्या कर दी। यह समाचार जब धृष्टबुद्धिको मिला, तो उसने भी छातीपर पत्थर मारकर अपनी हत्या कर ली। अन्तमें चन्द्रहासजीने देवीके समक्ष स्वयं तलवार लेकर अपनी हत्या करनी चाही, तो भगवतीजीने रोका। फिर चन्द्रहासजीने कहा कि तब इन दोनोंको जिला दिया जाए। देवीने दोनोंको जिला दिया। चन्द्रहास\-प्रसंगपर गोस्वामी तुलसीदासजीने तुलसी सतसईमें एक दोहा लिखा—
\end{sloppypar}

{\bfseries
\setlength{\mylenone}{0pt}
\settowidth{\mylentwo}{जाके पग नहि पानहीं ताहि दीन्ह गजराज}
\setlength{\mylenone}{\maxof{\mylenone}{\mylentwo}}
\settowidth{\mylentwo}{बिषहिं देत बिषया दई राम गरीब निवाज}
\setlength{\mylenone}{\maxof{\mylenone}{\mylentwo}}
\setlength{\mylentwo}{\baselineskip}
\setlength{\mylenone}{\mylenone + 1pt}
\setlength{\mylen}{(\textwidth - \mylenone)*\real{0.5}}
\begin{longtable}[l]{@{\hspace*{\mylen}}>{\setlength\parfillskip{0pt}}p{\mylenone}@{}@{}l@{}}
 & \\[-\the\mylentwo]
जाके पग नहि पानहीं ताहि दीन्ह गजराज & ।\\ \nopagebreak
बिषहिं देत बिषया दई राम गरीब निवाज & ॥\\ \nopagebreak
\caption*{(तु.स.स.)}
\end{longtable}
}

\begin{sloppypar}\justifying\hyphenrules{nohyphenation}
चन्द्रहासके पश्चात् \textbf{चित्रकेतुजी}की कथा भागवतजीके षष्ठ स्कन्धमें प्रसिद्ध ही है। \textbf{ग्राह} और \textbf{गज}की कथा भी भागवतजीके अष्टम स्कन्धमें द्वितीयसे लेकर चतुर्थ अध्याय तक बहुत व्यापक रूपमें कही गई है। और \textbf{पाण्डवों}की कथा संपूर्ण महाभारतमें प्रसिद्ध ही है। पाण्डवोंके लिये एक श्लोक बहुत प्रसिद्ध है—
\end{sloppypar}

{\bfseries
\setlength{\mylenone}{0pt}
\settowidth{\mylentwo}{धर्मो विवर्धति युधिष्ठिरकीर्तनेन शत्रुर्विनश्यति वृकोदरकीर्तनेन}
\setlength{\mylenone}{\maxof{\mylenone}{\mylentwo}}
\settowidth{\mylentwo}{तेजो विवर्धति धनञ्जयकीर्तनेन माद्रीसुतौ कथयतां न भयं नराणाम्}
\setlength{\mylenone}{\maxof{\mylenone}{\mylentwo}}
\setlength{\mylentwo}{\baselineskip}
\setlength{\mylenone}{\mylenone + 1pt}
\setlength{\mylen}{(\textwidth - \mylenone)*\real{0.5}}
\begin{longtable}[l]{@{\hspace*{\mylen}}>{\setlength\parfillskip{0pt}}p{\mylenone}@{}@{}l@{}}
 & \\[-\the\mylentwo]
धर्मो विवर्धति युधिष्ठिरकीर्तनेन शत्रुर्विनश्यति वृकोदरकीर्तनेन & ।\\ \nopagebreak
तेजो विवर्धति धनञ्जयकीर्तनेन माद्रीसुतौ कथयतां न भयं नराणाम् & ॥\\ \nopagebreak
\caption*{(पा.गी.~२)}
\end{longtable}
}

\begin{sloppypar}\justifying\hyphenrules{nohyphenation}
अर्थात् युधिष्ठिरजीका संकीर्तन करनेसे धर्म बढ़ता है, भीमसेनजीका संकीर्तन करनेसे शत्रुओंका नाश होता है, अर्जुनजीका संकीर्तन करनेसे तेजोवृद्धि होती है और माद्रीपुत्रोंका स्मरण करनेसे मनुष्य अभय हो जाता है। इस प्रकार पाण्डव धन्य हैं, जिनके लिये भगवान् क्या-क्या नहीं करते! कभी दूत बन जाते हैं, कभी सूत बन जाते हैं, कभी मन्त्री बन जाते हैं। स्वयं भागवतजीके सप्तम स्कन्धके दशम अध्यायके ४८वें श्लोकमें प्रह्लाद\-कथाका उपसंहार करते हुए नारदजी कहते हैं कि पाण्डवों! इस संसारमें आप लोग बहुत भाग्यशाली हैं। आप लोगोंके घरमें साक्षात् भगवान् परब्रह्म परमात्मा श्रीकृष्णचन्द्रजी अपने ऐश्वर्यको छिपाकर मनुष्य रूपमें विराज रहे हैं। देख रहे हो, जिन्हें योगी लोग ध्यानमें नहीं पाते, वे ही आज आपके राजसूय यज्ञमें नाई बनकर जूठन उठा रहे हैं और ब्राह्मणोंका चरण\-प्रक्षालन कर रहे हैं—
\end{sloppypar}

{\bfseries
\setlength{\mylenone}{0pt}
\settowidth{\mylentwo}{यूयं नृलोके बत भूरिभागा लोकं पुनाना मुनयोऽभियन्ति}
\setlength{\mylenone}{\maxof{\mylenone}{\mylentwo}}
\settowidth{\mylentwo}{येषां गृहानावसतीति साक्षाद्गूढं परं ब्रह्म मनुष्यलिङ्गम्}
\setlength{\mylenone}{\maxof{\mylenone}{\mylentwo}}
\setlength{\mylentwo}{\baselineskip}
\setlength{\mylenone}{\mylenone + 1pt}
\setlength{\mylen}{(\textwidth - \mylenone)*\real{0.5}}
\begin{longtable}[l]{@{\hspace*{\mylen}}>{\setlength\parfillskip{0pt}}p{\mylenone}@{}@{}l@{}}
 & \\[-\the\mylentwo]
यूयं नृलोके बत भूरिभागा लोकं पुनाना मुनयोऽभियन्ति & ।\\ \nopagebreak
येषां गृहानावसतीति साक्षाद्गूढं परं ब्रह्म मनुष्यलिङ्गम् & ॥\\ \nopagebreak
\caption*{(भा.पु.~७.१०.४८)}
\end{longtable}
}

\begin{sloppypar}\justifying\hyphenrules{nohyphenation}
\textbf{कौषारव} अर्थात् \textbf{मैत्रेय}। मैत्रेयजीके पिताका नाम है \textbf{कुषारव}, उनके पुत्र होनेसे इन्हें कहते हैं \textbf{कौषारव}। कौषारव वेदव्यासजीके मित्र हैं और गोलोक\-प्रस्थान करते समय भगवान् श्रीकृष्णने उद्धवको यह संकेत किया था कि वे विदुरजीसे कहें कि वे जाकर मैत्रेयजीसे ही भागवतजीका श्रवण कर लें।
\end{sloppypar}
\begin{sloppypar}\justifying\hyphenrules{nohyphenation}
\textbf{कुन्ती}—ये भगवान् श्रीकृष्णकी बुआ हैं। परन्तु इनकी रीति विलक्षण है, सब तो भगवान्‌से संपत्ति माँगते हैं, और इन्होंने भगवान्‌से विपत्ति माँगी—
\end{sloppypar}

{\bfseries
\setlength{\mylenone}{0pt}
\settowidth{\mylentwo}{विपदः सन्तु नः शश्वत्तत्र तत्र जगद्गुरो}
\setlength{\mylenone}{\maxof{\mylenone}{\mylentwo}}
\settowidth{\mylentwo}{भवतो दर्शनं यत्स्यादपुनर्भवदर्शनम्}
\setlength{\mylenone}{\maxof{\mylenone}{\mylentwo}}
\setlength{\mylentwo}{\baselineskip}
\setlength{\mylenone}{\mylenone + 1pt}
\setlength{\mylen}{(\textwidth - \mylenone)*\real{0.5}}
\begin{longtable}[l]{@{\hspace*{\mylen}}>{\setlength\parfillskip{0pt}}p{\mylenone}@{}@{}l@{}}
 & \\[-\the\mylentwo]
विपदः सन्तु नः शश्वत्तत्र तत्र जगद्गुरो & ।\\ \nopagebreak
भवतो दर्शनं यत्स्यादपुनर्भवदर्शनम् & ॥\\ \nopagebreak
\caption*{(भा.पु.~१.८.२५)}
\end{longtable}
}

\begin{sloppypar}\justifying\hyphenrules{nohyphenation}
हे प्रभु! आप हमें निरन्तर विपत्ति ही दीजिये, जिससे आपके दर्शन होते रहें। यही कुन्ती हैं, जिन्होंने अर्जुनके मुखसे जब भगवान्‌का लीला\-संवरण सुना और उनकी गोलोक\-यात्रा सुनी, तो तुरन्त अपने प्राण छोड़ दिए। यथा—
\end{sloppypar}

{\bfseries
\setlength{\mylenone}{0pt}
\settowidth{\mylentwo}{पृथाप्यनुश्रुत्य धनञ्जयोदितं नाशं यदूनां भगवद्गतिं च ताम्}
\setlength{\mylenone}{\maxof{\mylenone}{\mylentwo}}
\settowidth{\mylentwo}{एकान्तभक्त्या भगवत्यधोक्षजे निवेशितात्मोपरराम संसृतेः}
\setlength{\mylenone}{\maxof{\mylenone}{\mylentwo}}
\setlength{\mylentwo}{\baselineskip}
\setlength{\mylenone}{\mylenone + 1pt}
\setlength{\mylen}{(\textwidth - \mylenone)*\real{0.5}}
\begin{longtable}[l]{@{\hspace*{\mylen}}>{\setlength\parfillskip{0pt}}p{\mylenone}@{}@{}l@{}}
 & \\[-\the\mylentwo]
पृथाप्यनुश्रुत्य धनञ्जयोदितं नाशं यदूनां भगवद्गतिं च ताम् & ।\\ \nopagebreak
एकान्तभक्त्या भगवत्यधोक्षजे निवेशितात्मोपरराम संसृतेः & ॥\\ \nopagebreak
\caption*{(भा.पु.~१.१५.३३)}
\end{longtable}
}

\begin{sloppypar}\justifying\hyphenrules{nohyphenation}
\textbf{कुन्तीवधू}—द्रौपदीजीके लिये भक्तमालकार \textbf{कुन्तीवधू} इसलिये कहते हैं कि ये कुन्तीकी वास्तविक पुत्रवधू हैं। कुन्तीके ही अनुरोधपर इन्होंने पाँच पतियोंको स्वीकारा, अपना सब कुछ मिटा डाला, लौकिक कलङ्क भी सहा, कर्णके व्यङ्ग्यवचन सहे—यह सब केवल कुन्तीजीके कारण। इसलिये इन्हें कुन्तीवधू कहा गया। \textbf{पट ऐंचत लज्जाहरी}में \textbf{लज्जाहरी} शब्दमें \textbf{लज्जा आहरी}—यह पदच्छेद समझना चाहिये, अर्थात् जब दुःशासन द्रौपदीजीके वस्त्रोंको खींच रहा था, तब भगवान्‌ने उनकी लज्जाका आहरण किया अर्थात् लज्जा लौटा दी।
\end{sloppypar}
\begin{sloppypar}\justifying\hyphenrules{nohyphenation}
ऐसे जो श्रीहरिको प्रिय हैं और जिनको श्रीहरि प्रिय हैं, उनसे नाभाजी प्रार्थना करते हैं—\textbf{जिन चरनरेनु आशा धरी}, जिन्होंने भगवान्‌के चरण\-कमलोंकी धूलिको प्राप्त करनेके लिये अपनी आशा धारण की अर्थात् जीवन\-यात्रा सतत रखी है कि कभी-न-कभी चरणधूलि मिलेगी, और \textbf{जिन चरनरेनु आशा धरी}, नाभाजी कहते हैं कि इन्हीं हरिवल्लभोंके चरण\-कमलकी धूलिको प्राप्त करनेके लिये मैंने भी अपने जीवनकी आशा धारण की है कि कभी-न-कभी इनकी चरण\-धूलि मुझे मिलेगी।
\end{sloppypar}

\addcontentsline{toc}{section}{\texorpdfstring{पद १०: जिनके हरि नित उर बसैं}{१०: जिनके हरि नित उर बसैं}}

{\relscale{1.1875}
{\bfseries
\setlength{\mylenone}{0pt}
\settowidth{\mylentwo}{}
\setlength{\mylenone}{\maxof{\mylenone}{\mylentwo}}
\settowidth{\mylentwo}{पदपंकज बाँछौं सदा जिनके हरि उर नित बसैं}
\setlength{\mylenone}{\maxof{\mylenone}{\mylentwo}}
\settowidth{\mylentwo}{योगेश्वर श्रुतदेव अंग मुचु(कुंद) प्रियब्रत जेता}
\setlength{\mylenone}{\maxof{\mylenone}{\mylentwo}}
\settowidth{\mylentwo}{पृथू परीक्षित शेष सूत शौनक परचेता}
\setlength{\mylenone}{\maxof{\mylenone}{\mylentwo}}
\settowidth{\mylentwo}{शतरूपा त्रय सुता सुनीति सति सबहि मँदालस}
\setlength{\mylenone}{\maxof{\mylenone}{\mylentwo}}
\settowidth{\mylentwo}{जग्यपत्नि ब्रजनारि किये केशव अपने बस}
\setlength{\mylenone}{\maxof{\mylenone}{\mylentwo}}
\settowidth{\mylentwo}{ऐसे नर नारी जिते तिनही के गाऊँ जसैं}
\setlength{\mylenone}{\maxof{\mylenone}{\mylentwo}}
\settowidth{\mylentwo}{पदपंकज बाँछौं सदा जिनके हरि उर नित बसैं}
\setlength{\mylenone}{\maxof{\mylenone}{\mylentwo}}
\setlength{\mylentwo}{\baselineskip}
\setlength{\mylenone}{\mylenone + 1pt}
\setlength{\mylen}{(\textwidth - \mylenone)*\real{0.5}}
\begin{longtable}[l]{@{\hspace*{\mylen}}>{\setlength\parfillskip{0pt}}p{\mylenone}@{}@{}l@{}}
 & \\[-\the\mylentwo]
\centering{॥ १० \hspace*{-1.5mm}॥} & \\ \nopagebreak
पदपंकज बाँछौं सदा जिनके हरि उर नित बसैं & ॥\\
योगेश्वर श्रुतदेव अंग मुचु(कुंद) प्रियब्रत जेता & ।\\ \nopagebreak
पृथू परीक्षित शेष सूत शौनक परचेता & ॥\\
शतरूपा त्रय सुता सुनीति सति सबहि मँदालस & ।\\ \nopagebreak
जग्यपत्नि ब्रजनारि किये केशव अपने बस & ॥\\
ऐसे नर नारी जिते तिनही के गाऊँ जसैं & ।\\ \nopagebreak
पदपंकज बाँछौं सदा जिनके हरि उर नित बसैं & ॥
\end{longtable}
}
}
\fancyhead[LO,RE]{{\textmd{\Large १०: जिनके हरि नित उर बसैं}}}
\begin{sloppypar}\justifying\hyphenrules{nohyphenation}
\textbf{मूलार्थ}—मैं उनके चरण\-कमलोंको सेवाके लिये सदा इच्छाका विषय बनाता रहता हूँ अर्थात् उनके चरण\-कमलोंकी सेवा करनेके लिये सदैव इच्छा करता रहता हूँ जिनके हृदयमें \textbf{हरि} अर्थात् श्रीनाथ भगवान् निरन्तर निवास करते हैं। जैसे (१)~\textbf{नवयोगेश्वर}—कवि, हरि, अन्तरिक्ष, प्रबुद्ध, करभाजन, आविर्होत्र, द्रुमिल, चमस और पिप्पलायन (२)~\textbf{श्रुतिदेव}—मैथिलब्राह्मण जिन्होंने भगवान् श्रीकृष्णके पधारनेपर तन्मयतामें अपना सर्वस्व निछावर कर दिया था (३)~\textbf{अङ्ग}, जो वेनके अत्याचारसे घर छोड़कर परिव्राजक बन गए थे (४)~\textbf{मुचुकुन्द}, जिनको शयनमुद्रामें वर्तमान जानकर भगवान्‌ने जाकर स्वयं दर्शन दिया था, जब मुचुकुन्दकी क्रोधाग्निसे कालयवन जल गया था (५)~विजयी \textbf{प्रियव्रत}, जिनकी चर्चा मानसकारने स्वयं की है—
\end{sloppypar}

{\bfseries
\setlength{\mylenone}{0pt}
\setlength{\mylenthree}{0pt}
\settowidth{\mylentwo}{लघु सुत नाम प्रियब्रत ताही}
\setlength{\mylenone}{\maxof{\mylenone}{\mylentwo}}
\settowidth{\mylenfour}{बेद पुरान प्रशंसहिं जाही}
\setlength{\mylenthree}{\maxof{\mylenthree}{\mylenfour}}
\setlength{\mylentwo}{\baselineskip}
\setlength{\mylenone}{\mylenone + 1pt}
\setlength{\mylenfour}{\baselineskip}
\setlength{\mylenthree}{\mylenthree + 1pt}
\setlength{\mylen}{(\textwidth - \mylenone)}
\setlength{\mylen}{(\mylen - \mylenthree)*\real{0.5}}
\setlength{\mylen}{(\mylen - 4pt)}
\begin{longtable}[l]{@{\hspace*{\mylen}}>{\setlength\parfillskip{0pt}}p{\mylenone}@{}@{}l@{\hspace{6pt}}>{\setlength\parfillskip{0pt}}p{\mylenthree}@{}@{}l@{}}
 & & & \\[-\the\mylentwo]
लघु सुत नाम प्रियब्रत ताही & । & बेद पुरान प्रशंसहिं जाही & ॥\\ \nopagebreak
\caption*{(मा.~१.१४२.४)}
\end{longtable}
}

\begin{sloppypar}\justifying\hyphenrules{nohyphenation}
(६)~महाराज \textbf{पृथु} (७)~महाराज \textbf{परीक्षित्} (८)~पृथ्वीका भार वहन करनेवाले \textbf{शेषजी} (९)~पुराणके वक्ता और रोमहर्षणके पुत्र \textbf{सूतजी} (१०)~पुराणके प्रश्नकर्ता अट्ठासी हजार ऋषियोंके कुलपति \textbf{शौनकजी} (११)~प्राचीनबर्हि नामसे प्रसिद्ध बर्हिषद्‍के दस पुत्र \textbf{प्रचेतागण} (१२)~महारानी\textbf{ शतरूपा}—स्वायम्भुव मनुकी धर्मपत्नी (१३)~उनकी तीनों पुत्रियाँ—\textbf{आकूति}, \textbf{देवहूति} और \textbf{प्रसूति} (१४)~\textbf{सुनीति}—शतरूपाजीकी प्रथम पुत्रवधू, उत्तानपादकी धर्मपत्नी और ध्रुवकी माता (१५)~सभी \textbf{सतियाँ}—भूत, भविष्य और वर्तमानकी सभी पतिव्रताएँ (१६)~स्वयं \textbf{मदालसा}—ऋतध्वजकी पत्नी और विश्वावासु गन्धर्वकी पुत्री, जिन्होंने यह प्रतिज्ञा की थी कि उनके गर्भमें जो बालक आ जाएगा वह दुबारा गर्भमें नहीं आएगा (१७)~\textbf{यज्ञपत्नियाँ} और (१८)~\textbf{श्रीव्रजाङ्गनाएँ} जिन्होंने \textbf{केशव} अर्थात् भगवान् श्रीकृष्णको अपने वशमें कर लिया है। इनके चरण\-कमलकी सेवा मुझे अभीष्ट है। अर्थात् मैं इन सभी परिकरोंके चरणकमलोंकी सेवाके लिये सदैव इच्छा करता रहता हूँ। ऐसे जितने भी नर-नारी हैं, उनके यशको मैं सतत गाता रहूँ और उनके चरण\-कमलोंका निरन्तर मैं सेवाभिलाष धारण करूँ अर्थात् उनकी चरण\-रेणुकी प्राप्तिकी आशा मेरे लिये सदैव बनी रहे, जिनके हृदयमें श्रीहरि निरन्तर निवास करते हैं।
\end{sloppypar}
\begin{sloppypar}\justifying\hyphenrules{nohyphenation}
इस छप्पयमें नाभाजीने जिन महाभागवतोंकी चर्चा की है उनके हृदयमें प्रभु निरन्तर निवास करते ही हैं। यहाँ \textbf{सती} शब्द दक्षपुत्री और शङ्करपत्नी सतीके अर्थमें नहीं प्रयुक्त हुआ है, क्योंकि दक्षपुत्री सती भगवदीया नहीं थीं। वे तो भगवान्‌पर संशय करके अपने जीवनको संशयारूढ बना चुकी थीं। अतः उनमें भगवद्यशोगानकी पात्रता ही नहीं है। यहाँ सती शब्द सभी पतिव्रताओंका उपलक्षण है, न कि शङ्करपत्नी सतीका।
\end{sloppypar}
\begin{sloppypar}\justifying\hyphenrules{nohyphenation}
सभी पतिव्रताओंके साथ-साथ नाभाजी मदालसाका स्मरण करते हैं, जिन्होंने अपने प्रत्येक पुत्रको ऐसा दिव्य ज्ञान दिया जिससे वह फिर गर्भमें ही न आए। मदालसाका व्यक्तित्व बड़ा ही पावन है। \textbf{मदालसा} शब्दका अर्थ ही होता है \textbf{मदः अलसः यया सा मदालसा} अर्थात् जिनके कारण मद नीरस हो जाता है वे हैं मदालसा। स्वयं विश्वावसु गन्धर्वकी पुत्री और ऋतध्वज कुवलयाश्व महाराजकी धर्मपत्नी मदालसा अपने तीन-तीन पुत्रोंको—विक्रान्त, सुबाहु और शत्रुमर्दनको—बाल्यावस्थामें लोरी सुनाती हुईं कितना दिव्य उपदेश देती हैं। मदालसा विक्रान्तसे कहती हैं—
\end{sloppypar}

{\bfseries
\setlength{\mylenone}{0pt}
\settowidth{\mylentwo}{शुद्धोऽसि रे तात न तेऽस्ति नाम कृतं हि ते कल्पनयाऽधुनैव}
\setlength{\mylenone}{\maxof{\mylenone}{\mylentwo}}
\settowidth{\mylentwo}{पञ्चात्मकं देहमिदं तवैतन्नैवास्य त्वं रोदिषि कस्य हेतोः}
\setlength{\mylenone}{\maxof{\mylenone}{\mylentwo}}
\setlength{\mylentwo}{\baselineskip}
\setlength{\mylenone}{\mylenone + 1pt}
\setlength{\mylen}{(\textwidth - \mylenone)*\real{0.5}}
\begin{longtable}[l]{@{\hspace*{\mylen}}>{\setlength\parfillskip{0pt}}p{\mylenone}@{}@{}l@{}}
 & \\[-\the\mylentwo]
शुद्धोऽसि रे तात न तेऽस्ति नाम कृतं हि ते कल्पनयाऽधुनैव & ।\\ \nopagebreak
पञ्चात्मकं देहमिदं तवैतन्नैवास्य त्वं रोदिषि कस्य हेतोः & ॥\\ \nopagebreak
\caption*{(मा.पु.~२५.११)}
\end{longtable}
}

\begin{sloppypar}\justifying\hyphenrules{nohyphenation}
अर्थात् हे बालक! तुम शुद्ध हो, विशुद्ध जीवात्मा हो। तुम्हारा कोई नाम नहीं है। यह तो कल्पनासे अभी-अभी तुम्हारे भौतिक माता-पिता हमने यह नाम \textbf{विक्रान्त} रख दिया है। वास्तवमें तुम विक्रान्त नहीं हो। यह पञ्चात्मक शरीर भी तुम्हारा नहीं है, और तुम इसके नहीं हो। फिर किस कारणसे रो रहे हो?
\end{sloppypar}

{\bfseries
\setlength{\mylenone}{0pt}
\settowidth{\mylentwo}{शुद्धोऽसि बुद्धोऽसि निरञ्जनोऽसि संसारमायापरिवर्जितोऽसि}
\setlength{\mylenone}{\maxof{\mylenone}{\mylentwo}}
\settowidth{\mylentwo}{संसारनिद्रां त्यज स्वप्नरूपां मदालसा पुत्रमुवाच वाक्यम्}
\setlength{\mylenone}{\maxof{\mylenone}{\mylentwo}}
\setlength{\mylentwo}{\baselineskip}
\setlength{\mylenone}{\mylenone + 1pt}
\setlength{\mylen}{(\textwidth - \mylenone)*\real{0.5}}
\begin{longtable}[l]{@{\hspace*{\mylen}}>{\setlength\parfillskip{0pt}}p{\mylenone}@{}@{}l@{}}
 & \\[-\the\mylentwo]
शुद्धोऽसि बुद्धोऽसि निरञ्जनोऽसि संसारमायापरिवर्जितोऽसि & ।\\ \nopagebreak
संसारनिद्रां त्यज स्वप्नरूपां मदालसा पुत्रमुवाच वाक्यम् & ॥
\end{longtable}
}

\begin{sloppypar}\justifying\hyphenrules{nohyphenation}
तुम शुद्ध जीवात्मा हो, तुम बुद्ध हो अर्थात् सब कुछ जान गए हो, तुम निरञ्जन हो, और तुम संसारकी मायासे वर्जित अर्थात् अत्यन्त दूर हो। अतः बेटे! स्वप्नरूप संसारकी निद्राको छोड़ दो। इस प्रकार मदालसाने अपने पुत्रको संबोधित करके यह वाक्य कहा। विक्रान्त भगवत्परायण हो गए। यही परिस्थिति सुबाहुके साथ भी संपन्न हुई। यही घटना घटी। सुबाहुको भी मदालसाने यही लोरी सुनाई, सुबाहु भी भगवत्परायण हो गए। पुनः यही परिस्थिति शत्रुमर्दन नामक बालकके साथ भी आई। वहाँ भी मदालसाने यही लोरी सुनाई। शत्रुमर्दन भी भगवत्परायण विरक्त परिव्राजक बन गए। चतुर्थ बालक अलर्कने जब जन्म लिया, उस समय महाराजने मदालसासे विनती की कि मेरा वंश चलानेके लिये तो एक बालक चाहिये, इसको विरक्त मत बनाइये। मदालसाने महाराजकी बात मान ली, और अलर्कको प्रवृत्तिमार्गका उपदेश दिया। अन्ततोगत्वा मदालसाने एक पत्र लिखकर महाराज अलर्कके हाथमें विराजमान मुद्रिकाके भीतर छिपाकर रख दिया, और कहा—“जब संकट पड़े तब तुम यह पत्र पढ़ लेना।” वही हुआ। उस पत्रको पढ़कर अलर्क प्रवृत्तिको छोड़कर निवृत्तिके मार्गमें आ गए और धन्य-धन्य हो गए। धन्य हैं ये मदालसा, जिन्होंने अपने बेटोंको भगवत्परायण बना दिया!
\end{sloppypar}
\begin{sloppypar}\justifying\hyphenrules{nohyphenation}
\textbf{यज्ञपत्नियाँ}—भगवान् श्रीकृष्ण और ग्वालबालोंको जब भूख लगी तब यज्ञपत्नियोंसे भगवान्‌ने भोजनकी याचना कराई, और वे तुरन्त विविध प्रकारके व्यञ्जन बनाकर प्रभुके पास लेकर दौड़ पड़ीं। वहाँ भगवान्‌को निहारकर वे धन्य हो गईं। क्या ही सुन्दर दिव्य झाँकी दिखी—
\end{sloppypar}

{\bfseries
\setlength{\mylenone}{0pt}
\settowidth{\mylentwo}{श्यामं हिरण्यपरिधिं वनमाल्यबर्हधातुप्रवालनटवेषमनुव्रतांसे}
\setlength{\mylenone}{\maxof{\mylenone}{\mylentwo}}
\settowidth{\mylentwo}{विन्यस्तहस्तमितरेण धुनानमब्जं कर्णोत्पलालककपोलमुखाब्जहासम्}
\setlength{\mylenone}{\maxof{\mylenone}{\mylentwo}}
\setlength{\mylentwo}{\baselineskip}
\setlength{\mylenone}{\mylenone + 1pt}
\setlength{\mylen}{(\textwidth - \mylenone)*\real{0.5}}
\begin{longtable}[l]{@{\hspace*{\mylen}}>{\setlength\parfillskip{0pt}}p{\mylenone}@{}@{}l@{}}
 & \\[-\the\mylentwo]
श्यामं हिरण्यपरिधिं वनमाल्यबर्हधातुप्रवालनटवेषमनुव्रतांसे & ।\\ \nopagebreak
विन्यस्तहस्तमितरेण धुनानमब्जं कर्णोत्पलालककपोलमुखाब्जहासम् & ॥\\ \nopagebreak
\caption*{(भा.पु.~१०.२३.२२)}
\end{longtable}
}

\begin{sloppypar}\justifying\hyphenrules{nohyphenation}
क्या ही सुन्दर! कोटि-कोटि बालदिवाकरोंको भी विनिन्दित करनेवाले, दिव्य पीताम्बरको धारण किये हुए, वनमाला, मयूर\-मुकुट, धातु, प्रवाल आदि अलंकारोंसे युक्त, \textbf{अनुव्रतायाः अंसः अनुव्रतांसः तस्मिन् अनुव्रतांसे} अर्थात् अनुकूल व्रतका आचरण करनेवाली राधाजीके स्कन्धपर अपना वाम करकमल धारण किये हुए, और दक्षिण करकमलसे स्वयं एक कमलपुष्पको हिलाते हुए, और \textbf{कर्णोत्पलालक\-कपोल\-मुखाब्जहासम्} अर्थात् प्रभुके दिव्य कानोंमें उत्पल, उनका वह दिव्य अलक, मुखकमलपर मन्दहास—यह देखकर यज्ञपत्नियाँ धन्य हो गईं। उन्होंने प्रभुको प्रेमसे प्रसाद पवाया और प्रार्थना की—“प्रभु! हमें स्वीकार लीजिये।” प्रभुने कहा—“आप ब्राह्मण\-पत्नियाँ हैं। आप यज्ञमें पधारें! कोई भी कुछ भी नहीं बोलेगा। आपके पति भी आपको स्वीकारेंगे।”
\end{sloppypar}
\begin{sloppypar}\justifying\hyphenrules{nohyphenation}
\textbf{ब्रजनारी} अर्थात् वे धन्य व्रजबालाएँ जिनके लिये नाभाजीने कहा—\textbf{किये केशव अपने बस}—\textbf{कं ब्रह्माणमीशं शिवं च वशयति इति केशवः} अर्थात् जिन्होंने ब्रह्मा और शिवको भी वशमें कर लिया है ऐसे जगन्नियन्ता केशवको ही व्रजनारियोंने वशमें कर लिया। इसीलिये तो रसखानने कहा—
\end{sloppypar}

{\bfseries
\setlength{\mylenone}{0pt}
\settowidth{\mylentwo}{शेष महेश गणेश दिनेश सुरेशहु जाहि निरन्तर गावैं}
\setlength{\mylenone}{\maxof{\mylenone}{\mylentwo}}
\settowidth{\mylentwo}{जाहि अनादि अनंत अखंड अछेद अभेद सुबेद बतावैं}
\setlength{\mylenone}{\maxof{\mylenone}{\mylentwo}}
\settowidth{\mylentwo}{नारद से सुक ब्यास रटैं पचि हारें तऊ पुनि पार न पावैं}
\setlength{\mylenone}{\maxof{\mylenone}{\mylentwo}}
\settowidth{\mylentwo}{ताहि अहीर की छोहरियाँ छछिया भरि छाछ पै नाच नचावैं}
\setlength{\mylenone}{\maxof{\mylenone}{\mylentwo}}
\setlength{\mylentwo}{\baselineskip}
\setlength{\mylenone}{\mylenone + 1pt}
\setlength{\mylen}{(\textwidth - \mylenone)*\real{0.5}}
\begin{longtable}[l]{@{\hspace*{\mylen}}>{\setlength\parfillskip{0pt}}p{\mylenone}@{}@{}l@{}}
 & \\[-\the\mylentwo]
शेष महेश गणेश दिनेश सुरेशहु जाहि निरन्तर गावैं & ।\\ \nopagebreak
जाहि अनादि अनंत अखंड अछेद अभेद सुबेद बतावैं & ।\\
नारद से सुक ब्यास रटैं पचि हारें तऊ पुनि पार न पावैं & ।\\ \nopagebreak
ताहि अहीर की छोहरियाँ छछिया भरि छाछ पै नाच नचावैं & ॥
\end{longtable}
}

\begin{sloppypar}\justifying\hyphenrules{nohyphenation}
इस प्रकारके जितने भी नर-नारी हैं, मैं उनके यशको सतत गाना चाहता हूँ। जिनके हृदयमें श्रीहरि निरन्तर बसते हैं, मैं सेवा करनेके लिये उन्हीं परम\-भागवतोंके चरणकमलोंको प्राप्त करनेकी सदैव इच्छा करता हूँ। नाभाजी आगे कहते हैं—
\end{sloppypar}

\addcontentsline{toc}{section}{\texorpdfstring{पद ११: भक्तोंकी चरणधूलियाचना}{११: भक्तोंकी चरणधूलियाचना}}

{\relscale{1.1875}
{\bfseries
\setlength{\mylenone}{0pt}
\settowidth{\mylentwo}{}
\setlength{\mylenone}{\maxof{\mylenone}{\mylentwo}}
\settowidth{\mylentwo}{अंघ्री अम्बुज पांसु को जन्म जन्म हौं जाचिहौं}
\setlength{\mylenone}{\maxof{\mylenone}{\mylentwo}}
\settowidth{\mylentwo}{प्राचीनबर्हि सत्यब्रत रहूगण सगर भगीरथ}
\setlength{\mylenone}{\maxof{\mylenone}{\mylentwo}}
\settowidth{\mylentwo}{बाल्मीकि मिथिलेस गए जे जे गोबिँद पथ}
\setlength{\mylenone}{\maxof{\mylenone}{\mylentwo}}
\settowidth{\mylentwo}{रुक्मांगद हरिचंद भरत दधीचि उदारा}
\setlength{\mylenone}{\maxof{\mylenone}{\mylentwo}}
\settowidth{\mylentwo}{सुरथ सुधन्वा शिबिर सुमति अति बलिकी दारा}
\setlength{\mylenone}{\maxof{\mylenone}{\mylentwo}}
\settowidth{\mylentwo}{नील मोरध्वज ताम्रध्वज अलरक कीरति राचिहौं}
\setlength{\mylenone}{\maxof{\mylenone}{\mylentwo}}
\settowidth{\mylentwo}{अंघ्री अम्बुज पांसु को जन्म जन्म हौं जाचिहौं}
\setlength{\mylenone}{\maxof{\mylenone}{\mylentwo}}
\setlength{\mylentwo}{\baselineskip}
\setlength{\mylenone}{\mylenone + 1pt}
\setlength{\mylen}{(\textwidth - \mylenone)*\real{0.5}}
\begin{longtable}[l]{@{\hspace*{\mylen}}>{\setlength\parfillskip{0pt}}p{\mylenone}@{}@{}l@{}}
 & \\[-\the\mylentwo]
\centering{॥ ११ \hspace*{-1.5mm}॥} & \\ \nopagebreak
अंघ्री अम्बुज पांसु को जन्म जन्म हौं जाचिहौं & ॥\\
प्राचीनबर्हि सत्यब्रत रहूगण सगर भगीरथ & ।\\ \nopagebreak
बाल्मीकि मिथिलेस गए जे जे गोबिँद पथ & ॥\\
रुक्मांगद हरिचंद भरत दधीचि उदारा & ।\\ \nopagebreak
सुरथ सुधन्वा शिबिर सुमति अति बलिकी दारा & ॥\\
नील मोरध्वज ताम्रध्वज अलरक कीरति राचिहौं & ।\\ \nopagebreak
अंघ्री अम्बुज पांसु को जन्म जन्म हौं जाचिहौं & ॥
\end{longtable}
}
}
\fancyhead[LO,RE]{{\textmd{\Large ११: चरणधूलियाचना}}}
\begin{sloppypar}\justifying\hyphenrules{nohyphenation}
\textbf{मूलार्थ}—\textbf{अंघ्री} अर्थात् चरण, \textbf{अम्बुज} अर्थात् कमल, \textbf{पांसु} अर्थात् धूलि। मैं (१)~महाराज \textbf{प्राचीनबर्हि} (२)~महाराज \textbf{सत्यव्रत} (३)~महाराज \textbf{रहूगण} (४)~महाराज \textbf{सगर} (५)~महाराज \textbf{भगीरथ} (६)~महर्षि \textbf{वाल्मीकि} (७)~मिथिलेश अर्थात् महाराज \textbf{सीरध्वज जनक} और \textbf{बहुलाश्व}—इस प्रकार जो-जो परम भागवत भगवान् गोविन्दके पथका अनुसरण किये हैं अर्थात् जो-जो भगवत्पथपर आरूढ हुए हैं, ऐसे (८)~महाराज \textbf{रुक्माङ्गद} (९)~महाराज \textbf{हरिश्चन्द्र} (१०)~भक्तशिरोमणि दशरथ-कैकेयीके संकल्पसे प्रकट हुए \textbf{भैया भरत} (११)~उदार \textbf{दधीचि} (१२)~महाराज \textbf{सुरथ} (१३)~महाराज \textbf{सुधन्वा} (१४)~महाराज \textbf{शिबि} (१५)~अत्यन्त सुन्दर बुद्धिवाली, महाराज बलिकी पत्नी \textbf{श्रीविन्ध्यावलीजी} (१६)~\textbf{नील} अर्थात् \textbf{नीलध्वज} (१७)~\textbf{मोरध्वज} और \textbf{ताम्रध्वज} एवं (१८)~\textbf{अलर्क}की कीर्तिमें \textbf{राचिहौं} अर्थात् रँग जाऊँगा। और इन भागवतोंके चरणकमलकी धूलिको मैं \textbf{जन्म जन्म} अर्थात् अगणित जन्मों तक याचनाका विषय बनाता रहूँगा, अर्थात् इनकी चरणधूलिको मैं माँगता रहूँगा कि मुझे मिल जाए तो मैं धन्य हो जाऊँगा।
\end{sloppypar}
\begin{sloppypar}\justifying\hyphenrules{nohyphenation}
\textbf{प्राचीनबर्हि}—जो महाराज ध्रुवके वंशमें जन्मे, उनके मनमें कर्मकाण्डके प्रति बहुत निष्ठा थी। उन्हें नारदजीने \textbf{पुरञ्जनोपाख्यान} सुनाकर कर्मकाण्डके अधिक प्रयोगसे हटाकर भगवत्प्रेमी बना दिया।
\end{sloppypar}
\begin{sloppypar}\justifying\hyphenrules{nohyphenation}
\textbf{सत्यव्रत}—जो अभी इस मन्वन्तरके वैवस्वत मनु हैं तथा जिनके संकल्पसे भगवान्‌का मत्स्यावतार हुआ।
\end{sloppypar}
\begin{sloppypar}\justifying\hyphenrules{nohyphenation}
\textbf{रहूगण}—यही सौवीराधिपति पालकीपर चढ़कर महर्षि कपिलसे विद्या प्राप्त करने जा रहे थे। जब एक पालकी\-चालककी उच्छृङ्खलतासे वे क्षुब्ध हुए, तब पालकी चलानेवाले जडभरतने स्पष्ट कहा—“तुम मूर्ख होकर भी पण्डितों जैसी बात बोलते हो। विद्वान् लोग कभी भी इस व्यवहारको तत्त्वावबोधके साथ नहीं जोड़ते।” और उन्हीं रहूगणको जडभरतने यह समझाया—“रहूगण! यह अध्यात्म\-विद्या तपस्यासे नहीं प्राप्त हो सकती। यज्ञ या मुण्डन अथवा ग्रहोंसे नहीं प्राप्त होती, तथा सूर्य, अग्नि और जलकी उपासनासे नहीं प्राप्त होती। यह तो जब तक साधक महापुरुषोंके चरणकमलकी धूलिका प्रयोग करके अपने मनको शुद्ध नहीं करता, तब तक प्राप्त नहीं हो सकती। संपूर्ण व्यवहारोंकी जड़ है आचार्यको संतोष।”
\end{sloppypar}
\begin{sloppypar}\justifying\hyphenrules{nohyphenation}
महाराज \textbf{सगर}—ये अयोध्याके चक्रवर्ती महाराज थे। इन्होंने सौ अश्वमेध यज्ञ किये। सौवें यज्ञमें इन्द्रने विघ्न डाला और सगरके घोड़ेको चुराकर कपिलके आश्रममें बंद कर दिया। सगरके साठ हजार पुत्र ढूँढते-ढूँढते वहाँ आए और उन्होंने कपिलको दुर्वाक्य कहे। कपिलदेवकी क्रोधाग्निसे उनका शरीर भस्म हो गया। साठ हजार युवक राजपुत्रोंकी भस्मराशि देखकर स्वयं अंशुमान् क्षुब्ध हुए और कपिलदेवकी आज्ञासे उन्होंने, उनके पुत्र दिलीपने, तथा उनके पौत्र भगीरथने गङ्गाजीको लानेका यत्न किया। इस प्रकार सगर जैसे महापुरुषने भगवत्प्राप्ति करके सागरकी परम्पराको अक्षुण्ण और प्रामाणिक बनाया। भगीरथ इन्हीं महाराज सगरके प्रपौत्र थे। जब अंशुमान्‌को कपिलदेवने आज्ञा दी कि किसी प्रकार गङ्गा ले आएँगे तभी सगर\-पुत्रोंका उद्धार हो सकेगा, तब गङ्गाको लानेके लिये तपस्या करके अंशुमान्‌ने शरीर छोड़ा, महाराज दिलीपने शरीर छोड़ा, फिर भगीरथने तपस्या की। और भगीरथने यत्न करके गङ्गाजीको प्रसन्न कर लिया और शिवजीकी सहायतासे गङ्गाजीको भगीरथ धराधामपर ले आए, इसलिये उनका नाम \textbf{भागीरथी} पड़ गया। धन्य हो गया वह व्यक्तित्व जिसने वसुधामें सुधारसका संचरण किया।
\end{sloppypar}
\begin{sloppypar}\justifying\hyphenrules{nohyphenation}
महर्षि \textbf{वाल्मीकि}—ये यद्यपि भृगुवंशी ब्राह्मणकुलमें उत्पन्न हुए, जन्मना ये ब्राह्मण थे, परन्तु कुसंगतिके कारण किरातोंके संसर्गसे ये दूषितहो गए थे। इनका ब्रह्मत्व तिरोहित हो गया था। परमेश्वरकी कृपासे और सप्तर्षियोंके संकल्पसे वाल्मीकिके जीवनमें सुधार आया। इनका पूर्वका नाम \textbf{अग्निशर्मा} था। किसी-किसीके मतमें इनका नाम \textbf{रत्नाकर} भी बताया जाता है। सप्तर्षियोंने इन्हें \textbf{मरा मरा}का ही उपदेश दिया—\textbf{मरा मरा मरा चैव मरेति जप सर्वदा} (भ.पु.प्र.प.)। \textbf{मरा मरा} जपते-जपते इनके मुखसे \textbf{राम} निकल गया। इन्होंने रामनामका इतना जप किया कि इनके शरीरपर दीमककी माटी आ गई, जिसे संस्कृतमें \textbf{वल्मीक} कहते हैं। वल्मीकसे ढके होनेके कारण इनका नाम \textbf{वाल्मीकि} है। अनन्तर इन्होंने ही \textbf{वाल्मीकीय\-रामायण}का सृजन किया, जो विश्वका प्रथम काव्य और आदिकाव्य बना। इसमें मर्यादा\-पुरुषोत्तम भगवान् श्रीरामकी लोकमङ्गल\-कथा कहकर महर्षि वाल्मीकिने राष्ट्रकी व्यथा ही हर ली। इतना ही नहीं, उन्होंने श्रीरामचरितका वर्णन करनेके लिये सौ करोड़ रामायणें लिखीं—\textbf{चरितं रघुनाथस्य शतकोटिप्रविस्तरम्} (रा.र.स्तो.~१)।
\end{sloppypar}
\begin{sloppypar}\justifying\hyphenrules{nohyphenation}
\textbf{मिथिलेश}—सीरध्वज जनक। ये भी भगवान्‌के पथपर आरूढ़ हुए। इन्हें श्रीरामके प्रति गूढ प्रेम था—
\end{sloppypar}

{\bfseries
\setlength{\mylenone}{0pt}
\setlength{\mylenthree}{0pt}
\settowidth{\mylentwo}{प्रनवउँ परिजन सहित बिदेहू}
\setlength{\mylenone}{\maxof{\mylenone}{\mylentwo}}
\settowidth{\mylenfour}{जाहि राम पद गूढ़ सनेहु}
\setlength{\mylenthree}{\maxof{\mylenthree}{\mylenfour}}
\settowidth{\mylentwo}{जोग भोग महँ राखेउ गोई}
\setlength{\mylenone}{\maxof{\mylenone}{\mylentwo}}
\settowidth{\mylenfour}{राम बिलोकत प्रगटेउ सोई}
\setlength{\mylenthree}{\maxof{\mylenthree}{\mylenfour}}
\setlength{\mylentwo}{\baselineskip}
\setlength{\mylenone}{\mylenone + 1pt}
\setlength{\mylenfour}{\baselineskip}
\setlength{\mylenthree}{\mylenthree + 1pt}
\setlength{\mylen}{(\textwidth - \mylenone)}
\setlength{\mylen}{(\mylen - \mylenthree)*\real{0.5}}
\setlength{\mylen}{(\mylen - 4pt)}
\begin{longtable}[l]{@{\hspace*{\mylen}}>{\setlength\parfillskip{0pt}}p{\mylenone}@{}@{}l@{\hspace{6pt}}>{\setlength\parfillskip{0pt}}p{\mylenthree}@{}@{}l@{}}
 & & & \\[-\the\mylentwo]
प्रनवउँ परिजन सहित बिदेहू & । & जाहि राम पद गूढ़ सनेहु & ॥\\
जोग भोग महँ राखेउ गोई & । & राम बिलोकत प्रगटेउ सोई & ॥\\ \nopagebreak
\caption*{(मा.~१.१७.१-२)}
\end{longtable}
}

\begin{sloppypar}\justifying\hyphenrules{nohyphenation}
जनकजीका भगवान्‌के प्रति इतना अनन्य प्रेम था कि प्रथम दर्शनमें ही उन्होंने विश्वामित्रसे कह दिया कि श्रीरामको देखनेमें मेरा मन इतना अनुरक्त हो रहा है कि वह ब्रह्मसुखको हठात् छोड़ता जा रहा है—
\end{sloppypar}

{\bfseries
\setlength{\mylenone}{0pt}
\setlength{\mylenthree}{0pt}
\settowidth{\mylentwo}{इनहिं बिलोकत अति अनुरागा}
\setlength{\mylenone}{\maxof{\mylenone}{\mylentwo}}
\settowidth{\mylenfour}{बरबस ब्रह्मसुखहिं मन त्यागा}
\setlength{\mylenthree}{\maxof{\mylenthree}{\mylenfour}}
\setlength{\mylentwo}{\baselineskip}
\setlength{\mylenone}{\mylenone + 1pt}
\setlength{\mylenfour}{\baselineskip}
\setlength{\mylenthree}{\mylenthree + 1pt}
\setlength{\mylen}{(\textwidth - \mylenone)}
\setlength{\mylen}{(\mylen - \mylenthree)*\real{0.5}}
\setlength{\mylen}{(\mylen - 4pt)}
\begin{longtable}[l]{@{\hspace*{\mylen}}>{\setlength\parfillskip{0pt}}p{\mylenone}@{}@{}l@{\hspace{6pt}}>{\setlength\parfillskip{0pt}}p{\mylenthree}@{}@{}l@{}}
 & & & \\[-\the\mylentwo]
इनहिं बिलोकत अति अनुरागा & । & बरबस ब्रह्मसुखहिं मन त्यागा & ॥\\ \nopagebreak
\caption*{(मा.~१.२१६.५)}
\end{longtable}
}

\begin{sloppypar}\justifying\hyphenrules{nohyphenation}
ऐसे गोविन्दपथपर आरूढ भक्तके चरणकी धूलिकी याचना स्वाभाविक ही है।
\end{sloppypar}
\begin{sloppypar}\justifying\hyphenrules{nohyphenation}
\textbf{रुक्माङ्गद}—ये अयोध्याके महाराज थे। इनकी एकादशी\-व्रतपर बहुत निष्ठा थी, और कई बार भगवान्‌ने इनकी परीक्षा ली फिर भी ये डिगे नहीं। भगवान्‌ने इनकी परीक्षा लेनेके लिये एक मायाकी नारी इनके समक्ष भेज दी। उसका नाम ही था \textbf{मोहिनी}। महाराज उससे आकृष्ट हुए, उससे विवाह भी किया। उसने जब यह कहा था कि आपको मेरी प्रत्येक बात माननी पड़ेगी, उस समय महाराजने “हाँ” कह दिया था। परन्तु जब उस मोहिनीने कहा—“आपको एकादशी\-व्रत छोड़ना होगा,” तब रुक्माङ्गदने कहा—“तुम जाओ चाहे रहो, मैं एकादशी\-व्रत नहीं छोड़ता।” तब भगवान् ही प्रकट हो गए।
\end{sloppypar}
\begin{sloppypar}\justifying\hyphenrules{nohyphenation}
\textbf{हरिश्चन्द्र}—ये भी अयोध्याके महाराज थे। इनकी यशोगाथा सुनकर विश्वामित्रजीने इनकी परीक्षा लेनी चाही और इनका संपूर्ण राज्य ले लिया, यहाँ तक कि महाराज हरिश्चन्द्र काशीमें स्वयं पत्नीके सहित बिक गए और डोमकी सेवामें लगे, मृत्यु\-कर लेनेका कार्य करने लगे। रोहिताश्वको विश्वामित्रने सर्प होकर डस लिया। उसे लेकर हरिश्चन्द्रकी पत्नी शैव्या, जो ब्राह्मण\-दासी हो गईं थीं, श्मशानमें आईं। उनसे महाराजने कर माँगा। उनके पास कुछ नहीं था। वे जब अपनी साड़ी ही देने लगीं तब भगवान्‌ने हाथ पकड़ लिया।
\end{sloppypar}
\begin{sloppypar}\justifying\hyphenrules{nohyphenation}
चूँकि यह चर्चा और यह प्रसंग अयोध्याके राजाओंका है—रुक्माङ्गद अयोध्याके राजा, हरिश्चन्द्र अयोध्याके राजा, अतः उनके संसर्गसे \textbf{भरत} भी अयोध्याधिपतिके पुत्र ही यहाँ स्वीकार किये जाएँगे, न तो दुष्यन्त-शकुन्तला पुत्र भरत, और न ही ऋषभ-जयन्ती पुत्र भरत। भरत अर्थात् श्रीरामके छोटे भ्राता, जो परम भागवत हैं। वास्तवमें यदि यहाँ \textbf{भरत} शब्दसे दशरथनन्दन भरतका ग्रहण नहीं किया जाएगा तब तो भक्तमाल अधूरा ही रह जाएगा, क्योंकि नाभाजी सभी भक्तोंकी चर्चा करके भी यदि भरतजीकी चर्चा नहीं करेंगे तो यह ग्रन्थ अधूरा रहेगा। इसलिये भक्तमालके अध्येताओंसे मेरा विनम्र निवेदन है कि यहाँ \textbf{भरत} शब्दसे उन्हें न तो ऋषभ-जयन्ती पुत्र भरतका ग्रहण करना चाहिये, और न ही दुष्यन्त-शकुन्तला पुत्र भरतका ग्रहण करना चाहिये। वस्तुतः यहाँ भरतशब्दसे दशरथनन्दन, श्रीरामके छोटे भाई, भावते भैया भरतका ही ग्रहण करना चाहिये। श्रीभरतकी भक्तिके संबन्धमें हमारी \textbf{भरत महिमा} पुस्तक और हमारे \textbf{प्रभु करि कृपा पाँवरी दीन्ही}, \textbf{सब बिधि भरत सराहन जोगू} आदि प्रबन्धग्रन्थ पढ़ने चाहिये।
\end{sloppypar}
\begin{sloppypar}\justifying\hyphenrules{nohyphenation}
इसी प्रकार उदार \textbf{दधीचि}, जिन्होंने देवताओंके लिये अपना अस्थिदान कर दिया था।
\end{sloppypar}
\begin{sloppypar}\justifying\hyphenrules{nohyphenation}
\textbf{सुरथ} और \textbf{सुधन्वा}की कथा महाभारतके जैमिनीयाश्व\-मेधपर्वमें उपलब्ध होती है। जब युधिष्ठिरजीका अश्वमेध यज्ञ प्रारम्भ हुआ और उनके अश्वकी रक्षामें पृथानन्दन अर्जुन नियुक्त हुए, उस समय सुरथ और सुधन्वाके पिताने उन्हें युद्धमें भेजना चाहा। सुधन्वा एकनारीव्रत थे। अपनी माताके आदेशका पालन करते हुए वे पत्नीके द्वारा की जा रही आरती उतरवानेमें थोड़े-से विलम्बित हो गए। तब उन्हें शङ्ख और लिखित जैसे कुटिल मन्त्रियोंकी सम्मतिसे खौलते हुए तेलकी कढ़ाहीमें महाराजने फिंकवा दिया। परन्तु सुधन्वा यथावत् बचे रहे। उनको कोई हानि नहीं हुई। यह आश्चर्य देखकर शङ्ख और लिखितने एक नारियल कढ़ाहीमें फेंककर उनकी परीक्षा ली। नारियलके टुकड़े उन्हींके सिरपर जाकर टकरा गए। वही सुधन्वा, भगवान्‌के साथ उपस्थित हुए अर्जुनसे युद्ध करनेके लिये आए। घोर युद्ध किया। अर्जुनके हाथों सुधन्वाने वीरगति प्राप्त की और अर्जुनको पकड़े-पकड़े वे भगवान्‌के चरणपर गिर पड़े। उनका सिर भगवान्‌ने फेंक दिया, जिसे शिवजीने मुण्डमालामें लगा लिया। इसी प्रकार सुरथने भी अर्जुनसे घोर युद्ध किया और भगवान्‌का नाम लेकर जब अर्जुनने सुरथपर बाण चलाया तो सुरथको यह अनुमान लगाते विलम्ब न लगा कि प्रभु ही मुझे लेनेके लिये आए हैं। तुरन्त सुरथने दौड़कर अर्जुनको पकड़ लिया और अर्जुन द्वारा मारे जानेपर सुरथका सिर नीचे गिरा भगवान्‌के चरणोंमें। भगवान्‌ने वह सिर गरुडके द्वारा प्रयाग भिजवाया। उसे भी शिवजीने अपनी मुण्डमालामें लगा लिया।
\end{sloppypar}
\begin{sloppypar}\justifying\hyphenrules{nohyphenation}
महाराज \textbf{शिबि}का महाभारत के भिन्न-भिन्न पर्वोंमें वर्णन है। अग्नि और इन्द्रके द्वारा कबूतर और बाजके रूपमें ली गई महाराज शिबिकी परीक्षा तो सर्वविदित है ही, जिसका वर्णन महाभारतके वनपर्वमें है। शरणमें आए हुए कबूतरके प्राणोंकी रक्षाके लिये महाराज शिबिने बाजके द्वारा कबूतरके भारके समान उनका मांस माँगे जानेपर स्वयं अपने शरीरका मांस काट-काटकर तराजूपर तोला। कबूतरके उत्तरोत्तर भारी होनेपर जब महाराज शिबिके पास काटनेको मांस नहीं बचा, तो वे स्वयं तराजूपर चढ़ गए। तभी अग्नि और इन्द्र अपने मूलरूपमें प्रकट हुए और उन्होंने राजा शिबिको आशीर्वाद दिया। अनेक वर्षोंतक भगवद्भक्त राजा शिबिने धर्मानुसार पृथ्वीका पालन किया।
\end{sloppypar}
\begin{sloppypar}\justifying\hyphenrules{nohyphenation}
बलिकी पत्नी \textbf{विन्ध्यावली}, जिनको नाभाजीने \textbf{सुमति} कहा, ये सुन्दर बुद्धिवाली हैं। उनके पति अर्थात् बलिका भगवान्‌ने सब कुछ ले लिया, फिर भी उन्हें क्रोध नहीं आया। और उन्होंने प्रभुकी कृतज्ञताका बोध किया—“धन्य हैं प्रभु! मेरे पतिके अहंकारको आपने समाप्त कर दिया और मेरे पतिके सिरपर आपने चरण रख दिया, उन्हें अपना कृपा\-भाजन बना लिया। मैं भी एक वरदान आपसे माँगती हूँ कि आप पातालमें विराजें और प्रातःकाल मैं जिस द्वारपर निहारूँ उस द्वारपर आपके दर्शन हो जाएँ।” धन्य हैं वे विन्ध्यावलीजी!
\end{sloppypar}
\begin{sloppypar}\justifying\hyphenrules{nohyphenation}
\textbf{नीलध्वज}, \textbf{मोरध्वज}, \textbf{ताम्रध्वज} और \textbf{अलर्क}की कीर्तिमें मैं रच जाऊँ, उनकी कीर्तिमें मैं मग्न हो जाऊँ। \textbf{नीलध्वज} बड़े प्रतापी राजा थे। जब युधिष्ठिरका अश्वमेधीय अश्व महाराज नीलध्वजकी राजधानीमें आया तब अग्निदेवके कहनेपर नीलध्वजने अर्जुनसे युद्ध नहीं किया और उन्हींकी सहायतामें लग गए।
\end{sloppypar}
\begin{sloppypar}\justifying\hyphenrules{nohyphenation}
\textbf{मोरध्वज} अद्भुत प्रतापी राजा थे। उनकी परीक्षा लेनेके लिये भगवान्‌ने स्वयं एक ब्राह्मणका रूप बनाया, अर्जुनको बालक बनाया, यमराजको सिंह बनाया, और मोरध्वजसे कहा—“यदि तुम्हारा शरीर तुम्हारे बेटे द्वारा आरेसे चीरा जाए और वह प्रसन्नतासे यह विधि संपन्न करे और तुम्हारे भी मनमें किसी प्रकारकी ग्लानि न हो तो उसी मांसको मेरा सिंह खाएगा, तब मैं बालकके साथ भोजन कर लूँगा।” मोरध्वजने यह बात स्वीकार कर ली। उनके पुत्र \textbf{ताम्रध्वज}ने हँसते-हँसते आरा चलाना प्रारम्भ किया। मोरध्वज \textbf{जयगोविन्द}, \textbf{श्रीगोविन्द}, \textbf{हरिगोविन्द} जैसे दिव्य भगवन्नामोंका उच्चारण करते रहे। अन्ततोगत्वा वाम आँखमें थोड़ा-सा आँसू आ गया। तब भगवान्‌ने कहा—“अब तो मेरा सिंह भोजन नहीं करेगा।” तब मोरध्वजने टूटे स्वरमें कहा—“भगवन्! आप मेरे वाम अङ्गको नहीं स्वीकार रहे थे, इसकी निरर्थकतापर मेरे वाम नेत्रमें आँसू आ गए थे।” भगवान् प्रसन्न हो गए, और मोरध्वजको जीवित कर दिया। किन्हीं-किन्हींके मतमें मोरध्वजने अपने पुत्र ताम्रध्वजके ही शरीरको अपने हाथसे आरेसे चीरा था और भगवान्‌ने ताम्रध्वजको जीवित कर दिया था। इस कथाका संदर्भ सत्यनारायण\-व्रतकथामें वेदव्यासने इस प्रकार दिया है—
\end{sloppypar}

{\bfseries
\setlength{\mylenone}{0pt}
\settowidth{\mylentwo}{धार्मिकः सत्यसन्धश्च साधुर्मोरध्वजोऽभवत्‌}
\setlength{\mylenone}{\maxof{\mylenone}{\mylentwo}}
\settowidth{\mylentwo}{देहार्धं क्रकचैश्छित्त्वा दत्त्वा मोक्षमवाप ह}
\setlength{\mylenone}{\maxof{\mylenone}{\mylentwo}}
\setlength{\mylentwo}{\baselineskip}
\setlength{\mylenone}{\mylenone + 1pt}
\setlength{\mylen}{(\textwidth - \mylenone)*\real{0.5}}
\begin{longtable}[l]{@{\hspace*{\mylen}}>{\setlength\parfillskip{0pt}}p{\mylenone}@{}@{}l@{}}
 & \\[-\the\mylentwo]
धार्मिकः सत्यसन्धश्च साधुर्मोरध्वजोऽभवत्‌ & ।\\ \nopagebreak
देहार्धं क्रकचैश्छित्त्वा दत्त्वा मोक्षमवाप ह & ॥\\ \nopagebreak
\caption*{(स्क.पु.रे.ख.स.क.~५.२२)}
\end{longtable}
}

\begin{sloppypar}\justifying\hyphenrules{nohyphenation}
महाराज \textbf{अलर्क}, जिनकी माताकी चर्चा पूर्व छप्पयमें आ गई है, मदालसाके चतुर्थ पुत्र हैं। ऋतध्वजकी प्रार्थनापर मदालसाने इन्हें प्रवृत्ति\-मार्गमें लगा दिया था। स्वयं मदालसा जब वन जाने लगीं तब उन्होंने दो श्लोक लिखकर इनकी कलाईमें बाँध दिये था। जाते-जाते मदालसा यह कह कर गईं थीं कि जब संकटमें पड़ना तब मेरे इन दोनों श्लोकोंको पढ़ लेना। इधर सुबाहु आदि राजकुमारोंने अलर्कपर आक्रमण करवा दिया और महाराज संकटमें पड़ गए। मदालसाकी बात स्मरणमें आई और उन्होंने अपने हाथमें बँधे हुए श्लोकोंको खोलकर पढ़ा। मदालसाने दो अनुष्टुप् लिखे थे। प्रथम श्लोक था—
\end{sloppypar}

{\bfseries
\setlength{\mylenone}{0pt}
\settowidth{\mylentwo}{सङ्गः सर्वात्मना त्याज्यः स चेत्त्यक्तुं न शक्यते}
\setlength{\mylenone}{\maxof{\mylenone}{\mylentwo}}
\settowidth{\mylentwo}{स सद्भिः सह कर्तव्यः सतां सङ्गो हि भेषजम्}
\setlength{\mylenone}{\maxof{\mylenone}{\mylentwo}}
\setlength{\mylentwo}{\baselineskip}
\setlength{\mylenone}{\mylenone + 1pt}
\setlength{\mylen}{(\textwidth - \mylenone)*\real{0.5}}
\begin{longtable}[l]{@{\hspace*{\mylen}}>{\setlength\parfillskip{0pt}}p{\mylenone}@{}@{}l@{}}
 & \\[-\the\mylentwo]
सङ्गः सर्वात्मना त्याज्यः स चेत्त्यक्तुं न शक्यते & ।\\ \nopagebreak
स सद्भिः सह कर्तव्यः सतां सङ्गो हि भेषजम् & ॥\\ \nopagebreak
\caption*{(मा.पु.~३७.२३)}
\end{longtable}
}

\begin{sloppypar}\justifying\hyphenrules{nohyphenation}
अर्थात् कभी भी किसीसे आसक्ति अथवा लगाव नहीं रखना चाहिये। यदि वह न छूट सके तो वह लगाव संतोंके साथ करना चाहिये। संतोंका संग ही भवरोगका बहुत बड़ा भेषज है, दवा है, औषधि है। द्वितीय श्लोक था—
\end{sloppypar}

{\bfseries
\setlength{\mylenone}{0pt}
\settowidth{\mylentwo}{कामः सर्वात्मना हेयो हातुञ्चेच्छक्यते न सः}
\setlength{\mylenone}{\maxof{\mylenone}{\mylentwo}}
\settowidth{\mylentwo}{मुमुक्षां प्रति तत्कार्यं सैव तस्यापि भेषजम्}
\setlength{\mylenone}{\maxof{\mylenone}{\mylentwo}}
\setlength{\mylentwo}{\baselineskip}
\setlength{\mylenone}{\mylenone + 1pt}
\setlength{\mylen}{(\textwidth - \mylenone)*\real{0.5}}
\begin{longtable}[l]{@{\hspace*{\mylen}}>{\setlength\parfillskip{0pt}}p{\mylenone}@{}@{}l@{}}
 & \\[-\the\mylentwo]
कामः सर्वात्मना हेयो हातुञ्चेच्छक्यते न सः & ।\\ \nopagebreak
मुमुक्षां प्रति तत्कार्यं सैव तस्यापि भेषजम् & ॥\\ \nopagebreak
\caption*{(मा.पु.~३७.२४)}
\end{longtable}
}

\begin{sloppypar}\justifying\hyphenrules{nohyphenation}
कभी भी मनमें किसी प्रकारकी कामना नहीं करनी चाहिये। यदि कामनाका त्याग न हो सके तो उसको मुमुक्षाके प्रति करना चाहिये अर्थात् मोक्षकी कामना करनी चाहिये, क्योंकि वही कामना संसारके रोगोंका भेषज है।
\end{sloppypar}
\begin{sloppypar}\justifying\hyphenrules{nohyphenation}
इस प्रकार प्राचीनबर्हि, सत्यव्रत, रहूगण, सगर, भगीरथ, महर्षि वाल्मीकि, योगिराज सीरध्वज जनक, रुक्माङ्गद, हरिश्चन्द्र, दशरथनन्दन श्रीरामभक्त श्रीरामानुज भरत, उदार दधीचि, सुरथ, सुधन्वा, शिबि, अत्यन्त शुद्ध बुद्धिवाली बलिकी पत्नी विन्ध्यावली, नील अर्थात् नीलध्वज, मोरध्वज, ताम्रध्वज और अलर्ककी कीर्तिमें मैं सतत मग्न रहूँगा और इन्हींके चरण\-कमलकी धूलिकी मैं जन्म-जन्मान्तर पर्यन्त याचना करता रहूँगा।
\end{sloppypar}

\addcontentsline{toc}{section}{\texorpdfstring{पद १२: जे जे हरिमाया तरे}{१२: जे जे हरिमाया तरे}}

{\relscale{1.1875}
{\bfseries
\setlength{\mylenone}{0pt}
\settowidth{\mylentwo}{}
\setlength{\mylenone}{\maxof{\mylenone}{\mylentwo}}
\settowidth{\mylentwo}{तिन चरन धूरी मो भूरि सिर जे जे हरिमाया तरे}
\setlength{\mylenone}{\maxof{\mylenone}{\mylentwo}}
\settowidth{\mylentwo}{रिभु इक्ष्वाकु अरु ऐल गाधि रघु रै गै सुचि शतधन्वा}
\setlength{\mylenone}{\maxof{\mylenone}{\mylentwo}}
\settowidth{\mylentwo}{अमूरति अरु रन्ति उतंक भूरि देवल वैवस्वतमन्वा}
\setlength{\mylenone}{\maxof{\mylenone}{\mylentwo}}
\settowidth{\mylentwo}{नहुष जजाति दिलीप पुरु जदु गुह मान्धाता}
\setlength{\mylenone}{\maxof{\mylenone}{\mylentwo}}
\settowidth{\mylentwo}{पिप्पल निमि भरद्वाज दच्छ सरभंग सँघाता}
\setlength{\mylenone}{\maxof{\mylenone}{\mylentwo}}
\settowidth{\mylentwo}{संजय समीक उत्तानपाद जाग्यबल्क्य जस जग भरे}
\setlength{\mylenone}{\maxof{\mylenone}{\mylentwo}}
\settowidth{\mylentwo}{तिन चरन धूरी मो भूरि सिर जे जे हरिमाया तरे}
\setlength{\mylenone}{\maxof{\mylenone}{\mylentwo}}
\setlength{\mylentwo}{\baselineskip}
\setlength{\mylenone}{\mylenone + 1pt}
\setlength{\mylen}{(\textwidth - \mylenone)*\real{0.5}}
\begin{longtable}[l]{@{\hspace*{\mylen}}>{\setlength\parfillskip{0pt}}p{\mylenone}@{}@{}l@{}}
 & \\[-\the\mylentwo]
\centering{॥ १२ \hspace*{-1.5mm}॥} & \\ \nopagebreak
तिन चरन धूरी मो भूरि सिर जे जे हरिमाया तरे & ॥\\
रिभु इक्ष्वाकु अरु ऐल गाधि रघु रै गै सुचि शतधन्वा & ।\\ \nopagebreak
अमूरति अरु रन्ति उतंक भूरि देवल वैवस्वतमन्वा & ॥\\
नहुष जजाति दिलीप पुरु जदु गुह मान्धाता & ।\\ \nopagebreak
पिप्पल निमि भरद्वाज दच्छ सरभंग सँघाता & ॥\\
संजय समीक उत्तानपाद जाग्यबल्क्य जस जग भरे & ।\\ \nopagebreak
तिन चरन धूरी मो भूरि सिर जे जे हरिमाया तरे & ॥
\end{longtable}
}
}
\fancyhead[LO,RE]{{\textmd{\Large १२: जे जे हरिमाया तरे}}}
\begin{sloppypar}\justifying\hyphenrules{nohyphenation}
\textbf{मूलार्थ}—जो-जो भगवान्‌की मायानदीको पार कर चुके हैं, उन परम भागवतोंके चरणकी अनन्त धूलि मेरे सिरपर सतत विराजमान रहे। जैसे ऋभु, इक्ष्वाकु, ऐल, श्रीगाधि, रघु, रय, गय, पवित्र शतधन्वा, अमूर्ति, रन्तिदेव, उत्तङ्क, भूरिश्रवा, देवल, वैवस्वत मनु, श्रीनहुष, ययाति, दिलीप, पुरु, यदु, गुह, राजर्षि मान्धाता, महर्षि पिप्पलाद, निमि, भरद्वाज, दक्ष, शरभङ्ग आदि भगवत्परायण मुनिगण, सञ्जय, महर्षि शमीक, उत्तानपाद, याज्ञवल्क्य—ऐसे राजर्षि-महर्षियोंने अपने यशसे जगको भर दिया है। उन्हीं परमभागवतोंके चरणकी बहुत-सी धूलि मेरे सिरपर सदैव रहे।
\end{sloppypar}
\begin{sloppypar}\justifying\hyphenrules{nohyphenation}
यहाँ नाभाजीने जिन-जिन भागवतोंके नाम गिनाए हैं वे प्रायशः श्रीमद्भागवतजीमें वर्णित हैं। कुछ रामायणमें वर्णित हैं, और कुछ महाभारतमें। ये सब अपने भजनके प्रभावसे वैष्णवी\-माया\-नदीको पार कर चुके हैं। इनमें हैं—(१)~\textbf{श्रीऋभु} (२)~\textbf{इक्ष्वाकु} (३)~\textbf{ऐल} अर्थात् सुद्युम्न (४)~विश्वामित्रके पिता \textbf{गाधि} (५)~\textbf{रघु}, जिनके नामसे यह रघुवंश प्रसिद्ध हुआ, और इन्हीं रघुके पुत्र अज और इन्हीं अजके पुत्र दशरथ, और उनके पुत्र भगवान् राम। इनके संबन्धमें भागवतकार कितना सुन्दर श्लोक कहते हैं—
\end{sloppypar}

{\bfseries
\setlength{\mylenone}{0pt}
\settowidth{\mylentwo}{खट्वाङ्गाद्दीर्घबाहुश्च रघुस्तस्मात् पृथुश्रवाः}
\setlength{\mylenone}{\maxof{\mylenone}{\mylentwo}}
\settowidth{\mylentwo}{अजस्ततो महाराजस्तस्माद्दशरथोऽभवत्}
\setlength{\mylenone}{\maxof{\mylenone}{\mylentwo}}
\settowidth{\mylentwo}{तस्यापि भगवानेष साक्षाद्ब्रह्ममयो हरिः}
\setlength{\mylenone}{\maxof{\mylenone}{\mylentwo}}
\settowidth{\mylentwo}{अंशांशेन चतुर्धाऽगात्पुत्रत्वं प्रार्थितः सुरैः}
\setlength{\mylenone}{\maxof{\mylenone}{\mylentwo}}
\settowidth{\mylentwo}{रामलक्ष्मणभरतशत्रुघ्ना इति संज्ञया}
\setlength{\mylenone}{\maxof{\mylenone}{\mylentwo}}
\setlength{\mylentwo}{\baselineskip}
\setlength{\mylenone}{\mylenone + 1pt}
\setlength{\mylen}{(\textwidth - \mylenone)*\real{0.5}}
\begin{longtable}[l]{@{\hspace*{\mylen}}>{\setlength\parfillskip{0pt}}p{\mylenone}@{}@{}l@{}}
 & \\[-\the\mylentwo]
खट्वाङ्गाद्दीर्घबाहुश्च रघुस्तस्मात् पृथुश्रवाः & ।\\ \nopagebreak
अजस्ततो महाराजस्तस्माद्दशरथोऽभवत् & ॥\\
तस्यापि भगवानेष साक्षाद्ब्रह्ममयो हरिः & ।\\ \nopagebreak
अंशांशेन चतुर्धाऽगात्पुत्रत्वं प्रार्थितः सुरैः & ।\\ \nopagebreak
रामलक्ष्मणभरतशत्रुघ्ना इति संज्ञया & ॥\\ \nopagebreak
\caption*{(भा.पु.~९.१०.१-२)}
\end{longtable}
}

\begin{sloppypar}\justifying\hyphenrules{nohyphenation}
इसी प्रकार (६)~\textbf{रय} (७)~राजर्षि \textbf{गय} (८)~पवित्र \textbf{शतधन्वा}, जिनका वर्णन श्रीमद्भागवतके दशमस्कन्धके उत्तरार्धमें है, जो स्यमन्तक\-मणि कृतवर्मा और अक्रूरके पास रखकर भाग रहे थे—मिथिलाके उपवनमें भगवान् श्रीकृष्णके चक्रसे उनका वध हुआ और उन्हें भगवद्धामकी प्राप्ति हो गई (९)~\textbf{अमूर्ति} (१०)~\textbf{रन्तिदेव}, जिनकी कथा भागवतजीके नवम स्कन्धमें है, अयाचितवृत्तिका पालन करते हुए ४८ दिन तक जब उन्होंने कुछ नहीं लिया और ४९वें दिन कुछ मिला तो कभी ब्राह्मण, कभी चाण्डाल, कभी कुत्ता, और अन्ततोगत्वा एक भूखे पुल्कसको सब कुछ दे डाला तब भगवान् प्रकट हो गए। और भगवान्‌के “वरदान माँगो,” यह कहनेपर उन्होंने कह दिया—“मैं यही वरदान माँगता हूँ कि किसीको अब कष्ट न हो।” ऐसे रन्तिदेव जिनके संबन्धमें गोस्वामीजीने कहा—
\end{sloppypar}

{\bfseries
\setlength{\mylenone}{0pt}
\setlength{\mylenthree}{0pt}
\settowidth{\mylentwo}{रंतिदेव बलि भूप सुजाना}
\setlength{\mylenone}{\maxof{\mylenone}{\mylentwo}}
\settowidth{\mylenfour}{धरम धरेउ सहि संकट नाना}
\setlength{\mylenthree}{\maxof{\mylenthree}{\mylenfour}}
\setlength{\mylentwo}{\baselineskip}
\setlength{\mylenone}{\mylenone + 1pt}
\setlength{\mylenfour}{\baselineskip}
\setlength{\mylenthree}{\mylenthree + 1pt}
\setlength{\mylen}{(\textwidth - \mylenone)}
\setlength{\mylen}{(\mylen - \mylenthree)*\real{0.5}}
\setlength{\mylen}{(\mylen - 4pt)}
\begin{longtable}[l]{@{\hspace*{\mylen}}>{\setlength\parfillskip{0pt}}p{\mylenone}@{}@{}l@{\hspace{6pt}}>{\setlength\parfillskip{0pt}}p{\mylenthree}@{}@{}l@{}}
 & & & \\[-\the\mylentwo]
रंतिदेव बलि भूप सुजाना & । & धरम धरेउ सहि संकट नाना & ॥\\ \nopagebreak
\caption*{(मा.~२.९५.४)}
\end{longtable}
}

\begin{sloppypar}\justifying\hyphenrules{nohyphenation}
(११)~\textbf{उत्तङ्क} (१२)~\textbf{भूरिश्रवा}, जो दुर्योधनके पितृव्य लगते थे, और महाभारतके युद्धमें सात्यकिसे युद्ध करते समय अर्जुनने जिनकी भुजा काट दी थी और फिर पृथ्वीपर बैठकर सात्यकिसे वार्तालाप करते हुए उन्हींकी तलवारसे वे वीरगतिको प्राप्त हो गए (१३)~महर्षि \textbf{देवल} (१४)~\textbf{वैवस्वत मनु} जिनके संबन्धमें रघुवंश\-महाकाव्यके प्रथम सर्गके ११वें श्लोकमें कालिदास कहते हैं—
\end{sloppypar}

{\bfseries
\setlength{\mylenone}{0pt}
\settowidth{\mylentwo}{वैवस्वतो मनुर्नाम माननीयो मनीषिणाम्}
\setlength{\mylenone}{\maxof{\mylenone}{\mylentwo}}
\settowidth{\mylentwo}{आसीन्महीक्षितामाद्यः प्रणवश्छन्दसामिव}
\setlength{\mylenone}{\maxof{\mylenone}{\mylentwo}}
\setlength{\mylentwo}{\baselineskip}
\setlength{\mylenone}{\mylenone + 1pt}
\setlength{\mylen}{(\textwidth - \mylenone)*\real{0.5}}
\begin{longtable}[l]{@{\hspace*{\mylen}}>{\setlength\parfillskip{0pt}}p{\mylenone}@{}@{}l@{}}
 & \\[-\the\mylentwo]
वैवस्वतो मनुर्नाम माननीयो मनीषिणाम् & ।\\ \nopagebreak
आसीन्महीक्षितामाद्यः प्रणवश्छन्दसामिव & ॥\\ \nopagebreak
\caption*{(र.वं.~१.११)}
\end{longtable}
}

\begin{sloppypar}\justifying\hyphenrules{nohyphenation}
अर्थात् राजाओंमें वैवस्वत मनु उसी प्रकार माननीय हुए जैसे वेदोंमें ॐकार माननीय है। हम सब जिस मन्वन्तरमें रह रहे हैं, उस मन्वन्तरके अधिपति यही वैवस्वत मनु हैं, जिन्हें श्राद्धदेव भी कहते हैं। (१५)~\textbf{नहुष}, जो ब्राह्मणोंके शापसे गिरगिट बने और फिर भगवान् कृष्णका स्पर्श पाकर जिनका उद्धार हो गया (१६)~\textbf{ययाति}, जो यदु और पुरुके पिता थे, वे भी दृढ़ वैराग्य प्राप्त करके परमगतिको प्राप्त हुए (१७)~\textbf{दिलीप}—एक तो भगीरथके पिता दिलीप और दूसरे रघुजीके पिता श्रीदिलीप जिन्होंने निन्यानवे यज्ञ पूर्ण कर लिये थे और सौवें अश्वमेध यज्ञमें इन्द्रने उनका घोड़ा पकड़ा था। और रघुसे तुमुल युद्ध होनेके पश्चात् अन्तमें जब इन्द्र रघुसे संतुष्ट हुए तब रघुने यही कहा—“आप घोड़ा ले जाएँ, पर सौवें अश्वमेध यज्ञका फल मेरे पिताजीको मिल जाना चाहिये।” और ऐसा ही हुआ, और वे परमपदको प्राप्त हुए। (१८)~\textbf{पुरु}—इन्होंने ययातिको अपनी युवावस्था दी थी, और पिताकी आज्ञाका पालन करनेके कारण ये भी भगवान्‌की मायाको पार कर गए, और इन्हें परम पदकी प्राप्ति हुई (१९)~\textbf{यदु}—साक्षात् भगवान् श्रीकृष्णचन्द्रजीके वंशप्रवर्तक—इन्होंने धर्मकी सूक्ष्मताका विचार करके पिताके माँगनेपर भी उन्हें अपना यौवन नहीं दिया, क्योंकि उन्हें यह लगा कि इस यौवनसे पिता माताका उपभोग करेंगे और मुझपर मातृभोगीणताका पाप लग जाएगा, इसीलिये तो इनके वंशमें भगवान् श्रीकृष्णचन्द्रका प्रादुर्भाव हुआ (२०)~\textbf{गुह}—इनकी कथा श्रीरामायणमें प्रसिद्ध है। ये भगवान्‌के अन्तरङ्ग सखा हैं। इनके लिये वाल्मीकि कहते हैं—\textbf{गुहमासाद्य धर्मात्मा निषादाधिपतिं प्रियम्} (वा.रा.~१.१.२९)। और इनके संबन्धमें संतोंके मुखसे कथा सुनी जाती है कि भगवान् श्रीरामके वनवास चले जानेपर गुह सतत रोते रहते थे। और उन्होंने इतना रोया कि इनके नेत्रसे पहले तो आँसू गिरे और फिर रक्त गिरने लगा। धीरे-धीरे इनके नेत्रकी दृष्टि चली गई। और जब भगवान् श्रीराम वनवाससे प्रत्यावृत्त हुए अर्थात् लौटे तब सबने इन्हें समाचार दिया कि प्रभु श्रीराम आ गए हैं। \textbf{सुनत गुहउ धायउ प्रेमाकुल} (मा.~७.१२१.१०)—ये सुनकरके दौड़े अर्थात् दिखता नहीं था इन्हें। पर \textbf{आयउ निकट परम सुख संकुल} (मा.~७.१२१.१०) , और फिर \textbf{प्रभुहिं सहित बिलोकि बैदेही} (मा.~७.१२१.११)—जब भगवान् श्रीरामके पास ये पहुँचे तब इन्हें फिर दृष्टि मिल गई, और इन्होंने सीतारामजीके दर्शन किये। इन्हींके संबन्धमें भगवान् रामने उत्तरकाण्डमें यह कहा—
\end{sloppypar}

{\bfseries
\setlength{\mylenone}{0pt}
\setlength{\mylenthree}{0pt}
\settowidth{\mylentwo}{तुम मम सखा भरत सम भ्राता}
\setlength{\mylenone}{\maxof{\mylenone}{\mylentwo}}
\settowidth{\mylenfour}{सदा रहेहु पुर आवत जाता}
\setlength{\mylenthree}{\maxof{\mylenthree}{\mylenfour}}
\setlength{\mylentwo}{\baselineskip}
\setlength{\mylenone}{\mylenone + 1pt}
\setlength{\mylenfour}{\baselineskip}
\setlength{\mylenthree}{\mylenthree + 1pt}
\setlength{\mylen}{(\textwidth - \mylenone)}
\setlength{\mylen}{(\mylen - \mylenthree)*\real{0.5}}
\setlength{\mylen}{(\mylen - 4pt)}
\begin{longtable}[l]{@{\hspace*{\mylen}}>{\setlength\parfillskip{0pt}}p{\mylenone}@{}@{}l@{\hspace{6pt}}>{\setlength\parfillskip{0pt}}p{\mylenthree}@{}@{}l@{}}
 & & & \\[-\the\mylentwo]
तुम मम सखा भरत सम भ्राता & । & सदा रहेहु पुर आवत जाता & ॥\\ \nopagebreak
\caption*{(मा.~७.२०.३)}
\end{longtable}
}

\begin{sloppypar}\justifying\hyphenrules{nohyphenation}
(२१)~\textbf{मान्धाता}—ये तो चक्रवर्ती नरेन्द्र थे ही। कहा यह जाता है कि जहाँ तक सूर्यनारायणकी रश्मियाँ जाती थीं वहाँ तककी भूमि मान्धाताकी थी। इन्हीं मान्धातासे पचास कन्याएँ प्राप्त की थीं महर्षि सौभरिने। इन्हीं मान्धाताके पुत्र थे मुचुकुन्द। (२२)~\textbf{पिप्पलाद}—ये उच्च कोटिके महर्षि थे (२३)~\textbf{निमि}—जनकवंशके प्रवर्तक (२४)~\textbf{भरद्वाज}—सप्तर्षियोंमें एक, ये महर्षि वाल्मीकिके शिष्य भी थे। इन्होंने ही याज्ञवल्क्यजीसे भगवान् श्रीरामके आध्यात्मिक पक्षकी चर्चा की, और इन्हींके प्रश्नके आधारपर याज्ञवल्क्यजीने कर्मघाटके आधारपर श्रीरामकथा इन्हें सुनाई। (२५)~\textbf{दक्ष}—पुराणमें दक्ष दो हैं। प्रथम सतीजीके पिता दक्ष, जिनका वध शिवजीने किया था। वे अभिप्रेत नहीं हैं। वे भगवान्‌की मायाको नहीं तरे। यहाँ द्वितीय दक्षकी चर्चा है। इन्हीं दक्षने फिर जाकर प्रचेताओंके यहाँ जन्म लिया और इन्होंने भगवान्‌की तपस्या करके उनसे प्रजावृद्धिका वरदान पाया। इन्होंने दो बार दस-दस लाख पुत्रोंको जन्म दिया, जिन्हें नारदजीने परिव्राजक बनाया। फिर नारदजीको इन्होंने यह शाप दिया—“तुम चौबीस मिनटसे अधिक कहीं नहीं रह सकते।” अनन्तर इन्होंने साठ कन्याओंको जन्म दिया, जिनसे संपूर्ण सृष्टि भरी-पूरी हो गई। इन्हीं दक्षकी यहाँ चर्चा की जा रही है। (२६)~\textbf{सरभंग सँघाता}—शरभङ्ग रामायणके प्रसिद्ध ऋषि हैं। इन्होंने भगवान् रामसे कहा—“प्रभु! जब मैं ब्रह्मलोक जा रहा था, उसी समय मैंने वनमें आपके आनेकी बात सुनी। मैं ब्रह्माजीके सिंहासनसे कूद पड़ा और अपनी कुटियामें आ गया। तबसे आपकी प्रतीक्षा कर रहा हूँ। आज आपके दर्शनसे मेरा हृदय शीतल हो गया,” यथा—
\end{sloppypar}

{\bfseries
\setlength{\mylenone}{0pt}
\setlength{\mylenthree}{0pt}
\settowidth{\mylentwo}{जात रहेउँ बिरंचि के धामा}
\setlength{\mylenone}{\maxof{\mylenone}{\mylentwo}}
\settowidth{\mylenfour}{सुनेउँ स्रवन बन ऐहैं रामा}
\setlength{\mylenthree}{\maxof{\mylenthree}{\mylenfour}}
\settowidth{\mylentwo}{चितवत पंथ रहेउँ दिन राती}
\setlength{\mylenone}{\maxof{\mylenone}{\mylentwo}}
\settowidth{\mylenfour}{अब प्रभु देखि जुड़ानी छाती}
\setlength{\mylenthree}{\maxof{\mylenthree}{\mylenfour}}
\setlength{\mylentwo}{\baselineskip}
\setlength{\mylenone}{\mylenone + 1pt}
\setlength{\mylenfour}{\baselineskip}
\setlength{\mylenthree}{\mylenthree + 1pt}
\setlength{\mylen}{(\textwidth - \mylenone)}
\setlength{\mylen}{(\mylen - \mylenthree)*\real{0.5}}
\setlength{\mylen}{(\mylen - 4pt)}
\begin{longtable}[l]{@{\hspace*{\mylen}}>{\setlength\parfillskip{0pt}}p{\mylenone}@{}@{}l@{\hspace{6pt}}>{\setlength\parfillskip{0pt}}p{\mylenthree}@{}@{}l@{}}
 & & & \\[-\the\mylentwo]
जात रहेउँ बिरंचि के धामा & । & सुनेउँ स्रवन बन ऐहैं रामा & ॥\\
चितवत पंथ रहेउँ दिन राती & । & अब प्रभु देखि जुड़ानी छाती & ॥\\ \nopagebreak
\caption*{(मा.~३.८.२-३)}
\end{longtable}
}

\begin{sloppypar}\justifying\hyphenrules{nohyphenation}
\textbf{सँघाता}का तात्पर्य यह है—फिर इनके संपर्कमें आनेवाले अनेक मुनिगण जो शरभङ्गके परलोक जाते समय श्रीरामजीके साक्षी बने, जिनके लिये गोस्वामी तुलसीदासजीने कहा—
\end{sloppypar}

{\bfseries
\setlength{\mylenone}{0pt}
\setlength{\mylenthree}{0pt}
\settowidth{\mylentwo}{ऋषिनिकाय मुनिवर गति देखी}
\setlength{\mylenone}{\maxof{\mylenone}{\mylentwo}}
\settowidth{\mylenfour}{सुखी भए निज हृदय बिशेषी}
\setlength{\mylenthree}{\maxof{\mylenthree}{\mylenfour}}
\settowidth{\mylentwo}{अस्तुति करहिं सकल मुनिबृंदा}
\setlength{\mylenone}{\maxof{\mylenone}{\mylentwo}}
\settowidth{\mylenfour}{जयति प्रनतहित करुनाकंदा}
\setlength{\mylenthree}{\maxof{\mylenthree}{\mylenfour}}
\settowidth{\mylentwo}{पुनि रघुनाथ चले बन आगे}
\setlength{\mylenone}{\maxof{\mylenone}{\mylentwo}}
\settowidth{\mylenfour}{मुनिवरबृंद बिपुल सँग लागे}
\setlength{\mylenthree}{\maxof{\mylenthree}{\mylenfour}}
\setlength{\mylentwo}{\baselineskip}
\setlength{\mylenone}{\mylenone + 1pt}
\setlength{\mylenfour}{\baselineskip}
\setlength{\mylenthree}{\mylenthree + 1pt}
\setlength{\mylen}{(\textwidth - \mylenone)}
\setlength{\mylen}{(\mylen - \mylenthree)*\real{0.5}}
\setlength{\mylen}{(\mylen - 4pt)}
\begin{longtable}[l]{@{\hspace*{\mylen}}>{\setlength\parfillskip{0pt}}p{\mylenone}@{}@{}l@{\hspace{6pt}}>{\setlength\parfillskip{0pt}}p{\mylenthree}@{}@{}l@{}}
 & & & \\[-\the\mylentwo]
ऋषिनिकाय मुनिवर गति देखी & । & सुखी भए निज हृदय बिशेषी & ॥\\
अस्तुति करहिं सकल मुनिबृंदा & । & जयति प्रनतहित करुनाकंदा & ॥\\
पुनि रघुनाथ चले बन आगे & । & मुनिवरबृंद बिपुल सँग लागे & ॥\\ \nopagebreak
\caption*{(मा.~३.९.३-५)}
\end{longtable}
}

\begin{sloppypar}\justifying\hyphenrules{nohyphenation}
इन्हींमें सुतीक्ष्णजी आदि दण्डकवनके सभी ऋषिगण हैं। (२७)~\textbf{सञ्जय}, जो व्यासजीके प्रसादसे दिव्यदृष्टि पाकर गीताशास्त्रके श्रोता और द्रष्टा बने। गीतामें \textbf{सञ्जय उवाच} प्रसिद्ध ही है। सञ्जयका अन्तिम निर्णय बहुत ही रोचक और बहुत ही सिद्धान्त\-संगत है—
\end{sloppypar}

{\bfseries
\setlength{\mylenone}{0pt}
\settowidth{\mylentwo}{यत्र योगेश्वरो कृष्णः यत्र पार्थो धनुर्धरः}
\setlength{\mylenone}{\maxof{\mylenone}{\mylentwo}}
\settowidth{\mylentwo}{तत्र श्रीर्विजयो भूतिर्ध्रुवा नीतिर्मतिर्मम}
\setlength{\mylenone}{\maxof{\mylenone}{\mylentwo}}
\setlength{\mylentwo}{\baselineskip}
\setlength{\mylenone}{\mylenone + 1pt}
\setlength{\mylen}{(\textwidth - \mylenone)*\real{0.5}}
\begin{longtable}[l]{@{\hspace*{\mylen}}>{\setlength\parfillskip{0pt}}p{\mylenone}@{}@{}l@{}}
 & \\[-\the\mylentwo]
यत्र योगेश्वरो कृष्णः यत्र पार्थो धनुर्धरः & ।\\ \nopagebreak
तत्र श्रीर्विजयो भूतिर्ध्रुवा नीतिर्मतिर्मम & ॥\\ \nopagebreak
\caption*{(भ.गी.~१८.७८)}
\end{longtable}
}

\begin{sloppypar}\justifying\hyphenrules{nohyphenation}
(२८)~\textbf{शमीक}—इनके गलेमें महाराज परीक्षित्‌ने मृत सर्प डाल दिया फिर भी इन्हें क्रोध नहीं आया। उलटे पुत्रके द्वारा परीक्षित्‌को शाप देनेकी बात सुनकर शमीक बहुत दुःखी हुए, और उन्होंने भगवान्‌से क्षमा माँगते हुए कहा—“मेरे पुत्रने जो अनुचित किया, प्रभु! आप क्षमा कर दें।” (२९)~\textbf{उत्तानपाद}—परमभागवत ध्रुवके पिताश्री। पहले तो ध्रुवका इन्होंने अपमान किया परन्तु जब ध्रुव भगवान्‌से वरदान प्राप्त करके आए तो इन्होंने ध्रुवको हृदयसे लगा लिया, और ये ध्रुवको राज्य सौंपकर वनको चले गए। (३०)~\textbf{याज्ञवल्क्य}—इनकी भगवद्भक्तिकी कहाँ तक बात कही जाए? इन्होंने महर्षि भरद्वाजको श्रीराम\-कथा सुनाई और यही जनकजीके पुरोहित बने। इन्होंने जनक-सुनयनाजीको संपूर्ण श्रीराम\-कथा सुनाई थी। सुनयनाजीने इसका उद्धरण कौशल्याजीको दिया—
\end{sloppypar}

{\bfseries
\setlength{\mylenone}{0pt}
\setlength{\mylenthree}{0pt}
\settowidth{\mylentwo}{राम जाइ बन करि सुरकाजू}
\setlength{\mylenone}{\maxof{\mylenone}{\mylentwo}}
\settowidth{\mylenfour}{अचल अवधपुर करिहैं राजू}
\setlength{\mylenthree}{\maxof{\mylenthree}{\mylenfour}}
\settowidth{\mylentwo}{अमर नाग नर राम बाहुबल}
\setlength{\mylenone}{\maxof{\mylenone}{\mylentwo}}
\settowidth{\mylenfour}{सुख बसिहैं अपने अपने थल}
\setlength{\mylenthree}{\maxof{\mylenthree}{\mylenfour}}
\settowidth{\mylentwo}{यह सब जाग्यबल्क्य कहि राखा}
\setlength{\mylenone}{\maxof{\mylenone}{\mylentwo}}
\settowidth{\mylenfour}{देबि न होइ मुधा मुनि भाखा}
\setlength{\mylenthree}{\maxof{\mylenthree}{\mylenfour}}
\setlength{\mylentwo}{\baselineskip}
\setlength{\mylenone}{\mylenone + 1pt}
\setlength{\mylenfour}{\baselineskip}
\setlength{\mylenthree}{\mylenthree + 1pt}
\setlength{\mylen}{(\textwidth - \mylenone)}
\setlength{\mylen}{(\mylen - \mylenthree)*\real{0.5}}
\setlength{\mylen}{(\mylen - 4pt)}
\begin{longtable}[l]{@{\hspace*{\mylen}}>{\setlength\parfillskip{0pt}}p{\mylenone}@{}@{}l@{\hspace{6pt}}>{\setlength\parfillskip{0pt}}p{\mylenthree}@{}@{}l@{}}
 & & & \\[-\the\mylentwo]
राम जाइ बन करि सुरकाजू & । & अचल अवधपुर करिहैं राजू & ॥\\
अमर नाग नर राम बाहुबल & । & सुख बसिहैं अपने अपने थल & ॥\\
यह सब जाग्यबल्क्य कहि राखा & । & देबि न होइ मुधा मुनि भाखा & ॥\\ \nopagebreak
\caption*{(मा.~२.२८५.६-८)}
\end{longtable}
}

\begin{sloppypar}\justifying\hyphenrules{nohyphenation}
इन सबके यशसे संसार भर गया है। ऐसे श्रीहरिमायाको तरनेवाले भक्तोंके लिये स्पष्ट कह दिया नाभाजीने कि मेरे सिरपर इनके चरणकी धूलिकी राशि सतत विराजमान रहे।
\end{sloppypar}
\begin{sloppypar}\justifying\hyphenrules{nohyphenation}
अब नाभाजी निमि और नौ योगेश्वरोंके चरणत्राण अर्थात् पादुकाकी शरणागति चाह रहे हैं। क्योंकि योगेश्वर ब्राह्मण हैं, वे पनही तो धारण कर नहीं सकते, वे तो पादुका ही धारण करेंगे। और निमि भी पादुका ही धारण करेंगे। इसलिये उचित है कि यहाँ \textbf{पादत्रान}का अर्थ पादुका ही किया जाए।
\end{sloppypar}

\addcontentsline{toc}{section}{\texorpdfstring{पद १३: नौ योगेश्वर}{१३: नौ योगेश्वर}}

{\relscale{1.1875}
{\bfseries
\setlength{\mylenone}{0pt}
\settowidth{\mylentwo}{}
\setlength{\mylenone}{\maxof{\mylenone}{\mylentwo}}
\settowidth{\mylentwo}{निमि अरु नव योगेश्वरा पादत्रान की हौं सरन}
\setlength{\mylenone}{\maxof{\mylenone}{\mylentwo}}
\settowidth{\mylentwo}{कबि हरि करभाजन भक्तिरत्नाकर भारी}
\setlength{\mylenone}{\maxof{\mylenone}{\mylentwo}}
\settowidth{\mylentwo}{अन्तरिच्छ अरु चमस अनन्यता पधति उधारी}
\setlength{\mylenone}{\maxof{\mylenone}{\mylentwo}}
\settowidth{\mylentwo}{प्रबुध प्रेम की रासि भूरिदा आबिरहोता}
\setlength{\mylenone}{\maxof{\mylenone}{\mylentwo}}
\settowidth{\mylentwo}{पिप्पल द्रुमिल प्रसिद्ध भवाब्धि पार के पोता}
\setlength{\mylenone}{\maxof{\mylenone}{\mylentwo}}
\settowidth{\mylentwo}{जयंतीनंदन जगत के त्रिबिध ताप आमयहरन}
\setlength{\mylenone}{\maxof{\mylenone}{\mylentwo}}
\settowidth{\mylentwo}{निमि अरु नव योगेश्वरा पादत्रान की हौं सरन}
\setlength{\mylenone}{\maxof{\mylenone}{\mylentwo}}
\setlength{\mylentwo}{\baselineskip}
\setlength{\mylenone}{\mylenone + 1pt}
\setlength{\mylen}{(\textwidth - \mylenone)*\real{0.5}}
\begin{longtable}[l]{@{\hspace*{\mylen}}>{\setlength\parfillskip{0pt}}p{\mylenone}@{}@{}l@{}}
 & \\[-\the\mylentwo]
\centering{॥ १३ \hspace*{-1.5mm}॥} & \\ \nopagebreak
निमि अरु नव योगेश्वरा पादत्रान की हौं सरन & ॥\\
कबि हरि करभाजन भक्तिरत्नाकर भारी & ।\\ \nopagebreak
अन्तरिच्छ अरु चमस अनन्यता पधति उधारी & ॥\\
प्रबुध प्रेम की रासि भूरिदा आबिरहोता & ।\\ \nopagebreak
पिप्पल द्रुमिल प्रसिद्ध भवाब्धि पार के पोता & ॥\\
जयंतीनंदन जगत के त्रिबिध ताप आमयहरन & ।\\ \nopagebreak
निमि अरु नव योगेश्वरा पादत्रान की हौं सरन & ॥
\end{longtable}
}
}
\fancyhead[LO,RE]{{\textmd{\Large १३: नौ योगेश्वर}}}
\begin{sloppypar}\justifying\hyphenrules{nohyphenation}
\textbf{मूलार्थ}—(१)~\textbf{श्रीकवि} (२)~\textbf{श्रीहरि} और (३)~\textbf{श्रीकरभाजन}—ये भक्तिके विशाल महासागर हैं। (४)~\textbf{श्रीअन्तरिक्ष} और (५)~\textbf{श्रीचमस}ने अनन्यताकी पद्धतिका उद्धार किया है। (६)~\textbf{श्रीप्रबुद्ध} प्रेमकी राशि हैं। (७)~\textbf{श्रीआविर्होत्र}—\textbf{भूरिदा} अर्थात् दिव्य ज्ञान, भक्ति और विज्ञानके अनन्त दानी हैं। (८)~\textbf{श्रीपिप्पलायन} और (९)~\textbf{श्रीद्रुमिल}—ये भवसागरके पारके लिये प्रसिद्ध जहाज हैं। भगवान् ऋषभदेव और जयन्तीजीके ये नवों पुत्र संसारके तीनों तापों और रोगोंको हरनेवाले हैं। निमि और उन्हें भागवत धर्मका उपदेश करनेवाले इन नौ योगेश्वरोंकी चरणपादुकाकी मैं शरण चाहता हूँ, और मैं उनकी शरणमें हूँ। इन नवयोगश्वरोंकी चर्चा श्रीमद्भागवतजीके एकादश स्कन्धके द्वितीय अध्यायसे पञ्चम अध्याय तक वर्णित है।
\end{sloppypar}
\begin{sloppypar}\justifying\hyphenrules{nohyphenation}
अब नाभाजी नवधा भक्तिके नव आदर्शोंकी चर्चा करते हैं। प्रह्लादजीने हिरण्यकशिपुके समक्ष नवधा भक्तिकी इस प्रकार चर्चा की है—
\end{sloppypar}

{\bfseries
\setlength{\mylenone}{0pt}
\settowidth{\mylentwo}{श्रवणं कीर्तनं विष्णोः स्मरणं पादसेवनम्}
\setlength{\mylenone}{\maxof{\mylenone}{\mylentwo}}
\settowidth{\mylentwo}{अर्चनं वन्दनं दास्यं सख्यमात्मनिवेदनम्}
\setlength{\mylenone}{\maxof{\mylenone}{\mylentwo}}
\settowidth{\mylentwo}{इति पुंसाऽर्पिता विष्णौ भक्तिश्चेन्नवलक्षणा}
\setlength{\mylenone}{\maxof{\mylenone}{\mylentwo}}
\settowidth{\mylentwo}{क्रियते भगवत्यद्धा तन्मन्येऽधीतमुत्तमम्}
\setlength{\mylenone}{\maxof{\mylenone}{\mylentwo}}
\setlength{\mylentwo}{\baselineskip}
\setlength{\mylenone}{\mylenone + 1pt}
\setlength{\mylen}{(\textwidth - \mylenone)*\real{0.5}}
\begin{longtable}[l]{@{\hspace*{\mylen}}>{\setlength\parfillskip{0pt}}p{\mylenone}@{}@{}l@{}}
 & \\[-\the\mylentwo]
श्रवणं कीर्तनं विष्णोः स्मरणं पादसेवनम् & ।\\ \nopagebreak
अर्चनं वन्दनं दास्यं सख्यमात्मनिवेदनम् & ॥\\
इति पुंसाऽर्पिता विष्णौ भक्तिश्चेन्नवलक्षणा & ।\\ \nopagebreak
क्रियते भगवत्यद्धा तन्मन्येऽधीतमुत्तमम् & ॥\\ \nopagebreak
\caption*{(भा.पु.~७.५.२३-२४)}
\end{longtable}
}

\begin{sloppypar}\justifying\hyphenrules{nohyphenation}
भगवान्‌की कथाका श्रवण, भगवन्नामका संकीर्तन, भगवान्‌का स्मरण, भगवान्‌के श्रीचरण\-कमलका सेवन, भगवान्‌का पूजन, भगवान्‌का वन्दन, भगवान्‌के प्रति दास्य भाव, भगवान्‌के प्रति सख्य अर्थात् विश्वास और मित्रता, और भगवान्‌के प्रति आत्मनिवेदन अर्थात् सर्वसमर्पण—यही नवधा भक्ति है।
\end{sloppypar}

\addcontentsline{toc}{section}{\texorpdfstring{पद १४: नवधा भक्तिके नौ आदर्श}{१४: नवधा भक्तिके नौ आदर्श}}

{\relscale{1.1875}
{\bfseries
\setlength{\mylenone}{0pt}
\settowidth{\mylentwo}{}
\setlength{\mylenone}{\maxof{\mylenone}{\mylentwo}}
\settowidth{\mylentwo}{पदपराग करुना करौ जे नेता नवधा भक्ति के}
\setlength{\mylenone}{\maxof{\mylenone}{\mylentwo}}
\settowidth{\mylentwo}{श्रवन परीच्छित सुमति व्याससावक कीरंतन}
\setlength{\mylenone}{\maxof{\mylenone}{\mylentwo}}
\settowidth{\mylentwo}{सुठि सुमिरन प्रहलाद पृथु पूजा कमला चरननि मन}
\setlength{\mylenone}{\maxof{\mylenone}{\mylentwo}}
\settowidth{\mylentwo}{बंदन सुफलक सुबन दास दीपत्ति कपीश्वर}
\setlength{\mylenone}{\maxof{\mylenone}{\mylentwo}}
\settowidth{\mylentwo}{सख्यत्वे पारथ समर्पन आतम बलिधर}
\setlength{\mylenone}{\maxof{\mylenone}{\mylentwo}}
\settowidth{\mylentwo}{उपजीवी इन नाम के एते त्राता अगतिके}
\setlength{\mylenone}{\maxof{\mylenone}{\mylentwo}}
\settowidth{\mylentwo}{पदपराग करुना करौ जे नेता नवधा भक्ति के}
\setlength{\mylenone}{\maxof{\mylenone}{\mylentwo}}
\setlength{\mylentwo}{\baselineskip}
\setlength{\mylenone}{\mylenone + 1pt}
\setlength{\mylen}{(\textwidth - \mylenone)*\real{0.5}}
\begin{longtable}[l]{@{\hspace*{\mylen}}>{\setlength\parfillskip{0pt}}p{\mylenone}@{}@{}l@{}}
 & \\[-\the\mylentwo]
\centering{॥ १४ \hspace*{-1.5mm}॥} & \\ \nopagebreak
पदपराग करुना करौ जे नेता नवधा भक्ति के & ॥\\
श्रवन परीच्छित सुमति व्याससावक कीरंतन & ।\\ \nopagebreak
सुठि सुमिरन प्रहलाद पृथु पूजा कमला चरननि मन & ॥\\
बंदन सुफलक सुबन दास दीपत्ति कपीश्वर & ।\\ \nopagebreak
सख्यत्वे पारथ समर्पन आतम बलिधर & ॥\\
उपजीवी इन नाम के एते त्राता अगतिके & ।\\ \nopagebreak
पदपराग करुना करौ जे नेता नवधा भक्ति के & ॥
\end{longtable}
}
}
\fancyhead[LO,RE]{{\textmd{\Large १४: नौ आदर्श}}}
\begin{sloppypar}\justifying\hyphenrules{nohyphenation}
\textbf{मूलार्थ}—नाभाजी कहते हैं कि नवधा भक्तिके जो नेता रहे हैं, आदर्श रहे हैं, वे नवों महाभागवत अपने चरणकमलके परागके द्वारा मुझपर करुणा करें। ये हैं—(१)~श्रवणमें सुन्दर बुद्धिवाले महाराज \textbf{परीक्षित्} (२)~ भगवान्‌के कीर्तनमें सुन्दर बुद्धिवाले \textbf{व्याससावक} अर्थात् व्यासपुत्र \textbf{श्रीशुकाचार्यजी} महाराज (३)~भगवान्‌के सुन्दर स्मरणमें \textbf{श्रीप्रह्लाद} (४)~भगवान्‌के श्रीचरणकमलके सेवनमें \textbf{कमला} अर्थात् \textbf{श्रीलक्ष्मीजी} (५)~भगवान्‌के पूजनमें \textbf{श्रीपृथुजी} (६)~ भगवान्‌के वन्दनमें श्वफल्कके पुत्र \textbf{श्रीअक्रूरजी} (७)~भगवान्‌के दास्यभावकी दीप्तिमें अर्थात् प्रकाशमें \textbf{श्रीहनुमान्‌जी} महाराज (८)~भगवान्‌के सख्यत्व अर्थात् सख्यभक्तिमें पृथापुत्र \textbf{श्रीअर्जुन} (और उनके चारों भ्राता युधिष्ठिरजी, भीमजी, नकुलजी और सहदेवजी भी) और (९)~भगवान्‌के आत्मनिवेदनमें दैत्यराज \textbf{श्रीबलि}—इन नामोंके \textbf{उपजीवी} अर्थात् श्रीपरीक्षित्, श्रीशुकाचार्य, श्रीप्रह्लाद, भगवती लक्ष्मी, श्रीपृथु, श्रीअक्रूर, श्रीहनुमान्‌जी, श्रीअर्जुन और श्रीबलि—ये उनके रक्षक हैं जिनकी कोई गति नहीं है अथवा \textbf{अ}कार अर्थात् भगवान् वासुदेव ही जिनकी गति हैं—उनकी भी ये रक्षा करते रहते हैं। अथवा मैं नाभा इन नवों महाभक्तोंके नामोंका उपजीवी हूँ, अर्थात् इन्हींसे मेरी जीविका चल रही है, और ये मुझ गतिहीनके रक्षक हैं। इनके लिये एक श्लोक है—
\end{sloppypar}

{\bfseries
\setlength{\mylenone}{0pt}
\settowidth{\mylentwo}{श्रीकृष्णश्रवणे परीक्षिदभवद्वैयासकिः कीर्तने}
\setlength{\mylenone}{\maxof{\mylenone}{\mylentwo}}
\settowidth{\mylentwo}{प्रह्लादः स्मरणे तदङ्घ्रिभजने लक्ष्मीः पृथुः पूजने}
\setlength{\mylenone}{\maxof{\mylenone}{\mylentwo}}
\settowidth{\mylentwo}{अक्रूरस्त्वथ वन्दने च हनुमान्दास्येऽथ सख्येऽर्जुनः}
\setlength{\mylenone}{\maxof{\mylenone}{\mylentwo}}
\settowidth{\mylentwo}{सर्वस्वात्मनिवेदने बलिरभूत्कृष्णाप्तिरेषां फलम्}
\setlength{\mylenone}{\maxof{\mylenone}{\mylentwo}}
\setlength{\mylentwo}{\baselineskip}
\setlength{\mylenone}{\mylenone + 1pt}
\setlength{\mylen}{(\textwidth - \mylenone)*\real{0.5}}
\begin{longtable}[l]{@{\hspace*{\mylen}}>{\setlength\parfillskip{0pt}}p{\mylenone}@{}@{}l@{}}
 & \\[-\the\mylentwo]
श्रीकृष्णश्रवणे परीक्षिदभवद्वैयासकिः कीर्तने & \\ \nopagebreak
प्रह्लादः स्मरणे तदङ्घ्रिभजने लक्ष्मीः पृथुः पूजने & ।\\
अक्रूरस्त्वथ वन्दने च हनुमान्दास्येऽथ सख्येऽर्जुनः & \\ \nopagebreak
सर्वस्वात्मनिवेदने बलिरभूत्कृष्णाप्तिरेषां फलम् & ॥
\end{longtable}
}

\begin{sloppypar}\justifying\hyphenrules{nohyphenation}
अब नाभाजी उन महाभागवतोंकी चर्चा कर रहे हैं जो भगवान्‌की प्रसन्नताका आनन्द जानते हैं, और जो भगवान्‌के प्रसाद अर्थात् उपभुक्त प्रसादके स्वादका आनन्द भी जानते हैं। \textbf{प्रसाद} शब्द प्रसन्नता और नैवेद्यग्रहण—इन दोनों अर्थोंमें प्रसिद्ध है।
\end{sloppypar}

\addcontentsline{toc}{section}{\texorpdfstring{पद १५: भगवत्प्रसादके स्वादज्ञाता}{१५: भगवत्प्रसादके स्वादज्ञाता}}

{\relscale{1.1875}
{\bfseries
\setlength{\mylenone}{0pt}
\settowidth{\mylentwo}{}
\setlength{\mylenone}{\maxof{\mylenone}{\mylentwo}}
\settowidth{\mylentwo}{हरिप्रसाद रस स्वाद के भक्त इते परमान}
\setlength{\mylenone}{\maxof{\mylenone}{\mylentwo}}
\settowidth{\mylentwo}{शंकर शुक सनकादि कपिल नारद हनुमाना}
\setlength{\mylenone}{\maxof{\mylenone}{\mylentwo}}
\settowidth{\mylentwo}{बिष्वक्सेन प्रह्लाद बली भीषम जग जाना}
\setlength{\mylenone}{\maxof{\mylenone}{\mylentwo}}
\settowidth{\mylentwo}{अर्जुन ध्रुव अँबरीष विभीषण महिमा भारी}
\setlength{\mylenone}{\maxof{\mylenone}{\mylentwo}}
\settowidth{\mylentwo}{अनुरागी अक्रूर सदा उद्धव अधिकारी}
\setlength{\mylenone}{\maxof{\mylenone}{\mylentwo}}
\settowidth{\mylentwo}{भगवंत भुक्त अवशिष्ट की कीरति कहत सुजान}
\setlength{\mylenone}{\maxof{\mylenone}{\mylentwo}}
\settowidth{\mylentwo}{हरिप्रसाद रस स्वाद के भक्त इते परमान}
\setlength{\mylenone}{\maxof{\mylenone}{\mylentwo}}
\setlength{\mylentwo}{\baselineskip}
\setlength{\mylenone}{\mylenone + 1pt}
\setlength{\mylen}{(\textwidth - \mylenone)*\real{0.5}}
\begin{longtable}[l]{@{\hspace*{\mylen}}>{\setlength\parfillskip{0pt}}p{\mylenone}@{}@{}l@{}}
 & \\[-\the\mylentwo]
\centering{॥ १५ \hspace*{-1.5mm}॥} & \\ \nopagebreak
हरिप्रसाद रस स्वाद के भक्त इते परमान & ॥\\
शंकर शुक सनकादि कपिल नारद हनुमाना & ।\\ \nopagebreak
बिष्वक्सेन प्रह्लाद बली भीषम जग जाना & ॥\\
अर्जुन ध्रुव अँबरीष विभीषण महिमा भारी & ।\\ \nopagebreak
अनुरागी अक्रूर सदा उद्धव अधिकारी & ॥\\
भगवंत भुक्त अवशिष्ट की कीरति कहत सुजान & ।\\ \nopagebreak
हरिप्रसाद रस स्वाद के भक्त इते परमान & ॥
\end{longtable}
}
}
\fancyhead[LO,RE]{{\textmd{\Large १५: भगवत्प्रसादके स्वादज्ञाता}}}
\begin{sloppypar}\justifying\hyphenrules{nohyphenation}
\textbf{मूलार्थ}—(१)~\textbf{श्रीशङ्करजी} (२)~\textbf{श्रीशुकाचार्य} (३)~\textbf{श्रीसनकादि} (४)~\textbf{श्रीकपिल} (५)~\textbf{श्रीनारद} (६)~\textbf{श्रीहनुमान्‌जी} महाराज (७)~\textbf{श्रीविष्वक्सेन} (८)~\textbf{श्रीप्रह्लाद} (९)~\textbf{श्रीबलि} और (१०)~\textbf{श्रीभीष्म}—इनको सारा संसार जानता है। ये भगवान्‌की प्रसन्नताका स्वाद जानते हैं। इसी प्रकार (११)~\textbf{श्रीअर्जुन} (१२)~\textbf{श्रीध्रुव} (१३)~\textbf{श्रीअम्बरीष} और (१४)~\textbf{श्रीविभीषण}की बहुत बड़ी महिमा है। भगवान्‌के भुक्त प्रसादके (१५)~\textbf{श्रीअक्रूर} अत्यन्त अनुरागी हैं और (१६)~\textbf{श्रीउद्धव} इसके अधिकारी भी हैं। ये सभी भागवत भगवान्‌के नैवेद्यके उच्छिष्टकी कीर्ति सदैव कहते रहते हैं, अर्थात् इन्हें भगवान्‌के भुक्तके जूठनका भी अनुभव है और भगवान्‌की प्रसन्नताका भी अनुभव है।
\end{sloppypar}
\begin{sloppypar}\justifying\hyphenrules{nohyphenation}
विभीषणजी स्वयं गीतावलीजीमें कहते हैं—\textbf{तुलसी पट ऊतरे ओढ़िहौं उबरी जूठनि खाउँगो} (गी.~५.३०.४) और ध्रुव कहते हैं—\textbf{उच्छिष्टभोजिनो दासास्तव मायां जयेमहि} (भा.पु.~११.६.४६)।
\end{sloppypar}
\begin{sloppypar}\justifying\hyphenrules{nohyphenation}
अब नाभाजी बहुत-से राजर्षि-महर्षियोंकी चर्चा करते हैं।
\end{sloppypar}

\addcontentsline{toc}{section}{\texorpdfstring{पद १६: भगवद्ध्यानपरायण ऋषिगण}{१६: भगवद्ध्यानपरायण ऋषिगण}}

{\relscale{1.1875}
{\bfseries
\setlength{\mylenone}{0pt}
\settowidth{\mylentwo}{}
\setlength{\mylenone}{\maxof{\mylenone}{\mylentwo}}
\settowidth{\mylentwo}{ध्यान चतुर्भुज चित धर्यो तिनहिं सरन हौं अनुसरौं}
\setlength{\mylenone}{\maxof{\mylenone}{\mylentwo}}
\settowidth{\mylentwo}{अगस्त्य पुलस्त्य पुलह च्यबन बसिष्ठ सौभरि ऋषि}
\setlength{\mylenone}{\maxof{\mylenone}{\mylentwo}}
\settowidth{\mylentwo}{कर्दम अत्रि ऋचीक गर्ग गौतम ब्यासशिषि}
\setlength{\mylenone}{\maxof{\mylenone}{\mylentwo}}
\settowidth{\mylentwo}{लोमस भृगु दालभ्य अंगिरा शृंगि प्रकासी}
\setlength{\mylenone}{\maxof{\mylenone}{\mylentwo}}
\settowidth{\mylentwo}{मांडव्य बिश्वामित्र दुर्बासा सहस अठासी}
\setlength{\mylenone}{\maxof{\mylenone}{\mylentwo}}
\settowidth{\mylentwo}{जाबालि जमदग्नि मायादर्श कश्यप परबत पाराशर पदरज धरौं}
\setlength{\mylenone}{\maxof{\mylenone}{\mylentwo}}
\settowidth{\mylentwo}{ध्यान चतुर्भुज चित धर्यो तिनहिं सरन हौं अनुसरौं}
\setlength{\mylenone}{\maxof{\mylenone}{\mylentwo}}
\setlength{\mylentwo}{\baselineskip}
\setlength{\mylenone}{\mylenone + 1pt}
\setlength{\mylen}{(\textwidth - \mylenone)*\real{0.5}}
\begin{longtable}[l]{@{\hspace*{\mylen}}>{\setlength\parfillskip{0pt}}p{\mylenone}@{}@{}l@{}}
 & \\[-\the\mylentwo]
\centering{॥ १६ \hspace*{-1.5mm}॥} & \\ \nopagebreak
ध्यान चतुर्भुज चित धर्यो तिनहिं सरन हौं अनुसरौं & ॥\\
अगस्त्य पुलस्त्य पुलह च्यबन बसिष्ठ सौभरि ऋषि & ।\\ \nopagebreak
कर्दम अत्रि ऋचीक गर्ग गौतम ब्यासशिषि & ॥\\
लोमस भृगु दालभ्य अंगिरा शृंगि प्रकासी & ।\\ \nopagebreak
मांडव्य बिश्वामित्र दुर्बासा सहस अठासी & ॥\\
जाबालि जमदग्नि मायादर्श कश्यप परबत पाराशर पदरज धरौं & ।\\ \nopagebreak
ध्यान चतुर्भुज चित धर्यो तिनहिं सरन हौं अनुसरौं & ॥
\end{longtable}
}
}
\fancyhead[LO,RE]{{\textmd{\Large १६: ध्यानपरायण ऋषिगण}}}
\begin{sloppypar}\justifying\hyphenrules{nohyphenation}
\textbf{मूलार्थ}—जिन राजर्षि-महर्षियोंने \textbf{चतुर्भुज} अर्थात् चार भुजाओंवाले भगवान् विष्णुके ध्यानको, अथवा \textbf{चतुर्भुज} अर्थात् भक्तोंके पत्र-पुष्प-फल-जल रूप नैवेद्यको ग्रहण करनेवाले चारों वस्तुओंके भोक्ता भगवान् श्रीराम\-कृष्णान्यतरके ध्यानको जिन्होंने चित्तमें धारण कर लिया है, उनकी शरणका मैं अनुसरण करता हूँ। जैसे (१)~महर्षि \textbf{अगस्त्य} (२)~महर्षि \textbf{पुलस्त्य} (३)~महर्षि \textbf{पुलह} (४)~महर्षि \textbf{च्यवन} (५)~महर्षि \textbf{वसिष्ठ}, जो श्रीरामजीके गुरुदेव हैं (६)~महर्षि \textbf{सौभरि}, जिनको अन्तमें वैराग्य हुआ (७)~महर्षि \textbf{कर्दम}, जो कपिलदेवके पिताश्री हैं (८)~महर्षि \textbf{अत्रि}, जो सप्तर्षियोंमें एक हैं, और ब्रह्माजीके मानसपुत्रोंमें द्वितीय हैं। इन्होंने ही श्रीचित्रकूटमें भगवान् श्रीसीता-राम-लक्ष्मणका स्वागत किया और \textbf{नमामि भक्तवत्सलम्} (मा.~३.४.१-१२) जैसे स्तोत्रका गायन किया (९)~महर्षि \textbf{ऋचीक}, जो जमदग्निजीके पिता हैं, जिनके चरुके प्रसादसे जमदग्नि और विश्वामित्र दोनोंकी उत्पत्ति हुई और (१०)~महर्षि \textbf{गर्ग}—इन्होंने ही भगवान् कृष्णका नामकरण किया। इनके संदर्भमें भागवतके दसवें स्कन्धके आठवें अध्यायके प्रथम श्लोकमें कहा गया—
\end{sloppypar}

{\bfseries
\setlength{\mylenone}{0pt}
\settowidth{\mylentwo}{गर्गः पुरोहितो राजन् यदूनां सुमहातपाः}
\setlength{\mylenone}{\maxof{\mylenone}{\mylentwo}}
\settowidth{\mylentwo}{व्रजं जगाम नन्दस्य वसुदेवप्रचोदितः}
\setlength{\mylenone}{\maxof{\mylenone}{\mylentwo}}
\setlength{\mylentwo}{\baselineskip}
\setlength{\mylenone}{\mylenone + 1pt}
\setlength{\mylen}{(\textwidth - \mylenone)*\real{0.5}}
\begin{longtable}[l]{@{\hspace*{\mylen}}>{\setlength\parfillskip{0pt}}p{\mylenone}@{}@{}l@{}}
 & \\[-\the\mylentwo]
गर्गः पुरोहितो राजन् यदूनां सुमहातपाः & ।\\ \nopagebreak
व्रजं जगाम नन्दस्य वसुदेवप्रचोदितः & ॥\\ \nopagebreak
\caption*{(भा.पु.~१०.८.१)}
\end{longtable}
}

\begin{sloppypar}\justifying\hyphenrules{nohyphenation}
(११)~महर्षि \textbf{गौतम}—अहल्याजीके पति। इन्होंने ही तो अहल्याको पाषाण बननेका शाप दिया। इनके संबन्धमें रामचरितमानसमें कहा गया—
\end{sloppypar}

{\bfseries
\setlength{\mylenone}{0pt}
\settowidth{\mylentwo}{गौतम नारी स्राप बस उपल देह धरि धीर}
\setlength{\mylenone}{\maxof{\mylenone}{\mylentwo}}
\settowidth{\mylentwo}{चरन कमल रज चाहती कृपा करहु रघुबीर}
\setlength{\mylenone}{\maxof{\mylenone}{\mylentwo}}
\setlength{\mylentwo}{\baselineskip}
\setlength{\mylenone}{\mylenone + 1pt}
\setlength{\mylen}{(\textwidth - \mylenone)*\real{0.5}}
\begin{longtable}[l]{@{\hspace*{\mylen}}>{\setlength\parfillskip{0pt}}p{\mylenone}@{}@{}l@{}}
 & \\[-\the\mylentwo]
गौतम नारी स्राप बस उपल देह धरि धीर & ।\\ \nopagebreak
चरन कमल रज चाहती कृपा करहु रघुबीर & ॥\\ \nopagebreak
\caption*{(मा.~१.२१०)}
\end{longtable}
}

\begin{sloppypar}\justifying\hyphenrules{nohyphenation}
इसी प्रकार (१२)~वेदव्यासजीके अनेक शिष्य (१३)~महर्षि \textbf{लोमश}, जो काकभुशुण्डिजीको पहले तो शाप देते हैं फिर उनके गुरुदेव बनकर उन्हें धन्य कर देते हैं (१४)~महर्षि \textbf{भृगु} (१५)~महर्षि \textbf{दाल्भ्य} (१६)~\textbf{श्रीअङ्गिरा} (१७)~परम प्रकाशवान् \textbf{शृङ्गी} अथवा \textbf{ऋष्यशृङ्ग}—इन्हींके द्वारा किये गए पुत्रेष्टियज्ञसे भगवान् श्रीरामजीका आविर्भाव हुआ, इसलिये इन्हें \textbf{प्रकासी} कहा गया—प्रकाशमान ऋष्यशृङ्ग (१८)~महर्षि \textbf{माण्डव्य}—इन्होंने ही तो यमराजको शाप देकर विदुर बना दिया (१९)~महर्षि \textbf{विश्वामित्र}, जो गायत्रीजीके द्रष्टा और भगवान् श्रीरामके गुरु रहे हैं, और जिनकी कथा रामायणमें बहुत रोचकतासे प्रस्तुत की गई है—
\end{sloppypar}

{\bfseries
\setlength{\mylenone}{0pt}
\setlength{\mylenthree}{0pt}
\settowidth{\mylentwo}{बिश्वामित्र महामुनि ग्यानी}
\setlength{\mylenone}{\maxof{\mylenone}{\mylentwo}}
\settowidth{\mylenfour}{बसहिं बिपिन शुभ आश्रम जानी}
\setlength{\mylenthree}{\maxof{\mylenthree}{\mylenfour}}
\setlength{\mylentwo}{\baselineskip}
\setlength{\mylenone}{\mylenone + 1pt}
\setlength{\mylenfour}{\baselineskip}
\setlength{\mylenthree}{\mylenthree + 1pt}
\setlength{\mylen}{(\textwidth - \mylenone)}
\setlength{\mylen}{(\mylen - \mylenthree)*\real{0.5}}
\setlength{\mylen}{(\mylen - 4pt)}
\begin{longtable}[l]{@{\hspace*{\mylen}}>{\setlength\parfillskip{0pt}}p{\mylenone}@{}@{}l@{\hspace{6pt}}>{\setlength\parfillskip{0pt}}p{\mylenthree}@{}@{}l@{}}
 & & & \\[-\the\mylentwo]
बिश्वामित्र महामुनि ग्यानी & । & बसहिं बिपिन शुभ आश्रम जानी & ॥\\ \nopagebreak
\caption*{(मा.~१.२०६.२)}
\end{longtable}
}

\begin{sloppypar}\justifying\hyphenrules{nohyphenation}
(२०)~महर्षि \textbf{दुर्वासा}, जिनके क्रोधकी कथा रामायण, महाभारत और पुराणोंमें बहुशः प्रसिद्ध है (२१)~अट्ठासी सहस्र ऋषि, जो पुराणसत्रके श्रोता रहे हैं। इसी प्रकार (२२)~महर्षि \textbf{जाबालि}, जिनका वाल्मीकीय\-रामायणमें श्रीरामजीसे बहुत कथनोपकथन हुआ (२३)~महर्षि \textbf{जमदग्नि}, जो परशुरामजीके पिताश्री हैं और सम्प्रति सप्तर्षियोंमें द्वितीय महर्षिके रूपमें पूजित हो रहे हैं (२४)~\textbf{मायादर्श} अर्थात् मायाके दर्शन करनेवाले महर्षि \textbf{मार्कण्डेय} (२५)~महर्षि \textbf{कश्यप} जो सूर्यनारायण और संपूर्ण देवताओंके पिता हैं, और यही आगे चलकर श्रीदशरथ बनते हैं (२६)~परमऋषि \textbf{पर्वत} और (२७)~महर्षि \textbf{पराशर}, जो वेदव्यासजीके पिता और पराशरस्मृतिके रचयिता हैं—इनके चरणकमलकी धूलिको मैं अपने मस्तकपर धारण कर रहा हूँ।
\end{sloppypar}

\addcontentsline{toc}{section}{\texorpdfstring{पद १७: अष्टादश पुराण}{१७: अष्टादश पुराण}}

{\relscale{1.1875}
{\bfseries
\setlength{\mylenone}{0pt}
\settowidth{\mylentwo}{}
\setlength{\mylenone}{\maxof{\mylenone}{\mylentwo}}
\settowidth{\mylentwo}{साधन साध्य सत्रह पुराण फलरूपी श्रीभागवत}
\setlength{\mylenone}{\maxof{\mylenone}{\mylentwo}}
\settowidth{\mylentwo}{ब्रह्म विष्णु शिव लिंग पदम अस्कँद बिस्तारा}
\setlength{\mylenone}{\maxof{\mylenone}{\mylentwo}}
\settowidth{\mylentwo}{बामन मीन बराह अग्नि कूरम ऊदारा}
\setlength{\mylenone}{\maxof{\mylenone}{\mylentwo}}
\settowidth{\mylentwo}{गरुड नारदी भविष्य ब्रह्मबैबर्त श्रवण शुचि}
\setlength{\mylenone}{\maxof{\mylenone}{\mylentwo}}
\settowidth{\mylentwo}{मार्कंडेय ब्रह्मांड कथा नाना उपजे रुचि}
\setlength{\mylenone}{\maxof{\mylenone}{\mylentwo}}
\settowidth{\mylentwo}{परम धर्म श्रीमुखकथित चतुःश्लोकी निगम शत}
\setlength{\mylenone}{\maxof{\mylenone}{\mylentwo}}
\settowidth{\mylentwo}{साधन साध्य सत्रह पुराण फलरूपी श्रीभागवत}
\setlength{\mylenone}{\maxof{\mylenone}{\mylentwo}}
\setlength{\mylentwo}{\baselineskip}
\setlength{\mylenone}{\mylenone + 1pt}
\setlength{\mylen}{(\textwidth - \mylenone)*\real{0.5}}
\begin{longtable}[l]{@{\hspace*{\mylen}}>{\setlength\parfillskip{0pt}}p{\mylenone}@{}@{}l@{}}
 & \\[-\the\mylentwo]
\centering{॥ १७ \hspace*{-1.5mm}॥} & \\ \nopagebreak
साधन साध्य सत्रह पुराण फलरूपी श्रीभागवत & ॥\\
ब्रह्म विष्णु शिव लिंग पदम अस्कँद बिस्तारा & ।\\ \nopagebreak
बामन मीन बराह अग्नि कूरम ऊदारा & ॥\\
गरुड नारदी भविष्य ब्रह्मबैबर्त श्रवण शुचि & ।\\ \nopagebreak
मार्कंडेय ब्रह्मांड कथा नाना उपजे रुचि & ॥\\
परम धर्म श्रीमुखकथित चतुःश्लोकी निगम शत & ।\\ \nopagebreak
साधन साध्य सत्रह पुराण फलरूपी श्रीभागवत & ॥
\end{longtable}
}
}
\fancyhead[LO,RE]{{\textmd{\Large १७: अष्टादश पुराण}}}
\begin{sloppypar}\justifying\hyphenrules{nohyphenation}
\textbf{मूलार्थ}—सत्रहों पुराण तो साधन-साध्य हैं परन्तु श्रीभागवत इनका फलरूप ही है। जैसे (१)~\textbf{ब्रह्मपुराण} (२)~\textbf{विष्णुपुराण} (३)~\textbf{शिवपुराण} (४)~\textbf{लिङ्गपुराण} (५)~\textbf{पद्मपुराण} (६)~विस्तृत \textbf{स्कन्दपुराण} (७)~\textbf{वामनपुराण} (८)~\textbf{मत्स्यपुराण} (९)~\textbf{वराहपुराण} (१०)~\textbf{अग्निपुराण} (११)~परम उदार \textbf{कूर्मपुराण} (१२)~\textbf{गरुडपुराण} (१३)~\textbf{नारदपुराण} (१४)~\textbf{भविष्यपुराण} (१५)~श्रवण करनेमें पवित्र \textbf{ब्रह्मवैवर्तपुराण} (१६)~\textbf{मार्कण्डेयपुराण} और (१७)~\textbf{ब्रह्माण्डपुराण}, जिनकी नाना कथाओंमें रुचि उत्पन्न होती है—ये सत्रहों पुराण साधन-साध्य हैं। परन्तु \textbf{भागवतपुराण} इसलिये फलरूप है कि श्रीमुख द्वारा कथित इसमें परमधर्मका वर्णन है और श्रेष्ठ वेदके रूपमें यहाँ चतुःश्लोकी भागवत कही गई है।
\end{sloppypar}
\begin{sloppypar}\justifying\hyphenrules{nohyphenation}
भागवतकी चतुःश्लोकी इस प्रकार है—
\end{sloppypar}

{\bfseries
\setlength{\mylenone}{0pt}
\settowidth{\mylentwo}{अहमेवासमेवाग्रे नान्यद्यत्सदसत्परम्}
\setlength{\mylenone}{\maxof{\mylenone}{\mylentwo}}
\settowidth{\mylentwo}{पश्चादहं यदेतच्च योऽवशिष्येत सोऽस्म्यहम्}
\setlength{\mylenone}{\maxof{\mylenone}{\mylentwo}}
\settowidth{\mylentwo}{ऋतेऽर्थं यत्प्रतीयेत न प्रतीयेत चात्मनि}
\setlength{\mylenone}{\maxof{\mylenone}{\mylentwo}}
\settowidth{\mylentwo}{तद्विद्यादात्मनो मायां यथाऽऽभासो यथा तमः}
\setlength{\mylenone}{\maxof{\mylenone}{\mylentwo}}
\settowidth{\mylentwo}{यथा महान्ति भूतानि भूतेषूच्चावचेष्वनु}
\setlength{\mylenone}{\maxof{\mylenone}{\mylentwo}}
\settowidth{\mylentwo}{प्रविष्टान्यप्रविष्टानि तथा तेषु न तेष्वहम्}
\setlength{\mylenone}{\maxof{\mylenone}{\mylentwo}}
\settowidth{\mylentwo}{एतावदेव जिज्ञास्यं तत्त्वजिज्ञासुनाऽऽत्मनः}
\setlength{\mylenone}{\maxof{\mylenone}{\mylentwo}}
\settowidth{\mylentwo}{अन्वयव्यतिरेकाभ्यां यत्स्यात्सर्वत्र सर्वदा}
\setlength{\mylenone}{\maxof{\mylenone}{\mylentwo}}
\setlength{\mylentwo}{\baselineskip}
\setlength{\mylenone}{\mylenone + 1pt}
\setlength{\mylen}{(\textwidth - \mylenone)*\real{0.5}}
\begin{longtable}[l]{@{\hspace*{\mylen}}>{\setlength\parfillskip{0pt}}p{\mylenone}@{}@{}l@{}}
 & \\[-\the\mylentwo]
अहमेवासमेवाग्रे नान्यद्यत्सदसत्परम् & ।\\ \nopagebreak
पश्चादहं यदेतच्च योऽवशिष्येत सोऽस्म्यहम् & ॥\\
ऋतेऽर्थं यत्प्रतीयेत न प्रतीयेत चात्मनि & ।\\ \nopagebreak
तद्विद्यादात्मनो मायां यथाऽऽभासो यथा तमः & ॥\\
यथा महान्ति भूतानि भूतेषूच्चावचेष्वनु & ।\\ \nopagebreak
प्रविष्टान्यप्रविष्टानि तथा तेषु न तेष्वहम् & ॥\\
एतावदेव जिज्ञास्यं तत्त्वजिज्ञासुनाऽऽत्मनः & ।\\ \nopagebreak
अन्वयव्यतिरेकाभ्यां यत्स्यात्सर्वत्र सर्वदा & ॥\\ \nopagebreak
\caption*{(भा.पु.~२.९.३२-३५)}
\end{longtable}
}

\begin{sloppypar}\justifying\hyphenrules{nohyphenation}
भगवान् कहते हैं कि सृष्टिके प्रारम्भमें भी और सृष्टिके पूर्व भी मैं ही था। ये जो कुछ सत्-असत् दिखाई पड़ रहा है, स्थूल-सूक्ष्म ये कुछ नहीं था। पश्चात् भी मैं ही रहूँगा। जो इस समय वर्तमान है वह भी मैं ही हूँ। परमात्माके दर्शनके अभावमें जो प्रतीत हो रही है, और परमात्माका साक्षात्कार हो जानेपर जो नहीं प्रतीत होती उसीको परमात्माकी माया कहते हैं। जैसे रात्रिमें जुगनूका प्रकाश प्रतीत होता है और दिनमें प्रतीत नहीं होता जबकि वह रहता है, उसी प्रकार अज्ञानमें यह माया प्रतीत होती है और ज्ञान होनेपर नहीं प्रतीत होती है। जिस प्रकार पाँचों महाभूत सभी पदार्थोंमें अंशतः रहते हैं, पूर्णतः नहीं रहते; उसी प्रकार मैं परमात्मा सबके हृदयमें अंशतः अर्थात् अन्तर्यामी रूपमें रहता हूँ, पूर्णतः कहीं नहीं रहता। तत्त्वजिज्ञासुके द्वारा यही जिज्ञास्य है, यही जानने योग्य है कि जो अन्वय और व्यतिरेकके द्वारा सर्वत्र विराजमान है अर्थात् सृष्टिके रहनेपर भी जो रहेगा और न रहनेपर भी जो विद्यमान रहेगा वही तो परमात्मतत्त्व है।
\end{sloppypar}

\addcontentsline{toc}{section}{\texorpdfstring{पद १८: अष्टादश स्मृति}{१८: अष्टादश स्मृति}}

{\relscale{1.1875}
{\bfseries
\setlength{\mylenone}{0pt}
\settowidth{\mylentwo}{}
\setlength{\mylenone}{\maxof{\mylenone}{\mylentwo}}
\settowidth{\mylentwo}{दस आठ स्मृति जिन उच्चरी तिन पद सरसिज भाल मो}
\setlength{\mylenone}{\maxof{\mylenone}{\mylentwo}}
\settowidth{\mylentwo}{मनुस्मृति आत्रेय वैष्णवी हारीतक जामी}
\setlength{\mylenone}{\maxof{\mylenone}{\mylentwo}}
\settowidth{\mylentwo}{जाग्यबल्क्य अंगिरा शनैश्चर सांवर्तक नामी}
\setlength{\mylenone}{\maxof{\mylenone}{\mylentwo}}
\settowidth{\mylentwo}{कात्यायनि शांडिल्य गौतमी बासिष्ठी दाषी}
\setlength{\mylenone}{\maxof{\mylenone}{\mylentwo}}
\settowidth{\mylentwo}{सुरगुरु शातातापि पराशर क्रतु मुनि भाषी}
\setlength{\mylenone}{\maxof{\mylenone}{\mylentwo}}
\settowidth{\mylentwo}{आशा पास उदारधी परलोक लोक साधन सो}
\setlength{\mylenone}{\maxof{\mylenone}{\mylentwo}}
\settowidth{\mylentwo}{दस आठ स्मृति जिन उच्चरी तिन पद सरसिज भाल मो}
\setlength{\mylenone}{\maxof{\mylenone}{\mylentwo}}
\setlength{\mylentwo}{\baselineskip}
\setlength{\mylenone}{\mylenone + 1pt}
\setlength{\mylen}{(\textwidth - \mylenone)*\real{0.5}}
\begin{longtable}[l]{@{\hspace*{\mylen}}>{\setlength\parfillskip{0pt}}p{\mylenone}@{}@{}l@{}}
 & \\[-\the\mylentwo]
\centering{॥ १८ \hspace*{-1.5mm}॥} & \\ \nopagebreak
दस आठ स्मृति जिन उच्चरी तिन पद सरसिज भाल मो & ॥\\
मनुस्मृति आत्रेय वैष्णवी हारीतक जामी & ।\\ \nopagebreak
जाग्यबल्क्य अंगिरा शनैश्चर सांवर्तक नामी & ॥\\
कात्यायनि शांडिल्य गौतमी बासिष्ठी दाषी & ।\\ \nopagebreak
सुरगुरु शातातापि पराशर क्रतु मुनि भाषी & ॥\\
आशा पास उदारधी परलोक लोक साधन सो & ।\\ \nopagebreak
दस आठ स्मृति जिन उच्चरी तिन पद सरसिज भाल मो & ॥
\end{longtable}
}
}
\fancyhead[LO,RE]{{\textmd{\Large १८: अष्टादश स्मृति}}}
\begin{sloppypar}\justifying\hyphenrules{nohyphenation}
\textbf{मूलार्थ}—अठारह स्मृतियोंका जिन आचार्योंने उच्चारण किया है, ऐसे (१)~\textbf{मनु} (२)~\textbf{अत्रि} (३)~\textbf{विष्णु} (४)~\textbf{हारीत} (५)~\textbf{यम} (६)~\textbf{याज्ञवल्क्य} (७)~\textbf{अङ्गिरा} (८)~\textbf{शनैश्चर} (९)~\textbf{सांवर्तक} (१०)~\textbf{कात्यायन} (११)~\textbf{शाण्डिल्य} (१२)~\textbf{गौतम} (१३)~\textbf{वसिष्ठ} (१४)~\textbf{दक्ष} (१५)~\textbf{बृहस्पति} (१६)~\textbf{शातातप} (१७)~\textbf{पराशर} और (१८)~\textbf{क्रतु}—इन आचार्योंके चरण\-कमल मेरे मस्तकपर सतत विराजमान रहें। ये स्मृतियाँ हैं—मनुस्मृति, अत्रिस्मृति, वैष्णवी स्मृति, हारीतकस्मृति, यामी स्मृति, याज्ञवल्क्यस्मृति, अङ्गिरःस्मृति, शनैश्चरस्मृति, सांवर्तकस्मृति, कात्यायनस्मृति, शाण्डिल्यस्मृति, गौतमस्मृति, वसिष्ठस्मृति, दक्षस्मृति, बृहस्पतिस्मृति, शातातपस्मृति, पराशरस्मृति, और क्रतुस्मृति। ये आचार्य ही हमारे जीवनकी आशा हैं, ये उदार बुद्धिवाले हैं, और ये परलोक और लोक दोनोंमें हमारे लिये साधनस्वरूप हैं।
\end{sloppypar}

\addcontentsline{toc}{section}{\texorpdfstring{पद १९: श्रीरामके मन्त्री}{१९: श्रीरामके मन्त्री}}

{\relscale{1.1875}
{\bfseries
\setlength{\mylenone}{0pt}
\settowidth{\mylentwo}{}
\setlength{\mylenone}{\maxof{\mylenone}{\mylentwo}}
\settowidth{\mylentwo}{पावैं भक्ति अनपायिनी जे रामसचिव सुमिरन करैं}
\setlength{\mylenone}{\maxof{\mylenone}{\mylentwo}}
\settowidth{\mylentwo}{धृष्टी बिजय जयंत नीतिपर सुचि सुबिनीता}
\setlength{\mylenone}{\maxof{\mylenone}{\mylentwo}}
\settowidth{\mylentwo}{राष्टरबर्धन निपुण सुराष्टर परम पुनीता}
\setlength{\mylenone}{\maxof{\mylenone}{\mylentwo}}
\settowidth{\mylentwo}{अशोक सदा आनंद धर्मपालक तत्ववेता}
\setlength{\mylenone}{\maxof{\mylenone}{\mylentwo}}
\settowidth{\mylentwo}{मंत्रीवर्य सुमंत्र चतुर्जुग मंत्री जेता}
\setlength{\mylenone}{\maxof{\mylenone}{\mylentwo}}
\settowidth{\mylentwo}{अनायास रघुपति प्रसन्न भवसागर दुस्तर तरैं}
\setlength{\mylenone}{\maxof{\mylenone}{\mylentwo}}
\settowidth{\mylentwo}{पावैं भक्ति अनपायिनी जे रामसचिव सुमिरन करैं}
\setlength{\mylenone}{\maxof{\mylenone}{\mylentwo}}
\setlength{\mylentwo}{\baselineskip}
\setlength{\mylenone}{\mylenone + 1pt}
\setlength{\mylen}{(\textwidth - \mylenone)*\real{0.5}}
\begin{longtable}[l]{@{\hspace*{\mylen}}>{\setlength\parfillskip{0pt}}p{\mylenone}@{}@{}l@{}}
 & \\[-\the\mylentwo]
\centering{॥ १९ \hspace*{-1.5mm}॥} & \\ \nopagebreak
पावैं भक्ति अनपायिनी जे रामसचिव सुमिरन करैं & ॥\\
धृष्टी बिजय जयंत नीतिपर सुचि सुबिनीता & ।\\ \nopagebreak
राष्टरबर्धन निपुण सुराष्टर परम पुनीता & ॥\\
अशोक सदा आनंद धर्मपालक तत्ववेता & ।\\ \nopagebreak
मंत्रीवर्य सुमंत्र चतुर्जुग मंत्री जेता & ॥\\
अनायास रघुपति प्रसन्न भवसागर दुस्तर तरैं & ।\\ \nopagebreak
पावैं भक्ति अनपायिनी जे रामसचिव सुमिरन करैं & ॥
\end{longtable}
}
}
\fancyhead[LO,RE]{{\textmd{\Large १९: श्रीरामके मन्त्री}}}
\begin{sloppypar}\justifying\hyphenrules{nohyphenation}
\textbf{मूलार्थ}—\textbf{रामसचिव} अर्थात् श्रीरामजीके आठों मन्त्रियोंका जो स्मरण करते हैं, वे अनपायिनी भक्ति पा जाएँगे, उनपर अनायास ही भगवान् श्रीराम सदैव प्रसन्न रहेंगे, और वे दुस्तर भवसागरको पार कर लेंगे। वे हैं—(१)~\textbf{धृष्टि} (२)~\textbf{विजय} और (३)~\textbf{जयन्त}, जो नीतिपरायण, पवित्र और अत्यन्त विनम्र हैं (४)~\textbf{राष्ट्रवर्धन}, जो अत्यन्त निपुण हैं (५)~\textbf{सुराष्ट्र}, जो अत्यन्त पवित्र हैं (६)~\textbf{अशोक}, जो सदा आनन्दमें रहते हैं (७)~\textbf{धर्मपाल}, जो तत्त्ववेत्ता हैं और (८)~\textbf{सुमन्त्र}—चारों युगोंमें जितने मन्त्री हैं, उनमें सबसे श्रेष्ठ मन्त्री सुमन्त्र हैं।
\end{sloppypar}

\addcontentsline{toc}{section}{\texorpdfstring{पद २०: श्रीरामके सहचर यूथपति}{२०: श्रीरामके सहचर यूथपति}}

{\relscale{1.1875}
{\bfseries
\setlength{\mylenone}{0pt}
\settowidth{\mylentwo}{}
\setlength{\mylenone}{\maxof{\mylenone}{\mylentwo}}
\settowidth{\mylentwo}{शुभ दृष्टि वृष्टि मोपर करौ जे सहचर रघुवीर के}
\setlength{\mylenone}{\maxof{\mylenone}{\mylentwo}}
\settowidth{\mylentwo}{दिनकरसुत हरिराज बालिबछ केसरि औरस}
\setlength{\mylenone}{\maxof{\mylenone}{\mylentwo}}
\settowidth{\mylentwo}{दधिमुख द्विबिद मयंद रीछपति सम को पौरस}
\setlength{\mylenone}{\maxof{\mylenone}{\mylentwo}}
\settowidth{\mylentwo}{उल्कासुभट सुषेन दरीमुख कुमुद नील नल}
\setlength{\mylenone}{\maxof{\mylenone}{\mylentwo}}
\settowidth{\mylentwo}{शरभर गवय गवाच्छ पनस गँधमादन अतिबल}
\setlength{\mylenone}{\maxof{\mylenone}{\mylentwo}}
\settowidth{\mylentwo}{पद्म अठारह यूथपति रामकाज भट भीर के}
\setlength{\mylenone}{\maxof{\mylenone}{\mylentwo}}
\settowidth{\mylentwo}{शुभ दृष्टि वृष्टि मोपर करौ जे सहचर रघुवीर के}
\setlength{\mylenone}{\maxof{\mylenone}{\mylentwo}}
\setlength{\mylentwo}{\baselineskip}
\setlength{\mylenone}{\mylenone + 1pt}
\setlength{\mylen}{(\textwidth - \mylenone)*\real{0.5}}
\begin{longtable}[l]{@{\hspace*{\mylen}}>{\setlength\parfillskip{0pt}}p{\mylenone}@{}@{}l@{}}
 & \\[-\the\mylentwo]
\centering{॥ २० \hspace*{-1.5mm}॥} & \\ \nopagebreak
शुभ दृष्टि वृष्टि मोपर करौ जे सहचर रघुवीर के & ॥\\
दिनकरसुत हरिराज बालिबछ केसरि औरस & ।\\ \nopagebreak
दधिमुख द्विबिद मयंद रीछपति सम को पौरस & ॥\\
उल्कासुभट सुषेन दरीमुख कुमुद नील नल & ।\\ \nopagebreak
शरभर गवय गवाच्छ पनस गँधमादन अतिबल & ॥\\
पद्म अठारह यूथपति रामकाज भट भीर के & ।\\ \nopagebreak
शुभ दृष्टि वृष्टि मोपर करौ जे सहचर रघुवीर के & ॥
\end{longtable}
}
}
\fancyhead[LO,RE]{{\textmd{\Large २०: श्रीरामके यूथपति}}}
\begin{sloppypar}\justifying\hyphenrules{nohyphenation}
\textbf{मूलार्थ}—अठारह पद्म यूथोंके अधिपति प्रभु श्रीरामके नित्य सहचर हैं एवं युद्धके अवसरपर भगवान् श्रीरामका काज करनेवाले हैं, अर्थात् ये यूथपति युद्धके अवसरपर भगवान् श्रीरामके राक्षसवध रूप कार्यमें नित्य सहायक हैं। ऐसे सीतापति श्रीराघवकी संहार\-लीलाके नित्य परिकर भट मुझपर शुभ दृष्टिकी वृष्टि करते रहें, अर्थात् मुझे अपनी कल्याणमयी चितवनसे निहारकर मुझ अकिञ्चनमें श्रीराम\-प्रेमको भरते रहें। इनमें प्रमुख हैं—(१)~वानरोंके राजा सूर्यपुत्र \textbf{सुग्रीव} (२)~वालिपुत्र युवराज \textbf{अङ्गद} (३)~केसरीजीके औरसपुत्र अञ्जनानन्दवर्धन \textbf{श्रीहनुमान्‌जी} महाराज (४)~\textbf{दधिमुख} (५)~\textbf{द्विविद} (६)~\textbf{मयन्द} (७)~जिनके समान और किसीका पौरुष नहीं है अर्थात् अतुल बलशाली ऋक्षराज \textbf{जाम्बवान्} (८)~\textbf{उल्कासुभट} अर्थात् अन्धकारके समय दीपक जलाकर सेवा करनेवाले उल्का\-सुभट नामक विशेष यूथपति (९)~\textbf{सुषेण} (१०)~\textbf{दरीमुख} (११)~\textbf{कुमुद} (१२)~\textbf{नील} (१३)~\textbf{नल} (१४)~\textbf{शरभ} (१५)~\textbf{गवय} (१६)~\textbf{गवाक्ष} (१७)~\textbf{पनस} और (१८)~अत्यन्त बलशाली \textbf{गन्धमादन}। इस प्रकार अठारह पद्म यूथ वानरोंके पूर्व\-वर्णित अठारह यूथपति अर्थात् सुग्रीव, अङ्गद, हनुमान्, दधिमुख, द्विविद, मयन्द, जाम्बवान्, उल्का\-सुभट, सुषेण, दरीमुख, कुमुद, नील, नल, शरभ, गवय, गवाक्ष, पनस और गन्धमादन—जो युद्धके समय श्रीराम\-कार्यके संपादनमें परमवीरता करते हैं वे मुझपर कल्याणमयी दृष्टिका वर्षण करते रहें। इसी आशयको रामचरितमानसके सुन्दरकाण्डमें शुकने भी रावणसे स्पष्ट किया है—
\end{sloppypar}

{\bfseries
\setlength{\mylenone}{0pt}
\setlength{\mylenthree}{0pt}
\settowidth{\mylentwo}{अस मैं सुना श्रवन दशकंधर}
\setlength{\mylenone}{\maxof{\mylenone}{\mylentwo}}
\settowidth{\mylenfour}{पदुम अठारह जूथप बंदर}
\setlength{\mylenthree}{\maxof{\mylenthree}{\mylenfour}}
\setlength{\mylentwo}{\baselineskip}
\setlength{\mylenone}{\mylenone + 1pt}
\setlength{\mylenfour}{\baselineskip}
\setlength{\mylenthree}{\mylenthree + 1pt}
\setlength{\mylen}{(\textwidth - \mylenone)}
\setlength{\mylen}{(\mylen - \mylenthree)*\real{0.5}}
\setlength{\mylen}{(\mylen - 4pt)}
\begin{longtable}[l]{@{\hspace*{\mylen}}>{\setlength\parfillskip{0pt}}p{\mylenone}@{}@{}l@{\hspace{6pt}}>{\setlength\parfillskip{0pt}}p{\mylenthree}@{}@{}l@{}}
 & & & \\[-\the\mylentwo]
अस मैं सुना श्रवन दशकंधर & । & पदुम अठारह जूथप बंदर & ॥\\ \nopagebreak
\caption*{(मा.~५.५५.३)}
\end{longtable}
}


\addcontentsline{toc}{section}{\texorpdfstring{पद २१: नौ नन्द}{२१: नौ नन्द}}

{\relscale{1.1875}
{\bfseries
\setlength{\mylenone}{0pt}
\settowidth{\mylentwo}{}
\setlength{\mylenone}{\maxof{\mylenone}{\mylentwo}}
\settowidth{\mylentwo}{ब्रज बड़े गोप पर्जन्य के सुत नीके नव नंद}
\setlength{\mylenone}{\maxof{\mylenone}{\mylentwo}}
\settowidth{\mylentwo}{धरानंद ध्रुवनंद तृतिय उपनंद सुनागर}
\setlength{\mylenone}{\maxof{\mylenone}{\mylentwo}}
\settowidth{\mylentwo}{चतुर्थ तहाँ अभिनंद नंद सुखसिंधु उजागर}
\setlength{\mylenone}{\maxof{\mylenone}{\mylentwo}}
\settowidth{\mylentwo}{सुठि सुनंद पशुपाल निर्मल निश्चय अभिनंदन}
\setlength{\mylenone}{\maxof{\mylenone}{\mylentwo}}
\settowidth{\mylentwo}{कर्मा धर्मानंद अनुज बल्लभ जगबंदन}
\setlength{\mylenone}{\maxof{\mylenone}{\mylentwo}}
\settowidth{\mylentwo}{आसपास वा बगर के जहँ बिहरत पसुप स्वछंद}
\setlength{\mylenone}{\maxof{\mylenone}{\mylentwo}}
\settowidth{\mylentwo}{ब्रज बड़े गोप पर्जन्य के सुत नीके नव नंद}
\setlength{\mylenone}{\maxof{\mylenone}{\mylentwo}}
\setlength{\mylentwo}{\baselineskip}
\setlength{\mylenone}{\mylenone + 1pt}
\setlength{\mylen}{(\textwidth - \mylenone)*\real{0.5}}
\begin{longtable}[l]{@{\hspace*{\mylen}}>{\setlength\parfillskip{0pt}}p{\mylenone}@{}@{}l@{}}
 & \\[-\the\mylentwo]
\centering{॥ २१ \hspace*{-1.5mm}॥} & \\ \nopagebreak
ब्रज बड़े गोप पर्जन्य के सुत नीके नव नंद & ॥\\
धरानंद ध्रुवनंद तृतिय उपनंद सुनागर & ।\\ \nopagebreak
चतुर्थ तहाँ अभिनंद नंद सुखसिंधु उजागर & ॥\\
सुठि सुनंद पशुपाल निर्मल निश्चय अभिनंदन & ।\\ \nopagebreak
कर्मा धर्मानंद अनुज बल्लभ जगबंदन & ॥\\
आसपास वा बगर के जहँ बिहरत पसुप स्वछंद & ।\\ \nopagebreak
ब्रज बड़े गोप पर्जन्य के सुत नीके नव नंद & ॥
\end{longtable}
}
}
\fancyhead[LO,RE]{{\textmd{\Large २१: नौ नन्द}}}
\begin{sloppypar}\justifying\hyphenrules{nohyphenation}
\textbf{मूलार्थ}—व्रजमें श्रेष्ठ गोप व्रजेन्द्र \textbf{पर्जन्य}के \textbf{नवनन्द} नामक पुत्र बड़े प्यारे हैं। इनमेंसे (१)~\textbf{ध्रुवनन्द} (२)~\textbf{धरानन्द} (३)~अत्यन्त चतुर \textbf{उपनन्द} (४)~\textbf{अभिनन्द} (५)~\textbf{नन्द} जो सुखके सागर हैं और उजागर भी हैं (६)~\textbf{सुनन्द} जो अत्यन्त पवित्र हैं और जो पशुओंका पालन करते हैं, जिनका निश्चय और अभिनन्दन अत्यन्त निर्मल है (७)~\textbf{कर्मानन्द} (८)~\textbf{धर्मानन्द} और (९)~जगत्‌के वन्दनीय सबसे छोटे भाई \textbf{वल्लभ}। ये सब गोप उस व्रजके आस-पास रहते हैं जहाँ स्वच्छन्द रूपसे गोप विचरण करते रहते हैं। व्रजमें बड़े गोप पर्जन्यजीके ये नौ पुत्र बहुत प्रसिद्ध हैं।
\end{sloppypar}

\addcontentsline{toc}{section}{\texorpdfstring{पद २२: श्रीराधाकृष्णपरिकर}{२२: श्रीराधाकृष्णपरिकर}}

{\relscale{1.1875}
{\bfseries
\setlength{\mylenone}{0pt}
\settowidth{\mylentwo}{}
\setlength{\mylenone}{\maxof{\mylenone}{\mylentwo}}
\settowidth{\mylentwo}{बाल बृद्ध नर नारि गोप हौं अर्थी उन पाद रज}
\setlength{\mylenone}{\maxof{\mylenone}{\mylentwo}}
\settowidth{\mylentwo}{नंदगोप उपनंद ध्रुव धरानंद महरि जसोदा}
\setlength{\mylenone}{\maxof{\mylenone}{\mylentwo}}
\settowidth{\mylentwo}{कीरतिदा वृषभानु कुँवरि सहचरि मन मोदा}
\setlength{\mylenone}{\maxof{\mylenone}{\mylentwo}}
\settowidth{\mylentwo}{मधुमंगल सुबल सुबाहु भोज अर्जुन श्रीदामा}
\setlength{\mylenone}{\maxof{\mylenone}{\mylentwo}}
\settowidth{\mylentwo}{मंडलि ग्वाल अनेक श्याम संगी बहु नामा}
\setlength{\mylenone}{\maxof{\mylenone}{\mylentwo}}
\settowidth{\mylentwo}{घोष निवासिनि की कृपा सुर नर बाँछित आदि अज}
\setlength{\mylenone}{\maxof{\mylenone}{\mylentwo}}
\settowidth{\mylentwo}{बाल बृद्ध नर नारि गोप हौं अर्थी उन पाद रज}
\setlength{\mylenone}{\maxof{\mylenone}{\mylentwo}}
\setlength{\mylentwo}{\baselineskip}
\setlength{\mylenone}{\mylenone + 1pt}
\setlength{\mylen}{(\textwidth - \mylenone)*\real{0.5}}
\begin{longtable}[l]{@{\hspace*{\mylen}}>{\setlength\parfillskip{0pt}}p{\mylenone}@{}@{}l@{}}
 & \\[-\the\mylentwo]
\centering{॥ २२ \hspace*{-1.5mm}॥} & \\ \nopagebreak
बाल बृद्ध नर नारि गोप हौं अर्थी उन पाद रज & ॥\\
नंदगोप उपनंद ध्रुव धरानंद महरि जसोदा & ।\\ \nopagebreak
कीरतिदा वृषभानु कुँवरि सहचरि मन मोदा & ॥\\
मधुमंगल सुबल सुबाहु भोज अर्जुन श्रीदामा & ।\\ \nopagebreak
मंडलि ग्वाल अनेक श्याम संगी बहु नामा & ॥\\
घोष निवासिनि की कृपा सुर नर बाँछित आदि अज & ।\\ \nopagebreak
बाल बृद्ध नर नारि गोप हौं अर्थी उन पाद रज & ॥
\end{longtable}
}
}
\fancyhead[LO,RE]{{\textmd{\Large २२: श्रीराधाकृष्णपरिकर}}}
\begin{sloppypar}\justifying\hyphenrules{nohyphenation}
\textbf{मूलार्थ}—नाभाजी कहते हैं कि (१)~गोपोंमें श्रेष्ठ \textbf{श्रीनन्दराज} (२)~\textbf{श्रीउपनन्द} (३)~\textbf{श्रीध्रुवनन्द} (४)~\textbf{श्रीधरानन्द} (५)~नन्दगोपकी पटरानी महरि \textbf{यशोदा} माता, और \textbf{कीरतिदा वृषभानु कुँवरि सहचरि मन मोदा} अर्थात् (६)~श्रीराधाजीकी माँ कीर्तिदा \textbf{कलावतीजी} (७)~राधाजीके पिता \textbf{श्रीवृषभानुजी} (८)~स्वयं \textbf{श्रीराधाजी} (९)~राधाजीकी मनमें प्रसन्न रहनेवाली \textbf{आठ सखियाँ} (१०)~\textbf{मधुमङ्गल} (११)~\textbf{सुबल} (१२)~\textbf{सुबाहु} (१३)~\textbf{भोज} (१४)~\textbf{अर्जुन} (१५)~\textbf{श्रीदामा} और इसी प्रकार (१६)~भगवान् कृष्णके साथ रहनेवाले अनेक नामवाले अनेक ग्वालोंके मण्डल—ऐसे जिन घोष\-निवासियोंकी कृपाको देवता, मनुष्य और किं बहुना \textbf{आदि अज} अर्थात् ब्रह्माजी भी चाहते रहते हैं उन्हीं बाल-वृद्ध गोप नर-नारियोंकी चरण\-धूलिका मैं अभ्यर्थी रहूँ।
\end{sloppypar}
\begin{sloppypar}\justifying\hyphenrules{nohyphenation}
नाभाजीने इस छन्दमें \textbf{वृषभानु कुँवरि सहचरि} कहा है। राधाजीकी मुख्य आठ सखियाँ प्रसिद्ध हैं। यहाँ ध्यान रखना चाहिये कि जैसे भगवती सीताजीकी आठ सखियाँ प्रसिद्ध हैं—(१)~चारुशीला (२)~लक्ष्मणा (३)~हेमा (४)~क्षेमा (५)~वरारोहा (६)~पद्मगन्धा (७)~सुलोचना और (८)~सुभगा, उसी प्रकार राधाजीकी भी आठ सखियाँ प्रसिद्ध हैं—(१)~ललिता (२)~विशाखा (३)~चित्रा (४)~इन्दुलेखा (५)~चम्पकलता (६)~रङ्गदेवी (७)~सुदेवी और (८)~तुङ्गविद्या। इन्हीं आठ सखियोंके आलोकमें राधाजीकी लीलाएँ चलती रहती हैं और इनमें ही भगवान्‌की लीलाके दर्शनसे मनमें आनन्द रहता है।
\end{sloppypar}

\addcontentsline{toc}{section}{\texorpdfstring{पद २३: श्रीकृष्णके अन्तरङ्ग सेवक}{२३: श्रीकृष्णके अन्तरङ्ग सेवक}}

{\relscale{1.1875}
{\bfseries
\setlength{\mylenone}{0pt}
\settowidth{\mylentwo}{}
\setlength{\mylenone}{\maxof{\mylenone}{\mylentwo}}
\settowidth{\mylentwo}{ब्रजराजसुवन सँग सदन मन अनुग सदा तत्पर रहैं}
\setlength{\mylenone}{\maxof{\mylenone}{\mylentwo}}
\settowidth{\mylentwo}{रक्तक पत्रक और पत्रि सबही मन भावैं}
\setlength{\mylenone}{\maxof{\mylenone}{\mylentwo}}
\settowidth{\mylentwo}{मधुकंठी मधुवर्त रसाल बिसाल सुहावैं}
\setlength{\mylenone}{\maxof{\mylenone}{\mylentwo}}
\settowidth{\mylentwo}{प्रेमकंद मकरंद सदा आनँद चँदहासा}
\setlength{\mylenone}{\maxof{\mylenone}{\mylentwo}}
\settowidth{\mylentwo}{पयद बकुल रसदान सारदा बुद्धि प्रकासा}
\setlength{\mylenone}{\maxof{\mylenone}{\mylentwo}}
\settowidth{\mylentwo}{सेवा समय बिचारिकै चारु चतुर चित की लहैं}
\setlength{\mylenone}{\maxof{\mylenone}{\mylentwo}}
\settowidth{\mylentwo}{ब्रजराजसुवन सँग सदन मन अनुग सदा तत्पर रहैं}
\setlength{\mylenone}{\maxof{\mylenone}{\mylentwo}}
\setlength{\mylentwo}{\baselineskip}
\setlength{\mylenone}{\mylenone + 1pt}
\setlength{\mylen}{(\textwidth - \mylenone)*\real{0.5}}
\begin{longtable}[l]{@{\hspace*{\mylen}}>{\setlength\parfillskip{0pt}}p{\mylenone}@{}@{}l@{}}
 & \\[-\the\mylentwo]
\centering{॥ २३ \hspace*{-1.5mm}॥} & \\ \nopagebreak
ब्रजराजसुवन सँग सदन मन अनुग सदा तत्पर रहैं & ॥\\
रक्तक पत्रक और पत्रि सबही मन भावैं & ।\\ \nopagebreak
मधुकंठी मधुवर्त रसाल बिसाल सुहावैं & ॥\\
प्रेमकंद मकरंद सदा आनँद चँदहासा & ।\\ \nopagebreak
पयद बकुल रसदान सारदा बुद्धि प्रकासा & ॥\\
सेवा समय बिचारिकै चारु चतुर चित की लहैं & ।\\ \nopagebreak
ब्रजराजसुवन सँग सदन मन अनुग सदा तत्पर रहैं & ॥
\end{longtable}
}
}
\fancyhead[LO,RE]{{\textmd{\Large २३: श्रीकृष्णके सेवक}}}
\begin{sloppypar}\justifying\hyphenrules{nohyphenation}
\textbf{मूलार्थ}—व्रजेन्द्रनन्दन श्रीकृष्णचन्द्रके मन और भवनमें साथ रहनेवाले सोलह ऐसे सेवक हैं जो देखनेमें सुन्दर हैं, सेवामें चतुर हैं, और चित्तकी आकाङ्क्षाओंको भी स्वीकार कर लेते हैं। वे सदैव भगवान्‌की सेवामें तत्पर रहते हैं। वे हैं—(१)~\textbf{रक्तक} (२)~\textbf{पत्रक} तथा (३)~\textbf{पत्री}—ये सबको भाते रहते हैं। इसी प्रकार (४)~\textbf{मधुकण्ठ} (५)~\textbf{मधुव्रत} (६)~\textbf{रसाल} तथा (७)~\textbf{विशाल}—ये सेवक बहुत सुन्दर लगते हैं। (८)~\textbf{प्रेमकन्द} (९)~\textbf{मकरन्द} (१०)~\textbf{सदानन्द} (११)~\textbf{चन्द्रहास} (१२)~\textbf{पयोद} (१३)~\textbf{बकुल} (१४)~\textbf{रसदान} (१५)~\textbf{शारदाप्रकाश} एवं (१६)~\textbf{बुद्धिप्रकाश}—ये सभी परिकर भगवान् श्रीकृष्णके मन और भवनमें साथ रहते हुए, प्रभुका अनुगमन करते हुए, सेवामें सदैव तत्पर रहते हैं और सेवाके समयका विचार करके सबके संबलकी रक्षा करते हुए, चतुराईपूर्वक भगवान्‌की रुचिको समझकर सेवा करते रहते हैं।\footnote{तुलना करें—\textbf{रसालसुविलासाश्च प्रेमकन्दो मरन्दकः॥ आनन्दश्चन्द्रहासश्च पयोदो वकुलस्तथा। रसदः शारदाद्याश्च व्रजस्था अनुगा मताः॥} (भ.र.सि. ३.२.४१-४२): संपादक।}
अब नाभाजी अन्य सप्तद्वीपीय वैष्णवोंकी चर्चा करते हैं। वे हैं—
\end{sloppypar}

\addcontentsline{toc}{section}{\texorpdfstring{पद २४: सप्तद्वीपके दास}{२४: सप्तद्वीपके दास}}

{\relscale{1.1875}
{\bfseries
\setlength{\mylenone}{0pt}
\settowidth{\mylentwo}{}
\setlength{\mylenone}{\maxof{\mylenone}{\mylentwo}}
\settowidth{\mylentwo}{सप्त द्वीप में दास जे ते मेरे सिरताज}
\setlength{\mylenone}{\maxof{\mylenone}{\mylentwo}}
\settowidth{\mylentwo}{जम्बुद्वीप अरु प्लच्छ सालमलि बहुत राजरिषि}
\setlength{\mylenone}{\maxof{\mylenone}{\mylentwo}}
\settowidth{\mylentwo}{कुस पबित्र पुनि क्रौंच कौन महिमा जाने लिखि}
\setlength{\mylenone}{\maxof{\mylenone}{\mylentwo}}
\settowidth{\mylentwo}{साक बिपुल बिस्तार प्रसिद्ध नामी अति पुहकर}
\setlength{\mylenone}{\maxof{\mylenone}{\mylentwo}}
\settowidth{\mylentwo}{पर्बत लोकालोक ओक टापू कंचनघर}
\setlength{\mylenone}{\maxof{\mylenone}{\mylentwo}}
\settowidth{\mylentwo}{हरिभृत्य बसत जे जे जहाँ तिन सन नित प्रति काज}
\setlength{\mylenone}{\maxof{\mylenone}{\mylentwo}}
\settowidth{\mylentwo}{सप्त द्वीप में दास जे ते मेरे सिरताज}
\setlength{\mylenone}{\maxof{\mylenone}{\mylentwo}}
\setlength{\mylentwo}{\baselineskip}
\setlength{\mylenone}{\mylenone + 1pt}
\setlength{\mylen}{(\textwidth - \mylenone)*\real{0.5}}
\begin{longtable}[l]{@{\hspace*{\mylen}}>{\setlength\parfillskip{0pt}}p{\mylenone}@{}@{}l@{}}
 & \\[-\the\mylentwo]
\centering{॥ २४ \hspace*{-1.5mm}॥} & \\ \nopagebreak
सप्त द्वीप में दास जे ते मेरे सिरताज & ॥\\
जम्बुद्वीप अरु प्लच्छ सालमलि बहुत राजरिषि & ।\\ \nopagebreak
कुस पबित्र पुनि क्रौंच कौन महिमा जाने लिखि & ॥\\
साक बिपुल बिस्तार प्रसिद्ध नामी अति पुहकर & ।\\ \nopagebreak
पर्बत लोकालोक ओक टापू कंचनघर & ॥\\
हरिभृत्य बसत जे जे जहाँ तिन सन नित प्रति काज & ।\\ \nopagebreak
सप्त द्वीप में दास जे ते मेरे सिरताज & ॥
\end{longtable}
}
}
\fancyhead[LO,RE]{{\textmd{\Large २४: सप्तद्वीपके दास}}}
\begin{sloppypar}\justifying\hyphenrules{nohyphenation}
\textbf{मूलार्थ}—सप्तद्वीपमें जितने वैष्णव भक्त हैं वे मेरे सिरके आभूषण हैं। जैसे (१)~\textbf{जम्बूद्वीप} (२)~\textbf{प्लक्षद्वीप} और (३)~\textbf{शाल्मलि\-द्वीप}में बहुत-से राजर्षि हैं। इसी प्रकार पवित्र (४)~\textbf{कुशद्वीप} और (५)~\textbf{क्रौञ्चद्वीप}में भी इतने बड़े राजर्षि हैं कि जिनकी महिमाको लिखकर कौन जान सकता है, अथवा जिनकी महिमाके लेशको भी कौन जान सकता है? (६)~विपुल विस्तारवाले \textbf{शाकद्वीप} और (७)~प्रसिद्ध नामवाले \textbf{पुष्करद्वीप}में भी अनेक भगवद्भक्त हैं। लोकालोक पर्वत, जो टापू अर्थात् द्वीपका स्थान है और जो \textbf{कंचनघर} अर्थात् स्वर्णका घर है, वहाँ भी बहुत-से भगवद्भक्त विराजते हैं। भगवान्‌के जो-जो भक्त जहाँ-जहाँ बसते हैं, उनसे नित्य प्रति मेरा कोई-न-कोई प्रयोजन रहता है। अतः सप्तद्वीपके जो भक्त हैं, वे सब-के-सब मेरे सिरके ताज अर्थात् सिरके मुकुटमणि हैं। इनकी मैं उपेक्षा नहीं कर सकता।
\end{sloppypar}

\addcontentsline{toc}{section}{\texorpdfstring{पद २५: जम्बूद्वीपके भक्त}{२५: जम्बूद्वीपके भक्त}}

{\relscale{1.1875}
{\bfseries
\setlength{\mylenone}{0pt}
\settowidth{\mylentwo}{}
\setlength{\mylenone}{\maxof{\mylenone}{\mylentwo}}
\settowidth{\mylentwo}{मध्यदीप नवखंड में भक्त जिते मम भूप}
\setlength{\mylenone}{\maxof{\mylenone}{\mylentwo}}
\settowidth{\mylentwo}{इलाबर्त आधीस संकरषन अनुग सदाशिव}
\setlength{\mylenone}{\maxof{\mylenone}{\mylentwo}}
\settowidth{\mylentwo}{रमनक मछ मनु दास हिरन्य कूर्म अर्जम इव}
\setlength{\mylenone}{\maxof{\mylenone}{\mylentwo}}
\settowidth{\mylentwo}{कुरु बराह भूभृत्य बर्ष हरिसिंह प्रहलादा}
\setlength{\mylenone}{\maxof{\mylenone}{\mylentwo}}
\settowidth{\mylentwo}{किंपुरुष राम कपि भरत नरायन बीनानादा}
\setlength{\mylenone}{\maxof{\mylenone}{\mylentwo}}
\settowidth{\mylentwo}{भद्राश्व ग्रीवहय भद्रश्रव केतु काम कमला अनूप}
\setlength{\mylenone}{\maxof{\mylenone}{\mylentwo}}
\settowidth{\mylentwo}{मध्यदीप नवखंड में भक्त जिते मम भूप}
\setlength{\mylenone}{\maxof{\mylenone}{\mylentwo}}
\setlength{\mylentwo}{\baselineskip}
\setlength{\mylenone}{\mylenone + 1pt}
\setlength{\mylen}{(\textwidth - \mylenone)*\real{0.5}}
\begin{longtable}[l]{@{\hspace*{\mylen}}>{\setlength\parfillskip{0pt}}p{\mylenone}@{}@{}l@{}}
 & \\[-\the\mylentwo]
\centering{॥ २५ \hspace*{-1.5mm}॥} & \\ \nopagebreak
मध्यदीप नवखंड में भक्त जिते मम भूप & ॥\\
इलाबर्त आधीस संकरषन अनुग सदाशिव & ।\\ \nopagebreak
रमनक मछ मनु दास हिरन्य कूर्म अर्जम इव & ॥\\
कुरु बराह भूभृत्य बर्ष हरिसिंह प्रहलादा & ।\\ \nopagebreak
किंपुरुष राम कपि भरत नरायन बीनानादा & ॥\\
भद्राश्व ग्रीवहय भद्रश्रव केतु काम कमला अनूप & ।\\ \nopagebreak
मध्यदीप नवखंड में भक्त जिते मम भूप & ॥
\end{longtable}
}
}
\fancyhead[LO,RE]{{\textmd{\Large २५: जम्बूद्वीपके भक्त}}}
\begin{sloppypar}\justifying\hyphenrules{nohyphenation}
\textbf{मूलार्थ}—मध्यद्वीप अर्थात् जम्बूद्वीपके नवों खण्डोंमें जो भक्त हैं, वे मेरे राजा हैं। यहाँ प्रत्येक खण्डके नाम, उनके अधीश्वर भगवान्‌के नाम, और उनके प्रसिद्ध परिकर भक्तके नामका वर्णन है। जैसे—(१)~\textbf{इलावर्त} नामक खण्डके अधीश्वर भगवान् \textbf{संकर्षण} हैं, उनके अनुगामी \textbf{सदाशिव} हैं। (२)~इसी प्रकार \textbf{रमणक\-खण्ड}के अधीश्वर भगवान् \textbf{मत्स्य} हैं, और उनके भक्त \textbf{मनु} अर्थात् वैवस्वत मनु हैं, जिन्हें सत्यव्रत भी कहते हैं (भागवत ८.२४.५८के अनुसार चाक्षुष मन्वन्तरके राजर्षि सत्यव्रत ही इस वैवस्वत मन्वन्तरके मनु हैं)। (३)~\textbf{हिरण्यखण्ड}के ईश्वर भगवान् \textbf{कूर्म} हैं, और उनके सेवक \textbf{अर्यमा} हैं। (४)~इसी प्रकार \textbf{कुरुखण्ड}के अधीश्वर हैं भगवान् \textbf{वराह} और उनकी सेविका हैं \textbf{भूदेवी}। (५)~इसी प्रकार \textbf{हरिवर्ष\-खण्ड}के ईश्वर हैं भगवान् \textbf{नृसिंह} और उनके सेवक हैं परम भागवत \textbf{प्रह्लाद}। (६)~\textbf{किंपुरुष\-खण्ड}के अधीश्वर हैं भगवान् \textbf{राम} और उनके सेवक हैं \textbf{श्रीहनुमान्‌जी} महाराज। (७)~\textbf{भरतखण्ड}के अधीश्वर हैं \textbf{नारायण} और उनके सेवक हैं वीणापाणि \textbf{नारद}। (८)~\textbf{भद्राश्वखण्ड}के अधीश्वर हैं \textbf{हयग्रीव} और उनके सेवक हैं \textbf{भद्रश्रवा}। (९)~\textbf{केतुमालखण्ड}के अधीश्वर हैं भगवान् \textbf{कामेश्वर} और उनकी सेविका हैं अनुपमेय \textbf{कमला} अर्थात् लक्ष्मीजी।
\end{sloppypar}

\addcontentsline{toc}{section}{\texorpdfstring{पद २६: श्वेतद्वीपके भक्त}{२६: श्वेतद्वीपके भक्त}}

{\relscale{1.1875}
{\bfseries
\setlength{\mylenone}{0pt}
\settowidth{\mylentwo}{}
\setlength{\mylenone}{\maxof{\mylenone}{\mylentwo}}
\settowidth{\mylentwo}{श्वेतद्वीप के दास जे श्रवन सुनो तिनकी कथा}
\setlength{\mylenone}{\maxof{\mylenone}{\mylentwo}}
\settowidth{\mylentwo}{श्रीनारायन बदन निरंतर ताही देखैं}
\setlength{\mylenone}{\maxof{\mylenone}{\mylentwo}}
\settowidth{\mylentwo}{पलक परै जो बीच कोटि जमजातन लेखैं}
\setlength{\mylenone}{\maxof{\mylenone}{\mylentwo}}
\settowidth{\mylentwo}{तिनके दरसन काज गए तहँ बीनाधारी}
\setlength{\mylenone}{\maxof{\mylenone}{\mylentwo}}
\settowidth{\mylentwo}{श्याम दई कर सैन उलटि अब नहिं अधिकारी}
\setlength{\mylenone}{\maxof{\mylenone}{\mylentwo}}
\settowidth{\mylentwo}{नारायन आख्यान दृढ़ तहँ प्रसंग नाहिन तथा}
\setlength{\mylenone}{\maxof{\mylenone}{\mylentwo}}
\settowidth{\mylentwo}{श्वेतद्वीप के दास जे श्रवन सुनो तिनकी कथा}
\setlength{\mylenone}{\maxof{\mylenone}{\mylentwo}}
\setlength{\mylentwo}{\baselineskip}
\setlength{\mylenone}{\mylenone + 1pt}
\setlength{\mylen}{(\textwidth - \mylenone)*\real{0.5}}
\begin{longtable}[l]{@{\hspace*{\mylen}}>{\setlength\parfillskip{0pt}}p{\mylenone}@{}@{}l@{}}
 & \\[-\the\mylentwo]
\centering{॥ २६ \hspace*{-1.5mm}॥} & \\ \nopagebreak
श्वेतद्वीप के दास जे श्रवन सुनो तिनकी कथा & ॥\\
श्रीनारायन बदन निरंतर ताही देखैं & ।\\ \nopagebreak
पलक परै जो बीच कोटि जमजातन लेखैं & ॥\\
तिनके दरसन काज गए तहँ बीनाधारी & ।\\ \nopagebreak
श्याम दई कर सैन उलटि अब नहिं अधिकारी & ॥\\
नारायन आख्यान दृढ़ तहँ प्रसंग नाहिन तथा & ।\\ \nopagebreak
श्वेतद्वीप के दास जे श्रवन सुनो तिनकी कथा & ॥
\end{longtable}
}
}
\fancyhead[LO,RE]{{\textmd{\Large २६: श्वेतद्वीपके भक्त}}}
\begin{sloppypar}\justifying\hyphenrules{nohyphenation}
\textbf{मूलार्थ}—श्वेतद्वीपके जो भक्त हैं, उनकी कथा कानसे सुनिये। वे श्रीनारायणके मुखकमलको निरन्तर निहारते रहते हैं। एक भी पलक पड़ने भरका जब अन्तर पड़ता है तो उन्हें करोड़ों यम\-यातनाओंके समान कष्ट होता है। एक बार श्वेतद्वीपके भक्तोंका दर्शन करनेके लिये वीणापाणि नारदजी वहाँ गए। श्वेतद्वीपके भक्त निरन्तर भगवान्‌को निहारनेमें मग्न थे। भगवान्‌ने नेत्रका संकेत देकर नारदजीसे कहा—“लौट आओ, वहाँ कोई तुम्हारे ज्ञानका अधिकारी नहीं है।” जिस प्रकार अन्यत्र नारायणका आख्यान होता है, वह प्रसंग वहाँ नहीं है अर्थात् वहाँ कोई सुनेगा ही नहीं।
\end{sloppypar}

\addcontentsline{toc}{section}{\texorpdfstring{पद २७: अष्ट द्वारपाल सर्प}{२७: अष्ट द्वारपाल सर्प}}

{\relscale{1.1875}
{\bfseries
\setlength{\mylenone}{0pt}
\settowidth{\mylentwo}{}
\setlength{\mylenone}{\maxof{\mylenone}{\mylentwo}}
\settowidth{\mylentwo}{उरग अष्टकुल द्वारपति सावधान हरिधाम थिति}
\setlength{\mylenone}{\maxof{\mylenone}{\mylentwo}}
\settowidth{\mylentwo}{इलापत्र मुख अनँत अनँत कीरति बिस्तारत}
\setlength{\mylenone}{\maxof{\mylenone}{\mylentwo}}
\settowidth{\mylentwo}{पद्म संकु पन प्रगट ध्यान उरते नहीं टारत}
\setlength{\mylenone}{\maxof{\mylenone}{\mylentwo}}
\settowidth{\mylentwo}{अँशुकंबल बासुकी अजित आग्या अनुबरती}
\setlength{\mylenone}{\maxof{\mylenone}{\mylentwo}}
\settowidth{\mylentwo}{करकोटक तच्छक सुभट्ट सेवा सिर धरती}
\setlength{\mylenone}{\maxof{\mylenone}{\mylentwo}}
\settowidth{\mylentwo}{आगमोक्त शिवसंहिता अगर एकरस भजन रति}
\setlength{\mylenone}{\maxof{\mylenone}{\mylentwo}}
\settowidth{\mylentwo}{उरग अष्टकुल द्वारपति सावधान हरिधाम थिति}
\setlength{\mylenone}{\maxof{\mylenone}{\mylentwo}}
\setlength{\mylentwo}{\baselineskip}
\setlength{\mylenone}{\mylenone + 1pt}
\setlength{\mylen}{(\textwidth - \mylenone)*\real{0.5}}
\begin{longtable}[l]{@{\hspace*{\mylen}}>{\setlength\parfillskip{0pt}}p{\mylenone}@{}@{}l@{}}
 & \\[-\the\mylentwo]
\centering{॥ २७ \hspace*{-1.5mm}॥} & \\ \nopagebreak
उरग अष्टकुल द्वारपति सावधान हरिधाम थिति & ॥\\
इलापत्र मुख अनँत अनँत कीरति बिस्तारत & ।\\ \nopagebreak
पद्म संकु पन प्रगट ध्यान उरते नहीं टारत & ॥\\
अँशुकंबल बासुकी अजित आग्या अनुबरती & ।\\ \nopagebreak
करकोटक तच्छक सुभट्ट सेवा सिर धरती & ॥\\
आगमोक्त शिवसंहिता अगर एकरस भजन रति & ।\\ \nopagebreak
उरग अष्टकुल द्वारपति सावधान हरिधाम थिति & ॥
\end{longtable}
}
}
\fancyhead[LO,RE]{{\textmd{\Large २७: अष्ट द्वारपाल सर्प}}}
\begin{sloppypar}\justifying\hyphenrules{nohyphenation}
\textbf{मूलार्थ}—भगवान् श्रीरामके साकेतके आठ श्रेष्ठ सर्प द्वारपाल हैं, जो सावधान होकर भगवान्‌के साकेत धाममें स्थित रहते हैं। उनके नाम हैं—(१)~\textbf{इलापत्र} (२)~\textbf{अनन्त} (३)~\textbf{पद्म} (४)~\textbf{शङ्कु} (५)~\textbf{अंशुकम्बल} (६)~\textbf{वासुकि} (७)~\textbf{कर्कोटक} और (८)~\textbf{तक्षक}। इनमेंसे इलापत्र और अनन्त—ये अनन्त मुखोंसे भगवान्‌की कीर्तिका विस्तार करते रहते हैं। पद्म और शङ्कु—इनका प्रण प्रकट है, ये अपने मनसे भगवान्‌के ध्यानको कभी नहीं दूर करते। अंशुकम्बल और वासुकि—ये निरन्तर अजित भगवान् श्रीरामकी आज्ञाका अनुवर्तन करते रहते हैं। कर्कोटक और तक्षक—ये वीर सेवा रूप पृथ्वीको अपने सिरपर धारण किये रहते हैं। श्रीअग्रदासजी कहते हैं कि ये आठों आगमोक्त शिवसंहिता अर्थात् अहिर्बुध्न्य\-संहिताके अनुसार भगवान्‌की भक्तिमें एकरस निमग्न रहते हैं।\nopagebreak\\
\end{sloppypar}
\centering{\textbf{॥ श्रीः ॥\nopagebreak\\}}
\centering{\textbf{॥ समस्त भक्तोंकी जय हो ॥}}
\paraseplotus
